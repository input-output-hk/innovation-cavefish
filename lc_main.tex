
\documentclass[a4paper,USenglish,cleveref, autoref,anonymous, thm-restate]{lipics-v2021}
%This is a template for producing LIPIcs articles. 
%See lipics-v2021-authors-guidelines.pdf for further information.
%for A4 paper format use option "a4paper", for US-letter use option "letterpaper"
%for british hyphenation rules use option "UKenglish", for american hyphenation rules use option "USenglish"
%for section-numbered lemmas etc., use "numberwithinsect"
%for enabling cleveref support, use "cleveref"
%for enabling autoref support, use "autoref"
%for anonymousing the authors (e.g. for double-blind review), add "anonymous"
%for enabling thm-restate support, use "thm-restate"
%for enabling a two-column layout for the author/affilation part (only applicable for > 6 authors), use "authorcolumns"
%for producing a PDF according the PDF/A standard, add "pdfa"

%\pdfoutput=1 %uncomment to ensure pdflatex processing (mandatatory e.g. to submit to arXiv)
%\hideLIPIcs  %uncomment to remove references to LIPIcs series (logo, DOI, ...), e.g. when preparing a pre-final version to be uploaded to arXiv or another public repository


\usepackage{stmaryrd}
\usepackage{bm}
\usepackage{microtype}
\usepackage{amsmath}
\usepackage{tikz}
\usetikzlibrary{matrix,shapes,arrows,positioning,chains,calc,fit}
\usepackage{xcolor}
\usepackage{booktabs}
\usepackage{multirow}
\usepackage[normalem]{ulem}
\usepackage{amsthm}
\usepackage[subtle]{savetrees} 

\theoremstyle{definition} % For upright text
\newtheorem{assumption}{Assumption}


%\graphicspath{{./graphics/}}%helpful if your graphic files are in another directory

\bibliographystyle{plainurl}% the mandatory bibstyle

\title{Cavefish: Communication-Optimal Light Client Protocol for UTxO Ledgers}

\titlerunning{Cavefish} %TODO optional, please use if title is longer than one line

\author{Jane {Open Access}}{Dummy University Computing Laboratory, [optional: Address], Country \and My second affiliation, Country \and \url{http://www.myhomepage.edu} }{johnqpublic@dummyuni.org}{https://orcid.org/0000-0002-1825-0097}{(Optional) author-specific funding acknowledgements}%TODO mandatory, please use full name; only 1 author per \author macro; first two parameters are mandatory, other parameters can be empty. Please provide at least the name of the affiliation and the country. The full address is optional. Use additional curly braces to indicate the correct name splitting when the last name consists of multiple name parts.


\authorrunning{J. Open Access and J.\,R. Public} %TODO mandatory. First: Use abbreviated first/middle names. Second (only in severe cases): Use first author plus 'et al.'

\Copyright{Jane Open Access and Joan R. Public} %TODO mandatory, please use full first names. LIPIcs license is "CC-BY";  http://creativecommons.org/licenses/by/3.0/

\begin{CCSXML}
<ccs2012>
   <concept>
       <concept_id>10002978.10003014.10003015</concept_id>
       <concept_desc>Security and privacy~Security protocols</concept_desc>
       <concept_significance>500</concept_significance>
       </concept>
   <concept>
       <concept_id>10003752.10010070.10010099.10010103</concept_id>
       <concept_desc>Theory of computation~Exact and approximate computation of equilibria</concept_desc>
       <concept_significance>500</concept_significance>
       </concept>
   <concept>
       <concept_id>10002950.10003624.10003633.10003643</concept_id>
       <concept_desc>Mathematics of computing~Graphs and surfaces</concept_desc>
       <concept_significance>300</concept_significance>
       </concept>
 </ccs2012>
\end{CCSXML}

\ccsdesc[500]{Security and privacy~Security protocols}

\keywords{light clients, protocols, wallets, blockchain} %TODO mandatory; please add comma-separated list of keywords

%\category{} %optional, e.g. invited paper

%\relatedversion{} %optional, e.g. full version hosted on arXiv, HAL, or other respository/website
%\relatedversiondetails[linktext={opt. text shown instead of the URL}, cite=DBLP:books/mk/GrayR93]{Classification (e.g. Full Version, Extended Version, Previous Version}{URL to related version} %linktext and cite are optional

%\supplement{}%optional, e.g. related research data, source code, ... hosted on a repository like zenodo, figshare, GitHub, ...
%\supplementdetails[linktext={opt. text shown instead of the URL}, cite=DBLP:books/mk/GrayR93, subcategory={Description, Subcategory}, swhid={Software Heritage Identifier}]{General Classification (e.g. Software, Dataset, Model, ...)}{URL to related version} %linktext, cite, and subcategory are optional

%\funding{(Optional) general funding statement \dots}%optional, to capture a funding statement, which applies to all authors. Please enter author specific funding statements as fifth argument of the \author macro.


%\nolinenumbers %uncomment to disable line numbering



%Editor-only macros:: begin (do not touch as author)%%%%%%%%%%%%%%%%%%%%%%%%%%%%%%%%%%
\EventEditors{John Q. Open and Joan R. Access}
\EventNoEds{2}
\EventLongTitle{42nd Conference on Very Important Topics (CVIT 2016)}
\EventShortTitle{CVIT 2016}
\EventAcronym{CVIT}
\EventYear{2016}
\EventDate{December 24--27, 2016}
\EventLocation{Little Whinging, United Kingdom}
\EventLogo{}
\SeriesVolume{42}
\ArticleNo{23}
%%%%%%%%%%%%%%%%%%%%%%%%%%%%%%%%%%%%%%%%%%%%%%%%%%%%%%


%formatting
\newcommand{\var}[1]{\ensuremath{\mathit{#1}}}
\newcommand{\fun}[1]{\ensuremath{\mathsf{#1}}}
\newcommand{\type}[1]{\ensuremath{\mathsf{#1}}}
\renewcommand\H{\mathbb{H}}
\newcommand\A{\ensuremath{\mathcal{A}}}

\newcommand\BLD{\ensuremath{\mathsf{BLD}}}
\newcommand\WBPS{\ensuremath{\mathsf{WBPS}}}
\newcommand\WBPSP{\ensuremath{\mathsf{WBPS}[P]}}


\newcommand\hash{\ensuremath{\mathsf{H}}}
\newcommand\HGen{\ensuremath{\mathsf{HGen}}}

\newcommand\Setup{\ensuremath{\mathsf{Setup}}}
\newcommand\Keygen{\ensuremath{\mathsf{KeyGen}}}
\newcommand\Ver{\ensuremath{\mathsf{Ver}}}
\newcommand\Sign{\ensuremath{\mathsf{Sign}}}
\newcommand\Request{\ensuremath{\mathsf{Request}}}

\newcommand\GGen{\ensuremath{\mathsf{GGen}}}
\newcommand\NArg{\ensuremath{\mathsf{NArg}}}
\newcommand\NArgr{\ensuremath{\mathsf{NArg}[R]}}

\newcommand\Prove{\ensuremath{\mathsf{Prove}}}
\newcommand\SimProve{\ensuremath{\mathsf{SimProve}}}

\newcommand\PKE{\ensuremath{\mathsf{PKE}}}
\newcommand\SDS{\ensuremath{\mathsf{SDS}}}
\newcommand\DS{\ensuremath{\mathsf{DS}}}
\newcommand\Enc{\ensuremath{\mathsf{Enc}}}
\newcommand\Dec{\ensuremath{\mathsf{Dec}}}


\newcommand{\Interval}[1]{\type{Interval}[#1]}
\newcommand{\nextdef}{\ensuremath{\\[1em]}}
\newcommand{\where}{\ensuremath{~ ~ \mathbf{where}~ ~ }}

\newcommand{\todobox}[1]{ \fcolorbox{black}{gray!50} {\textbf{TODO:} \parbox{.7\textwidth}{#1}}}
\newcommand{\cut}[1]{}
%\newcommand{\cut}[1]{ \textcolor{orange}{#1}}


% Macros for eutxo things.
\newcommand{\tx}{\fun{tx}}
\newcommand{\TxId}{\type{TxId}}
\newcommand{\TxIn}{\type{TxIn}}
\newcommand{\Header}{\type{Header}}
\newcommand{\Block}{\type{Block}}
\newcommand{\LState}{\type{LState}}
\newcommand{\Tx}{\type{Tx}}
\newcommand{\Slot}{\type{Slot}}
\newcommand{\TxAbs}{\type{TxAbs}}
\newcommand{\B}{\{0,1\}}
\newcommand{\TxOut}{\type{TxOut}}
\newcommand{\UTxO}{\type{UTxO}}
\newcommand{\txId}{\msf{txId}}
\newcommand{\txrefid}{\fun{id}}
\newcommand{\Address}{\ensuremath{\s{Address}}}
\newcommand{\DataHash}{\ensuremath{\s{DataHash}}}
\newcommand{\hashData}{\fun{dataHash}}
\newcommand{\idx}{\fun{index}}
\newcommand{\inputs}{\fun{inputs}}
\newcommand{\outputs}{\fun{outputs}}
\newcommand{\Out}{\type{Output}}
\newcommand{\validityInterval}{\fun{validityInterval}}
\newcommand{\scripts}{\fun{scripts}}
\newcommand{\mint}{\fun{mint}}
\newcommand{\mintScripts}{\fun{mintScripts}}
\newcommand{\mintScsRdmrs}{\fun{mintRdmrs}}
\newcommand{\mintRdmrs}{\fun{mintRdmrs}}
\newcommand{\sigs}{\fun{sigs}}
\newcommand{\fee}{\fun{fee}}
\newcommand{\addr}{\fun{addr}}
\newcommand{\pubkey}{\fun{PubKey}}
\newcommand{\privkey}{\fun{PrivKey}}
\newcommand{\val}{\fun{value}}  %% \value is already defined
\newcommand{\Value}{\type{Value}}
\newcommand{\Redeemer}{\type{Redeemer}}
\newcommand{\TxOutRef}{\type{TxIn}}
\newcommand{\ScriptContext}{\type{ScriptContext}}
\newcommand{\ScriptPurpose}{\type{ScriptPurpose}}
\newcommand{\Datum}{\type{Datum}}
\newcommand{\DCert}{\type{DCert}}
\newcommand{\LCTx}{\type{LCTx}}
\newcommand{\TxInInfo}{\type{TxInInfo}}


\newcommand{\validator}{\fun{validator}}
\newcommand{\redeemer}{\fun{redeemer}}
\newcommand{\datum}{\fun{datum}}
\newcommand{\datumHash}{\fun{datumHash}}
\newcommand{\datumWits}{\fun{datumWitnesses}}
\newcommand{\Data}{\ensuremath{\s{Data}}}
\newcommand{\Input}{\ensuremath{\s{Input}}}
\newcommand{\Output}{\type{Output}}
\newcommand{\OutputRef}{\fun{OutputRef}}
\newcommand{\Signature}{\type{Signature}}
\newcommand{\Ledger}{\ensuremath{\s{Ledger}}}

\newcommand{\outputref}{\fun{outputRef}}
\newcommand{\outputrefs}{\fun{outputRefs}}
\newcommand{\txin}{\fun{in}}
\newcommand{\id}{\fun{id}}
\newcommand{\lookupTx}{\msf{lookupTx}}
\newcommand{\getSpent}{\msf{getSpentOutput}}

% \newcommand{\Tick}{\ensuremath{\s{Tick}}}
\newcommand{\Tick}{\type{Slot}}
\newcommand{\Script}{\type{Script}}
\newcommand{\spent}{\msf{spentOutputs}}
\newcommand{\unspent}{\msf{unspentOutputs}}
\newcommand{\txunspent}{\msf{unspentTxOutputs}}
\newcommand{\eutxotx}{\msf{Tx}}

\newcommand{\consumes}[1]{\msf{consumes(#1)}}
\newcommand{\consumesOne}[1]{\msf{consumesOne(#1)}}
\newcommand{\cid}{\fun{cid}}
\newcommand{\inputValue}{\fun{inputValue}}
\newcommand{\rMin}{r_{\fun{min}}}
\newcommand{\rMax}{r_{\fun{max}}}

\newcommand{\utxotx}{\msf{Tx}}

\newcommand{\N}{\mathbb{N}}
\newcommand{\Z}{\mathbb{Z}}
\newcommand{\G}{\mathbb{G}}
\newcommand{\Lang}{\mathcal{L}}
\newcommand{\crs}{\text{crs}}

\newcommand{\nt}{\ensuremath{\textsf{aux}_\textsf{nt}}}
\newcommand{\notelength}{\kappa}


\newcommand{\Quantity}{\type{Quantity}}
\newcommand{\TokenName}{\type{TokenName}}
\newcommand{\Asset}{\ensuremath{\s{Asset}}}
\newcommand{\AssetID}{\type{AssetID}}
\newcommand{\Policy}{\type{Policy}}
\newcommand{\Quantities}{\ensuremath{\s{Quantities}}}
\newcommand{\nativeCur}{\ensuremath{\mathrm{nativeC}}}
\newcommand{\nativeTok}{\ensuremath{\mathrm{nativeT}}}
\newcommand{\valC}{\mkValidator{\mathcal{C}}}
\newcommand{\polC}{\mkPolicy{\mathcal{C}}}
\newcommand\mkValidator[1]{\msf{validator}_#1}
\newcommand\mkPolicy[1]{\msf{policy}_#1}

\newcommand{\PublicKey}{\ensuremath{\s{PubKey}}}
\newcommand{\PubKey}{\ensuremath{\s{PubKey}}}
\newcommand{\PrivateKey}{\ensuremath{\s{PrivateKey}}}

\newcommand{\mkContext}{\ensuremath{\s{mkContext}}}
\newcommand{\mkTxInfo}{\ensuremath{\s{mkTxInfo}}}
\newcommand{\mkVlContext}{\ensuremath{\s{mkValidatorContext}}}
\newcommand{\mkMpsContext}{\ensuremath{\s{mkPolicyContext}}}
\newcommand{\checkSig}{\type{checkSig}}

\newcommand{\applyScript}[1]{\ensuremath{\llbracket#1\rrbracket}}
\newcommand{\chkSpec}[1]{\ensuremath{\llbracket#1\rrbracket}_{\type{DSL}}}
\newcommand{\applyMPScript}[1]{\ensuremath{\llbracket#1\rrbracket}}

\newcommand{\true}{\type{1}}
\newcommand{\false}{\type{0}}
\newcommand{\True}{\type{1}}
\newcommand{\False}{\type{0}}

\newcommand{\leteq}{:=}

\begin{document}


% !TEX root = ../lc_main.tex

\maketitle

\begin{abstract}
Blockchain light clients (LCs) are agents with limited computational or storage resources who cannot maintain a fully validated local copy of the ledger state. Instead, they rely on service providers (SPs), typically full nodes, to access data required for tasks such as constructing transactions or interacting with off-chain applications.

In this work, we introduce Cavefish, a novel protocol for UTxO-based platforms that enables LCs to interact with the ledger and submit transactions with minimal trust, storage, and computation. Cavefish defines a two-party computation protocol between an LC and an SP, in which the LC specifies a transaction and the SP constructs it. Consequently, the LC only receives a blinded version of the transaction, preventing it from modifying or reusing the transaction while still being able to verify that the transaction matches the original intent of the LC. The SP is compensated inside the constructed transaction, eliminating the need for another protocol or exchange.

To support this, we propose a variant of the predicate blind signature (PBS) scheme of Fuchsbauer and Wolf (Eurocrypt 2024), allowing the SP to obtain a valid signature on the unblinded transaction, which it can then broadcast on the network and post on chain. Moreover, the resulting signatures verify as standard Schnorr signatures.
Our construction achieves a trustless interaction in which the LC achieves their transaction goal, and the SP receives fair compensation for their effort.
When Cavefish is combined with hierarchical deterministic (HD) wallets, the LC can provide a single public key and chain code to the SP, reducing communication footprint to a minimum.

To further optimize communication and computational overhead, our PBS variant relaxes the unlinkability guarantees of traditional blind signatures in favor of efficiency. 
We argue that this relaxation is adequate, since transactions only need to be kept private until posted on a public ledger.
We implement and benchmark the Non-interactive Argument of Knowledge (NArg) component of our protocol on two major UTxO-based blockchains. Despite being the most computationally demanding part, our results show that proving and verification times, as well as circuit sizes, are practical for real-world deployment.

%Blockchain light clients (LCs) are users with limited resources who cannot maintain a fully validated local copy of the ledger state. Consequently, they rely on service providers (SPs), typically full nodes, to access the data needed for tasks such as transaction construction or interacting with off-chain applications.
%In this work, we propose a novel protocol and model for UTxO-based platforms that enables LCs to interact with the ledger and submit transactions with minimal local storage and computation. Our solution is called Cavefish and defines a two-party computation protocol between an LC and an SP. The LC instructs the SP to create a transaction according to its specifications, but only a blinded version of the result is shared. The blinded form prevents the LC from altering the transaction or constructing a new (valid) transaction, while still allowing the LC to verify that the transaction satisfies their original specification.
%To achieve this, we introduce a secure predicate mechanism and a weakly blind signature scheme. This allows the SP to obtain a valid signature on the original (unblinded) transaction, which can then be submitted to the network. The result is a trustless interaction in which the LC achieves their transaction goal, and the SP receives compensation for their effort.
%To optimize communication and computational overhead, we design an extension of Schnorr blind signatures called weakly blind predicate signatures, which relaxes the unlinkability requirement in standard blind signatures.
%We implement and benchmark the Non-interactive Argument of Knowledge component of our protocol on two major UTxO-based blockchain platforms. Despite this component being the most computationally intensive part, our evaluation demonstrates that verification and proving times, as well as associated circuit sizes, are well within practical bounds for real-world deployment.

%\textcolor{orange}{play up communication optimality: LC does not read blockchain. rationality }


%Blockchain light clients (LCs) are platform users that do not have the capacity to locally maintain current,
%fully-validated ledger state. For this reason, light clients rely on service providers (SPs, which are full nodes) for obtaining the blockchain data they require, e.g. for transaction construction or running off-chain apps.
%In this paper, we propose protocol and model in which a light client of a UTxO-style platform has the goal of submitting a transaction with minimal local state and minimal local processing. We propose a 2-party computation protocol between the LC and the SP. First, the light client creates a specification for the transaction they would like the SP to construct, then the SP builds such a transaction, while including a small payment to cover the cost of computation.
%The SP sends the LC a blinded version of the constructed transaction for signing and checking. It is modified such that it is impossible for the LC to use the provided data to construct a distinct new (valid) transaction without guessing pre-images of transaction hashes, but has enough information to allow the LC to
%check that it meets their specification. Using a secure predicate and partially blind signature scheme, the SP is able to obtain a signature on the original (unmodified) transaction,
%which is submitted to the network for inclusion in a block. This 2-PC protocol constitutes a trustless
%interaction between the LC and the SP resulting in the LC's desired transaction being applied
%to the ledger state, and the SP receiving compensation for their work.

%In order to realize our light client protocol in a communication optimal way, we develop an adaptation of Secure Blind Schnorr Signatures that we call Weakly Blind Predicate Signatures, alleviating the unlinkability requirement of standard Blind Schnorr Signatures.

%Finally, we implement and benchmark the Non-interactive Argument of Knowledge as part of our protocol and constitutes the most time and resource intensive component of our construction.
\end{abstract}



\section{Introduction}
Blockchain technologies have emerged as a foundational component of decentralized systems, offering strong guarantees of data integrity, censorship resistance, and fault tolerance through cryptographic protocols and distributed consensus. Within this domain, the Unspent Transaction Output (UTxO) model represents a distinctive paradigm for managing asset ownership and validating transactions. 
The UTxO model was initially introduced by Bitcoin and subsequently adopted by other platforms such as Cardano.
In contrast to account-based models, UTxO-based blockchains accommodate parallelism and concurrent processing more effectively
%and improve auditability, 
but also introduce challenges in terms of complexity and
client verification.

Full nodes in a UTxO-based blockchain are required to download and validate the entire chain history to ensure correctness and security. This requirement presents a significant barrier to participation for resource-constrained devices, such as smartphones and embedded systems. Light client (LC) protocols aim to mitigate this issue by enabling nodes to interact with the blockchain in a secure and efficient manner without maintaining full historical data. These protocols must strike a careful balance between minimizing resource consumption and preserving critical security properties, such as transaction inclusion, double-spending resistance, and above all, chain validity.

Blockchains are append-only data structures that grow continuously over time. As the chain length increases, it becomes prohibitively expensive for a light client (LC) to scan the entire history to verify past transactions or to locate a specific UTxO. \cut{In UTxO-based systems, where each transaction consumes and produces discrete outputs without a centralized account state, the ability to efficiently access historical transaction data becomes essential. Without mechanisms to support succinct historical queries or proofs of inclusion, LCs may be forced to rely on third-party services or sacrifice security.}

The question answered by this paper is ``how can a user of a light client engage with a full node acting as a service provider to request and approve transactions (e.g. from their wallet) in a secure  way without knowing anything about the current chain and ledger state and minimal communication effort?'' 
In this paper, we present a solution to this question in the form of \emph{Cavefish}, a novel intent-based light client protocol designed for UTxO-based blockchains.
Cavefish enables LCs to avoid querying the current ledger state and submit transactions requiring only minimal local storage and computation. In order to submit a transaction on chain, the LC engages with a service provider (SP) in a two-party computation protocol that yields a signed transaction indistinguishable from one created by a full node. In addition to the low storage and computational requirements, our LC protocol is communication-optimal. After the LC has instructed the SP about the type of transaction it wishes to create, the protocol can be completed in as few as two rounds. The LC does not need to download  let alone parse the blockchain. The only information the SP must obtain from the LC are the addresses where the funds are located.
Cavefish is compatible with hierarchical (HD) wallets \cite{bip32} allowing straightforward address discovery over a range of child addresses with the LC sending only a single public key together with its chain code.
\cut{These are sufficient for the SP to scan the UTxO ledger and construct a transaction according to the specifications requested by the LC.}
The only complexity is having the LC sign the requested signature after the SP constructs it. Sending the signature in the open would allow the LC to modify the transaction, potentially removing SPs fee. Instead, the SP transmits the result in a redacted form, which we call an \emph{abstract transaction}.

The LC and SP then complete a blind signature protocol where the LC verifies if the transaction satisfies its defined specifications and only then creates valid signature(s) which are sent back to the SP. The abstract transaction is used to speed up this check.
As a last step, the SP attaches the signatures to the transaction and posts them on the blockchain on behalf of the LC.

To make our scheme viable in the real world, we adapt the blind predicate Schnorr signature scheme from~\cite{blindsigs}.
More precisely, we do not require the unlinkability requirement once the transaction has been published on the blockchain.
Accounting for timing, values and payees the potential anonymity set for a blinded transaction is trivially small, and in any case the transaction signed by the LC only needs to stay \emph{private until posted}.
This insight allows us to introduce the notion of a \emph{weakly} blind predicate signature scheme, a simplification that reduces the space and time complexity of the zero-knowledge component which asserts that the abstract transaction meets the LC's specifications.

In addition to the low communication overhead, our light client protocol gives the SP the ability to be reimbursed for its computation time required to construct the transaction without the need of an additional protocol or exchange. The requested transaction includes the SP's fee, i.e., an additional UTxO output, which makes up for the SP's costs and small reward. Additionally, if the client wishes to engage with the SP over a period of time, our mechanism can be used to initialize a payment channel that can be used subsequently for fast payments to the SP without impacting the size of transactions beyond the first. 
%removed talk of how fee is set, let's assume it's set out of band, also meshes with mkAbs

\noindent To summarize, our contributions are:
\begin{itemize}
\item We introduce the notion of intents, describing the desired end result of the light client's actions, as opposed to the method by which they are accomplished. We believe this to be an important abstraction as it assists in the conception, development and study of specialized efficient protocols as opposed to aiming for parity with full clients. To this end, we describe a domain-specific language (DSL) allowing the concise construction of intents for UTxO-based ledgers.
\item We propose Cavefish, a light client protocol that avoids any enrollment or synchronizing with chain whilst maintaining safety, providing compensation to the SP, imposing minimal communication overhead and being compatible with any UTXO blockchains using Schnorr signatures.
\item We introduce the notion of weakly blind predicate signatures, motivated by our ``private until posted'' goal. This brings together the notions of blind predicate signatures of~\cite{blindsigs} with the notion of signatures on committed messages (SBCM) from~\cite{10.1007/978-3-031-78679-2_7}.
\item We implement and benchmark the zero-knowledge proof component of Cavefish for two major blockchain platforms, Bitcoin and Cardano, and show that an ``unoptimized'' implementation achieves proving and verification times that are viable in a real-world deployment.
\end{itemize}


%Since our protocol features atomicity of service and payment, no enrolment or set up phase is required,  letting the light client freely choose any suitable service provider.

%Our work can be implemented for any UTxO or EUTxO blockchain, provided the intent specification is adapted to capture the ledger model and the blockchain supports standard Schnorr signatures. 




%\textcolor{orange}{mr: needs 1-2 passes and some references.}

%\todobox{
%What sets us apart?
%
%\begin{itemize}
%    \item Atomicity of payment+service
%
%    \item permission-less, decentralized
%
%    \item Model for (trustless) 2-party transaction construction rather than proving things about chain/ledger state
%
%    \item Do not require establishing a relationship with SP or any other set-up
%
%    \item inherent timeliness of transaction construction incentivized by SPs desired to earn
%    their tip. This is in contrast with the possibility of stale info provided from old Mithril snapshots in other LC models
%\end{itemize}}



% !TEX root = ../lc_main.tex
\section{Background and Related Work}

\textbf{Light clients and off-chain payments.}
Our approach facilitates transaction submission for a light client that is not aware of the history of a blockchain.
Therefore, we compare our work with existing solutions for light clients as well as mechanisms that allow the interaction with a blockchain when in resource constrained operation.
Our approach is different from a more ``traditional'' (i.e., what commonly is referred to) light client that has the primary goal of syncing to the blockchain in order to acquire the information necessary to interact with a smart contract or to submit a transaction.
We summarize the most important concepts and works in the area of light clients below.
We first describe the common idea of a light client and then outline concepts related to our work that complement light clients.

There is existing work analyzing light client functionality \cite{committee} \cite{soklc}. The majority of light client designs include the following main functionalities it is expected to perform: (1) issue queries, such as for the balance of an account, or the state of a transaction, and (2) safeguard secret information and submit transactions to the blockchain. 
In order to implement these functionalities, light clients use several generic techniques, most notably: header verification and consensus evolution verification. Unlike a \emph{full node}, a light client \emph{only verifies the headers of blocks}, and skips the verification of transactions and account balances \cite{committee}. This technique was made popular with SPV in Bitcoin~\cite{bitcoin} and nearly universally adopted by most practical approaches for light clients, e.g.,~\cite{ethereum}.
Because the validator set can change, consensus evolution verification is needed for blockchains running on proof-of-stake.
Another common technique is to compress the blockchain and/or ledger state in order to reduce the information a light client need for functioning reliably (see, e.g.,~\cite{flyclient}).

Some light clients \cite{lazy} use game-theoretic assumptions, e.g. to implement the slashing of previously deposited collateral in case of misbehavior, for example~\cite{superlight}.
The main cryptographic building blocks that are used to realize those techniques are succinct representation and proofs, such as data accumulators (often Merkle trees) and commitments and SNARKs.
Suitable signatures and hash functions are needed for certain light client designs. Examples of these include aggregate signatures and threshold signatures \cite{mithril} \cite{committee}. Yet other designs propose building networks for identity-based light payments that are not fully untrusted \cite{celo}.

\textbf{Intents}.
We also consider \emph{intents} and \emph{solver networks} to be related work. These designs attempt to establish a relationship between solvers and users via their (light) clients~\cite{ethresearch-solvers}. 
A light client issues an intent via as an abstracted transaction object. Then, solvers process the intent, incentivized by transaction fee or intent execution reward. The users are then free to accept or reject a solver’s proposal. In case the solvers are required to provide a deposit, slashing incentivizes honest behavior of a rational actor.
Since concepts around solver networks are relatively new, there is currently no universal standard governing the specification for intents and abstract transaction objects. To the best of our knowledge, this is the first work describing an intent-based light client protocol taking advantage of specific features of the UTxO ledger model that both minimizes communication and offers built-in incentive structure. 

\textbf{Cross-chain intents.}
A lot of work on intents applies is focused on building cross-chain intent resolution, which 
requires more sophisticated protocols that are able to communicate with 
multiple chains at once \cite{celer} .
Some protocol designs choose to resolve intents on-chain directly \cite{khalani}, while 
other offer both layer-1 and layer-2 options \cite{anoma}. Yet another designs is a consensus protocol
relying on on minimal trust assumptions while conforming to the interledger standard \cite{ripple}. 
Focusing on a single chain allows 
us to construct a simple low-communication protocol which makes use of the unique features of the 
underlying blockchain, and requires reduced little communication (only 
between two parties). In its simplicity, our protocol is more similar to payment channels. 

\textbf{Payment channels}.
The concept of \emph{payment channels}, and the related notion of \emph{state channels}, also bear similarities
to our approach.
Payment channels are a type of off-chain mechanism for blockchains. Payment channels allow users to establish a private payment
channel between two parties, e.g. \cite{lightning} \cite{raiden} \cite{ethereum}. 
A payment channel can be used to conduct a series of transactions without interacting
with the main blockchain. The creation of a payment channel requires first locking funds in an on-chain smart contract ~\cite{payment-channel}. 
Some channels, such as Hydra \cite{hydra}, not only allow
simple payments between users, but can simulate the majority of the on-chain transaction processing mechanism internally 
that are within the user base and liquidity limits of the channel.
This type of channel is usually called a \emph{state channel}. 

Generally, interactions in state and payment channels are limited to a pre-determined set of participants, liquidity, and 
types of actions users can perform. \footnote{Payment networks attempt to address this issue
by using multi-path payments, where the original payment is split into smaller parts,
each routed through different channels, allowing for greater liquidity, adding complexity and risk}
The limits of our proposed design to support 
different actions is dictated by the language used for request specification, which may be augmented in future work. 
At the same time, our design provides the same level of atomicity expected from payment channels, 
i.e., payments are either fully successful or completely fail.

After the initial on-chain setup has been completed, payment and state channels
are capable of greatly reducing both transaction processing times and fees.
The target user base for our light client design, however, are users
that either do not wish to (or are not capable of) engaging in establishing relationships with other chain users, participating
in an on-chain setup process, or intend to make repeat or scheduled interactions with the same
users. For this reason, the on-chain space- and cost-saving benefits of payment channels,
are unlikely to be accessible to a user base which
we target with our low-commitment design.


\textbf{API and Explorer Services}.
Many realistic UTxO ledger implementations (e.g. Ergo \footnote{\url{https://ergoplatform.org/}},
Cardano \footnote{\url{https://cardano.org/}}, and BitCoin \footnote{\url{https://bitcoin.org/}}) are set up
in a way that an invalid transaction will, in most cases, not result in any update to the ledger
state, getting rejected instead. This makes
services such as blockchain explorers \footnote{\url{https://beta.explorer.cardano.org/}} or Blockfrost (API
as a service for accessing the Cardano blockchain) \footnote{\url{https://blockfrost.dev/}} some of the
strongest competitors with our proposal, as they provide the data needed
for the LC to construct their transaction, often with a relatively high degree of reliability.
Such services currently do not appear to have the infrastructure required to perform the construction of transactions
for the light client. Along with providing access to such a service, 
the design we propose also includes a compensation structure for the service provider. 
Revenue from explorer services services such as those above is
either ad-based, requires users to create an account paid for with fiat currency, or is free in order to
promote use of the specific service/blockchain. We, on the other hand, propose a protocol 
in which service providers are compensated
in assets on the same blockchain as the one to which their transaction gets applied, and the payment structure
consists of a one-time, no-setup atomic exchange of assets for services. 

\subsection{Hierarchical Deterministic wallets and address discovery}
UTXO based blockchains, including Bitcoin and Cardano use the concept of Hierarchical Deterministic (HD) wallets \cite{bip32} where child addresses can be created in a deterministic way using a public key of a keypair $sk,pk$, and some short auxiliary data (the chain code). Given a public key $pk$, chain code $c$ and requested index $i$ for a child key, a hash function is used to produce a scalar $s_i$ enablind the derivation of the child key as $sk_i=sk+s_i$. It is also possible to use $s_i$ without knowing $sk$ to directly derive the corresponding public key $pk_i=pk\cdot g^{s_i}$. The same method is used to produce a child chain code $c_i$, so that multiple generations of keys are possible.

In our application, we can assume that a wallet can give the SPO a single public key and chain code, letting the SPO do address derivation and lookup on their side, using some agreed-upon bounds, or heuristics on the index depth. This keeps the communication cost from the light client to the SPO constant.


\subsection{Schnorr Blind Signatures}
As soon as their patent expired in 2008, Schnorr signatures~\cite{10.1007/0-387-34805-0_22} have been gaining significant adoption,
replacing RSA in many scenarios due to their smaller size and faster verification.
Compared to (EC)DSA, Schnorr offers similar efficiency and guarantees,
but (EC)DSA security proofs are most likely not possible without 
strong idealization~\cite{cryptoeprint:2023/914}.
As a consequence, EdDSA~\cite{cryptoeprint:2011/368}, a popular variant of the
 Schnorr scheme, is currently under consideration by NIST for standardization.
The security of Schnorr signaatures is based on the discrete 
logarithm assumption~\cite{jofc-2000-14276},
with proofs in the random oracle model (ROM) as well as in 
the algebraic group model (AGM) and the ROM~\cite{10.1007/978-3-030-45724-2_3}.

Standard Schnorr signatures are now widely used in blockchains such as Bitcoin, Bitcoin Cash, Litecoin, and Polkadot.
Monero, Zcash, and Cardano use the EdDSA variant.
The adption of Schnorr and EdDSA signatures is driven by privacy~\cite{cryptoeprint:2022/1636} and
 scalability gains~\cite{10.1007/s10623-019-00608-x}
but also the straightforward extension to \emph{blind Schnorr signatures}~\cite{10.1007/3-540-48071-4_7}.
Most Schnorr-based blind signature schemes
require multiple rounds~\cite{10.1007/3-540-48071-4_7}, which can make the protocol suceptible to denial-of-service attacks.
Research has thus been focusing on making blind signatures \emph{concurrently} secure where more than one session can be intertwined~\cite{}. % as well as round-optimal schemes~\cite{}.

Our construction hinges on the the concurrently secure blind and partially blind signing protocol
for standard Schnorr signatures found in~\cite{blindsigs} by Fuchsbauer, et al.---the first work stating rigorous security guarantees for a practical blind signature scheme based on Schnorr signatures.
Unlike schemes that had been presented before,~\cite{blindsigs} is not vulnerable to the attacks described in~\cite{10.1007/3-540-45708-9_19,cryptoeprint:2020/945} which showed that the hardness-assumption\footnote{The ROS (Random inhomogeneities in a Overdetermined
Solvable system of linear equations) problem was initially studied by Schnorr in~\cite{10.1007/3-540-45600-7_1}} existing schemes relied on for concurrent security can be solved in polynomial-time.
As our light client protocol is based on~\cite{blindsigs}, it also provides partial and predicate blindness, two properties that allow us to describe a transaction in an abstract way featuring ``redacted'' parts.
We briefly outline the construction of Fuchsbauer et al.~\cite{blindsigs} in the following.
The scheme is equivalent to the blind signature scheme~\cite{10.1007/3-540-48071-4_7} by Chaum and Pedersen but, in order to make the protocol concurrently secure, adds a commitment phase at the beginning where the user sends an encrypted version of the message $m$ and blinding values $(\alpha,\beta)$ to the signer.
In addition to that, the second message by the user includes a zero-knowledge proof alongside the Schnorr challenge $c$.
The zero-knowledge proof asserts to the signer that the initial (encrypted) commitment and $c$ have been derived from the same $m, \alpha, \beta$.

In addition to obtining concurrent security, one can leverage the zero-knowledge proof to assert additional facts, most notably, the user can prove to the signer that certain predicate(s) over message $m$ holds. This effectively turns the fully blind scheme into a predicate blind scheme.
Furthermore, support for predicate blindness implies partial blindness~\cite{blindsigs} since constructing a predicate that checks equality for parts of $m$ can be used to assert to the user that parts of the message correspond to the expected value(s).

\todobox{
Compare our approach  with :
\begin{itemize}
    \item "Free" websites monitoring the chain -- mention they lack long-term sustainability. mention big APIs. blockfrost, etc.
    \item Bridges (trustless and trusted). only overview. maybe SOK paper. and describe how LC and bridge are related.
    \item Payment channels (require upfront capital) + Payment networks~\cite{cryptoeprint:2017/823}
    \item LCs that operate on single-prover model (eg. with an established relationship via deposit)
    \item LCs that operate on multi-prover model
	%\item Time lock micro payment channel~\cite{{10.1007/978-3-319-21741-3_1}
\end{itemize}}



% !TEX root = ../lc_main.tex

%\subsection{Technical Background}
%UTxO model \textcolor{orange}{mr: do we still need this subsection?}

\section{Technical Background}

\textbf{Notation.} We use $y \gets x$ to denote that  variable $y$ is assigned the (possibly randomized) evaluation of $x$. When $x$ is a set, we denote uniformly random sampling from the set. We use $a=b$ to denote boolean comparison between $a$ and $b$ and $z:=w$ to denote equality by definition. We use $||$ to denote concatenation of bitstrings. When algorithms are randomized, we denote them as $A(x;r)$, where $x$ is the input and $r$ is the randomness, belonging to a randomness space $\mathcal{R}$. When we write $y\gets A(x)$ we imply $r\gets \mathcal{R};y\gets A(x;r)$. In algorithm descriptions we write $x,y,z \subseteq a$ to imply parsing $a$ to obtain $x,y,z$. 

Our protocols operate in the discrete log setting, where we assume the discrete logarithm problem is hard. We also follow standard conventions with regards to public key encryption using $(\PKE.\Keygen,\PKE.\Enc,\PKE.\Dec)$ to describe schemes and require IND-CPA security. For signatures, we use Schnorr signatures $(\SDS.\Setup,\SDS.\Keygen,\SDS.\Sign,\SDS.\Ver)$. Finally, we use parametrized non interactive Arguments $(\NArg.\mathsf{Rel},\NArg.\Setup,\NArg.\Prove,\NArg.\Ver,$ $\NArg.\SimProve)$ to allow the relation being proven to depend on the group setting. For reasons of space we present the full description of these primitives in Appendix \ref{app:techbg} 




The security of  Schnorr signatures, has been well studied in the random oracle model (ROM), where a hash replaced with an ideal random function the adversary can only call as an oracle. However in this work the need arises to instantiate the underlying hash function $\hash$ so that we may reason about it within a proof system \cite{9155085} e.g. prove that a known $y$ is $y=\hash(x)$ for some secret $x$. This renders ROM-based proofs inapplicable necessitating an assumption on the security of the scheme. This does not differ significantly from typical implementation practices (where the hash function is infact drawn from a small pool for standardized options), and also from the treatment of this issue in the literature \cite{blindsigs}.  


\begin{definition} A hash function generator $\HGen(n) \to \hash$, on input $n\in \N$ generates a hash function $\hash : \{0,1\}^* \to \Z_n$.
\end{definition}





\begin{assumption} \label{ass:schnorr}\cite{blindsigs} There exists a group generator $\GGen$ and hash function generator $\HGen$ s.t. the Schnorr signature scheme $\SDS$ is strongly unforgeable under definition \ref{def:seufcma}. \end{assumption}

Similarly, in the ROM we can trivially guarantee $\hash$ is pseudorandom, whereas for our setting we require an additional assumption:

\begin{assumption} The hash function generator $\HGen$ is such that for all $n$\label{ass:hash}
$$\left| \Pr[\hash \gets \HGen(n); (m_0,m_1)\gets \A(\hash);b\gets\{0,1\};r\gets\{0,1\}^\lambda:b=\A(\hash(m_b||r))]-1/2 \right| $$ is negligible in $\lambda$. \end{assumption}



\input{sections/txmodel.tex}


% !TEX root = ../lc_main.tex

\section{Light Client Specification}

 We model light clients in terms of their functionality, constraints, and protocols for communication with full nodes and potential service providers. By functionality, we refer to the interaction between a light client and its user(s).
 We describe the functionality of a light client in terms of constructing (posting) and verifying resolved \emph{intents}. A light client specifies the intent it wants resolved using a special domain-specific language (DSL). Then, the client sends the intent to a service provider (SP), who is incentivized to respond to this intent with an \emph{abstract transaction}. An abstract transaction is a transaction that has been modified to partially conceal data in a way that (i) allows the light client to verify that their intent has been resolved in the original transaction and (ii) does not allow the client to use the abstract transaction to form a valid transaction that avoids paying the service provider fee. 

 A light client's constraints may be on the communication cost to answer a series of queries, available state/data storage, and assumptions about connectivity. As a consequence,
  a particular light client design may be limited in the intents it can construct even if a broader class of intents is supported by available software.

%\todobox{ [TODO: revisit time] [TODO: can we do the LC part of bridges with these intents?]}

 % In order to provide answers to queries by the user, a light client participates in a number of protocols between itself and a set of full nodes. These protocols may be one to one or one to many, depending on the design of the light client.


\subsection{Intent specification}
\label{sec:dsl}

An \emph{abstract transaction}, $\TxAbs$, is the data structure that a light client receives from the service provider instead of
the full plaintext transaction (see Fig.~\ref{fig:abs-tx}). 
In $\TxAbs$, the $\fun{aux}$ and the $\inputs$ fields are removed, and instead of the $\Signature$
field, it contains only the signing keys $\fun{sigKeys}$ and not the corresponding signatures. The value of the first output is zeroed out to allow for any leftover input funds to be returned to the LC's change address. The intent a light client constructs is any expression in the DSL $\mathcal{I}_\mathsf{post}$, see Figure~\ref{fig:dsl}. 

\begin{figure}\fbox{
\begin{displaymath}
\begin{array}{rlll}
  \TxAbs &=&(\outputs: [\TxOut],\ \  \fun{validityInterval}: \Interval{\Tick},\\
  & &\ \mint: \Value,\ \  \fun{fee}: \N ,\ \  \fun{sigKeys}: \fun{Set}~\pubkey) \\
%  & &\ \texttt{//Abstract transaction type}
  \nextdef 
  \fun{mkAbs} &:& \Tx \to \TxAbs \\
  \fun{mkAbs}~\var{tx} &=& \var{tx} ~\{ ~\outputs = \var{tx} . \outputs,\\
%  & & \texttt{//Note no inputs are present}\\
  & &\ \outputs[0].\Value{} = -1, \ \ \texttt{//Leftover balance to change address} \\
% & & \texttt{//Change address receives any amount}\\
  & &\ \fun{validityInterval}  = \var{tx} . \fun{validityInterval},\\
  & &\ \mint  = \var{tx} . \mint ,\ \   \fun{fee}  = \var{tx} . \fun{fee} \\
  & &\ \fun{sigKeys}  = \fun{dom}~(\var{tx} . \sigs)~\} \\
%  & &\ \texttt{//Abstract transaction constructor}
\end{array}
\end{displaymath}}
\caption{Abstract transaction type $\TxAbs$ and constructor $\fun{mkAbs}$}
\label{fig:abs-tx}
\end{figure}

For the implementation of the example usecases of our design, we require the definition of the following functions:
\begin{itemize}
    \item[(i)] The function that constructs a transaction based on the specification (this function may fail, outputting $\fun{nothing}$ instead of a transaction, if the intent cannot be resolved), see Fig.~\ref{fig:mktospec}.
    \[ \fun{mkToSpec} : \LState \times \mathcal{I}_\mathsf{post} \to \Tx^? \]
    \item[(ii)] The function that checks that a given abstract transaction matches the intent specification, see Figure~\ref{fig:dsl}
    \[\fun{chkSpec}~(\mathcal{I}_\mathsf{post} \times \TxAbs) \to \B\]
\end{itemize}
%\textcolor{red}{pc: should chkSpec take TxAbs or TX ?}
\begin{figure}
    \textsc{$\mathcal{I}_\mathsf{post}$ Constructors}
\begin{displaymath}
\begin{array}{rlll}
    \fun{MustMint}           &: \Value \to \mathcal{I}_\mathsf{post}\\
    \fun{SpendFrom}         &: \Script \to \mathcal{I}_\mathsf{post}  \\
    \fun{MaxInterval}   &: \Slot \to \mathcal{I}_\mathsf{post}  \\
    \fun{PayTo}  &: (\Value \times \Script)  \to \mathcal{I}_\mathsf{post}  \\
    \fun{ChangeTo} &: \Script \to \mathcal{I}_\mathsf{post}  \\
    \fun{MaxFee} &: \N  \to \mathcal{I}_\mathsf{post} \\
    \fun{AndExps}  &: [\mathcal{I}_\mathsf{post}] \to \mathcal{I}_\mathsf{post} \\
\end{array}
\end{displaymath}
\nextdef
\textsc{Evaluation of $\mathcal{I}_\mathsf{post}$}
\begin{displaymath}
\begin{array}{rlll}
  \chkSpec{\_} &:& \mathcal{I}_\mathsf{post} \to \TxAbs \to \B \\
  \chkSpec{\fun{MustMint}~v} (\var{tx}) &=&  v \leq \var{tx} . \mint \\
  \chkSpec{\fun{SpendFrom}~s} (\var{tx}) &=& \applyScript{s}(\fun{dom}~(\var{tx} . \sigs), \var{tx}.\validityInterval) \\
  \chkSpec{\fun{MaxInterval}~i} (\var{tx}) &=& (\var{tx}.\validityInterval)_2 - (\var{tx}.\validityInterval)_1 \leq i \\
  \chkSpec{\fun{PayTo}~(s, v)} (\var{tx}) &=& (s, v) \in \var{tx} . \outputs  \\
  \chkSpec{\fun{ChangeTo}~s} (\var{tx}) &=& (s, \fun{consumed}~-~\fun{produced}) \in \var{tx} . \outputs  \\
  \chkSpec{\fun{MaxFee}~f} (\var{tx}) &=&  \var{tx} . \fun{fee} \leq f \\
  \chkSpec{\fun{AndExps}~[a1 ; a2 ; ... ; ak]} (\var{tx}) &=& (\chkSpec{a1}~\var{tx}) \wedge (\chkSpec{a2}~\var{tx}) \wedge ... \wedge (\chkSpec{ak}~\var{tx})
\end{array}
\end{displaymath}
\caption{$\mathcal{I}_\mathsf{post}$ constructors and evaluation}
\label{fig:dsl}
\end{figure}
Note that the function $\fun{mkToSpec}$ constructs a transaction by editing an initial transaction $\fun{initTx}_{a,mf}$ (see Figure~\ref{fig:mktospec}). This transaction includes a minimum fee $mf$ as well as a tip $\fun{tip}$ to the service provider, output to its address $a$.
These functions are defined such that for any $l,~i$, where $\fun{mkToSpec}~(l,~i) \neq \fun{nothing}$,
necessarily $\chkSpec{i}~(\fun{mkAbs}~(\fun{mkToSpec}~(l,~i))) = \true$. The design of the DSL is minimal, 
and is meant only to showcase the operation of the light client protocol.
To expand the functionality of the light client, more constraints need to be
included in the DSL, e.g., a way to limit transaction size, or support for specifying
desired token exchanges. We leave this for future work. 

We showcase two examples in the following, for a given ledger state $l$:
\begin{itemize}
    \item[($\fun{txi}_1$)] Intent to mint some token $t$ within an interval of length $j$ and maximum fee $f \leq \fun{minfee}~l$ (note that this will not be a valid transaction since it will have no inputs):
    \[ i_1 =  \fun{AndExps}~[ \fun{MustMint}~t ; \fun{MaxInterval}~j ; \fun{MaxFee}~f ; \fun{ChangeTo}~s ]\]

    \item[($\fun{txi}_2$)] Intent to pay $x$ from outputs locked by $\fun{RequireSig}~k1$ to  $\fun{RequireSig}~k2$ (note that this intent may not be possible to resolve in the key $k1$ does not have sufficient funds on the ledger $l$)
    \[ i_1 =  \fun{AndExps}~[ \fun{SpendFrom}~(\fun{RequireSig}~k1) ; \fun{PayTo}~(\fun{RequireSig}~k2,~x) ; \fun{ChangeTo}~(\fun{RequireSig}~k1)]\]
\end{itemize}
The resulting transactions are given in Figure~\ref{fig:txs}, assuming the intents can be resolved.
The minimum required capacity of our light client is to perform all the actions that are specified in the protocol we describe in the next section.

%In the next section, we describe our security requirements  for such protocols as well as some modeling assumptions  for their environment.




% !TEX root = ../lc_main.tex


\section{Light Client Protocols}
\label{sec:protocols}
We given an overview on our light client protocol and its components. Our protocol maximizes space and communication efficiency by essentially pushing all storage and lookup tasks to the SPO, whilst  guaranteeing the resulting transaction is safe for the light client to sign, and compensating their SPO for their effort. This is not trivial as the transaction must be signed sight-unseen (blinded) otherwise, the light client could simply remove the payment to the SPO and submit the modified transaction.

To satisfy both parties we rely on cryptographic tools as well as some established features in implemented blockchains such as transaction validity periods and handling of invalid transactions.

\subsection{Definitions and Requirements}

\begin{definition}[Correctness] A transaction building protocol is \textbf{correct} if for an honest light client, and service provider and any state $\LState$ and intent $\mathcal{I}_\mathsf{post}$ such that: if $\fun{mkToSpec}(\LState, \mathcal{I}_\mathsf{post}) \neq \bot$, then the protocol completes and the transaction $\Tx'$ produced by the SP is such that $\fun{\fun{checkTxL}~}(LState, \Tx')=1$. \end{definition}

\todobox{ Let ${\fun{checkTxL}~}(l, tx)$  be shorthand for  $\fun{checkTx}~((\var{slot}, \var{fee}), \var{utxo},~\var{tx})$, with ledger state $l$ 
containing UTxO state $\fun{utxo}$ and current slot number $\fun{slot}$ and minimum fee $\fun{fee}$}.

\begin{definition}[Safety] A transaction building protocol is \textbf{safe} if for an honest light client and any service provider SP we have that: if transaction $\Tx'$ is produced by the SP is such that  $\fun{\fun{checkTxL}~}(LState, \Tx')=1$
%against the client's key (XXX note: maybe we can ommit this)
, then it must be that $\fun{chkSpec}(\mathcal{I}_\mathsf{post} , \Tx')=1$. \end{definition}

The safety property is derived from two factors: first, the unforgeability property of $\WBPSP$ will ensure that only transactions that satisfy the given predicate will be signed. Second, our assumptions on ledger rules ensure that signed transactions involving spent UTXOs will have no effect (as opposed to something undesirable to the user).  

\begin{definition}[Private until posted] A transaction building protocol has the  \textbf{private until posted} (PuP) property if a PPT adversarial client $C_\mathcal{A}$ cannot win the following experiment with probability significantly higher than $1\over 2$.

For an honest service provider SP and any client $C_\mathcal{A}$ we have that: client provides 2 states $\LState_0, \LState_1$ and one intent $\mathcal{I}_\mathsf{post}$ such that for $i=0,1$ it is $ \fun{mkToSpec}(\LState_i, \mathcal{I}_\mathsf{post}) \neq \bot$.  The experiment flips a coin $d$ and runs the protocol between SP and C* using state $\LState_d$. $C_\mathcal{A}$ outputs $d^*$. The adversarial client wins iff $d=d^*$. \end{definition}

The Private until posted property of our protocol is derived from Theorem \ref{thm:blind} and the protocol structure. Given an adversary against $\A$ PuP, we can build  an adversary against the weak blindness of the $\WBPSP$ scheme by running the SP once on $\LState_0$ and once on $\LState_1$ to produce two challenge messages for the blindness security game and simulate the role of the challenge towards $\A$ by passing on the rest of the protocol messages (the experiment enforces random padding internally). If $A$ can distinguish which ledger was used in producing the messages we can use the same index as our guess in the PuP game. 

\subsection{A Communication Optimal Transaction Building Protocol}
We given an overview on our light client protocol and its components.
As the protocol is uniquely suited for UTxO-based blockchains, we develop our approach
to support the general ledger model explained in Section~\ref{sec:model}
and the intents set forth in Section~\ref{sec:dsl}.

We point out that our protocol is optimal in terms of light client storage and light client \& SPO communication. The light client only needs to store information about which wallet subkeys have been active. In terms of communication, our protocol only transmits a constant number of elements when using a succinct proof system. This compares favorably to solutions such as SPV which incur linear costs or even solutions based on succinct proof systems for the chain tip,  but where the client needs to verify transaction inclusion into blocks via Merkle proofs.




\subsubsection{Security Model and Requirements}
\label{sec:secmodel}
\textbf{Rational Actors} For liveness, we assume that protocol participants are rational in the sense that they will opt to complete tasks that benefit them directly (i.e in the form of rewards, or in the form of a desired transaction being posted). For safety we assume parties can be fully malicious.

\noindent\textbf{Transaction Expiry} For safety, we require that transactions can be set to be valid only within a certain time window. This is to prevent trivial attacks where a malicious SPO delays posting a transaction indefinitely. The user could contact a different SPO to create a second transaction with similar parameters (e.g. by paying the same recipient with different coins) with the risk that the first SPO would also post their transaction, making the user pay twice.

\noindent\textbf{Idempotent Doublespends} For safety we also require that attempted double spends are idempotent, i.e. they are simply not valid transactions and do not penalize (slash) the user for signing them. 

\subsection{Protocol Description}
\label{sec:mainproto}
The protocol operates as follows: the light client connects to the SP and sends a posting intent $\var{int_{post}}$, as well as a public key that has received payments (and, optionally a BIP-032 chaincode so that the SP can also derive child keys from the initial one). The SP will then search for UTXOs belonging to the client and form a transaction that satisfies the intent $\var{int_{post}}$, including in the transaction a tip for the SP's own work. Lastly, the SP adds a random note $\nt$ to the note field of the transaction. Then, the client and SP engage in a weakly blind predicate signature protocol so as to efficiently have the client sign the transaction whilst whilst ensuring that (1) the client needs to sign before learning the full transaction (typically the used UTXOs are hidden), and (2) that the signed transaction satisfies $\var{int_{post}}$ by means of satisfying the corresponding predicate.

%\subsubsection{Outline}
%The following computations are performed in the protocol
%
%\begin{itemize}
%    \item[(i)] Intent specification is generated (see Section \ref{sec:dsl})
%    \item[] \textcolor{orange}{mr: do we still need this extra signature? we can just say communication happens over secure channel.}
%    \item[(ii)] Signature is generated as follows :
%    \[\var{sig} = \fun{sign}~(\var{int_{post}}, sk_{\mathrm{client}})\]
%    \item[(iii)] The signature is checked as follows :
%    \[ \fun{checkSig} (\var{int_{post}}, vk_c, \var{sig}) = \true \]
%    \item[(iv)] The transaction is generated as follows, including a randomly generated bytestring
%    in the $\fun{auxData}$ field for extra entropy :
%    \[ \var{tx} = \fun{mkToSpec}~(\var{ls},~\var{int_{post}},~\var{tx}_t)~\{~\fun{auxData}~=~\fun{random}~\} \]
%    \item[(v)] Abstract transaction is generated as :
%    \[ \var{tx}_A = \fun{mkAbs}~\var{tx}  \]
%    \item[(vi)] Blind Signature Commitment (see Section \ref{sec:blind} for details)
%
%    \item[(vii)] Intent specification Proof (see Section \ref{sec:dsl}) :
%    \[ \chkSpec{(\var{int_{post}})}(\var{tx}_A )= \true \]
%    \item[(viii)] Proof Check. Blind signatures are computed by (see Section \ref{sec:blind} for details), with $\var{sks} = \fun{getSKs}~\fun{dom}~(\var{tx} . \fun{sigs})$ :
%    \[ \var{sigs} = \{~(s,~\fun{blindSign}~(h,~\var{tx}_A,~ s)) \vert s \in sk_{\mathrm{client}} \cup ~\var{sks}~\} \]
%    \item[(ix)] Signatures are computed from blind signatures (see Section \ref{sec:blind} for details) :
%    \[ \var{sigs'} = \{~(s,~\fun{mkRealSig}~(s,~h,~\var{tx}_A,~k)) ~\vert~(s, k) \in \var{sigs}~\} \]
%    \item[(ix)] Signatures are added to the transaction
%    \[ \var{tx'} = \var{tx}~\{~sig = \var{sigs'}~\} \]
%\end{itemize}


We show our complete light client protocol  in Figure~\ref{fig:protocol-short}, leveraging the  weakly blind signatures predicate of section \ref{sect:ourscheme}.




\tikzset{
  leftArrowBelow/.style={
    append after command={
      (\tikzlastnode.south east) ++(0,0) edge[->, thick] ([yshift=0pt]\tikzlastnode.south west)
    }
  },
  rightArrowBelow/.style={
    append after command={
      (\tikzlastnode.south east) ++(0,0) edge[<-, thick] ([yshift=0pt]\tikzlastnode.south west)
    }
  }
}
\begin{figure}[ht]
\begin{tikzpicture}[every node/.style={font=\small}]
    \label{fig:protocol-short}
	\matrix (m)[matrix of nodes, column  sep=0.1cm,row sep=0mm,
			nodes={draw=none, anchor=center, 
			minimum height=3mm},
			column 1/.style={nodes={minimum width=4.2cm}}, 
			column 3/.style={nodes={minimum width=5.2cm}}, 
			column 2/.style={nodes={minimum width=2.3cm}} 
			 ]{
		\textbf{Signer (LC)} & & \textbf{Service Provider (SP)}\\
    & & Obtain newest $\LState_t\ ls$\\
		Generate intent $\var{int_{post}} \in \mathcal{I}_\mathsf{post}$\\
		& \node[rightArrowBelow] {$\var{int_{post}}, vk_c$};  \\
		& & Determine client key(s) $\bm{X}_c$ and UTXOs \\
		& & from $(\var{int_{post}}, vk_c)$\\
		& & Generate $\var{tx} = \fun{mkToSpec}(\var{ls},\var{int_{post}})$,\\
    & & $\var{tx}_A = \fun{mkAbs}(\var{tx})$, $\nt \xleftarrow{\$} \{ 0,1 \}^\notelength$ \\
    & &  Produce commitment  $com_\var{tx}$ \\ % = $\var{Com}(\var{tx}{\color{blue}||\var{nt}};\rho)$\\
		& \node[leftArrowBelow] {$com_\var{tx},\var{tx}_A,\bm{X}_c$}; & \\
		%\ h=\fun{txid}~\var{tx}
		% %& ZK.proof $p$ of $P (h,\  \var{tx}_A) = \true$ & \\
		Produce blind sig. \\
		commitment $\var{R}=g^r$ \\
		% sign abstract transaction\\[-7mm]
		&  \node[rightArrowBelow] {$\var{R}$};\\
		& & Produce challenge $c$ and proof $\pi$\\
%    & & $c=H(R,X_c,\var{tx}{\color{blue}||\var{nt}})$ and\\
%		& & proof $\pi$ : $\fun{chkSpec}(\var{int_{post}},\var{tx}) = 1\ \land$\\
%	 	& & \quad\quad\quad $com_\var{tx} = \var{Com}(\var{tx}{\color{blue}||\var{nt}};\rho)\ \land$\\
%		& & \quad\quad\quad $c=H(R,X_c,\var{tx}{\color{blue}||\var{nt}})$ \\
		&  \node[leftArrowBelow] {$c,\pi$}; & \\
		% %& ZK.proof $p$ of $P (h,\  \var{tx}_A) = \true$ & \\
		 Check proof $\pi$ \\
		Produce  $s=r+cx$ \\
		% sign abstract transaction\\[-7mm]
		&  \node[rightArrowBelow] {$\var{s}$}; \\
		& & Produce signature $\sigma = (R,s)$\\
		& & If valid, post $(\var{tx}||\var{aux},\sigma)$ on chain.\\
	};
	\draw[thick] (m-1-1.south east)--(m-1-1.south west);
	\draw[thick] (m-1-3.south east)--(m-1-3.south west);
	\node[fit={(m-8-3.south west) (m-19-3.south east) (m-12-1.north west)},
   ultra thick, dotted, inner xsep=11pt, inner ysep=0pt,
    rounded corners=1mm, draw=gray, label={[gray,anchor=north west]north west:WBPS Execution for $m:=\var{tx}||\nt $}]{};	
\end{tikzpicture}
\caption{Light client protocol featuring WBPS scheme. A detailed descripton of the WBPS scheme is in Fig.~\ref{fig:WBSP}.}
\label{fig:protocol-short}
\end{figure}

%\begin{tikzpicture}
%    \label{fig:protocol}
%	\matrix (m)[matrix of nodes, column  sep=1.5cm,row  sep=7mm, nodes={draw=none, anchor=center,text depth=0pt} ]{
%		\textbf{Light Client (LC)} & & \textbf{Service Provider (SP)}\\[-4mm]
%		Generate intent & & \\[-7mm]
%		(i) $\var{int_{post}} \in \mathcal{I}_\mathsf{post}$, & & \\[-7mm]
%		(ii) generate signature $\var{sig}$  & & \\[-7mm]
%		% and predicate $\fun{opt} : \TxAbs \to \B$\\
%		&  $(\var{int_{post}}, vk_c, \var{sig})$ & \\[-7mm] %\fun{opt}(\ ),\
%		& & (iii) Check signature \\[-7mm]
%		& & (iv) Generate transaction  \\[-7mm]
%		& & and (v) initiate WBPS signing \\[-7mm]
%		&& with abstract $\var{tx}_A$  \\[-7mm]
%		\phantom{x}& $\var{tx}_A$ & \\[-7mm]
%		%\ h=\fun{txid}~\var{tx}
%		% %& ZK.proof $p$ of $P (h,\  \var{tx}_A) = \true$ & \\
%		(vi) Produce Blind.  \\[-7mm]
%			Sig. Commitment \\[-7mm]
%		% sign abstract transaction\\[-7mm]
%		& $\var{R}$\\[-7mm]
%		& & (vii) Produce proof   \\[-7mm]
%		& & $\fun{chkSpec}(\var{int_{post}},\var{tx}_A) = 1$ \\[-7mm]
%		& $(c,\pi)$ & \\[-7mm]
%		% %& ZK.proof $p$ of $P (h,\  \var{tx}_A) = \true$ & \\
%		(viii) Check proof \\[-7mm]
%		Produce pre signature \\[-7mm]
%		% sign abstract transaction\\[-7mm]
%		& $\var{bsig}$\\[-7mm]
%		& & (ix) Produce signature $\sigma$\\[-7mm]
%		& & from pre signature.\\[-7mm]
%		& & Post $(\var{tx'},\sigma)$ on-chain.\\
%	};
%
%	\draw[shorten <=-1.5cm,shorten >=-1.5cm] (m-1-1.south east)--(m-1-1.south west);
%	\draw[shorten <=-1.5cm,shorten >=-1.5cm] (m-1-3.south east)--(m-1-3.south west);
%	%\draw[shorten <=-1cm,shorten >=-1cm,-latex] (m-6-2.south west)--(m-6-2.south east);
%	\draw[shorten <=-1cm,shorten >=-1cm,-latex] (m-10-2.south east)--(m-10-2.south west);
%	\draw[shorten <=-1cm,shorten >=-1cm,-latex] (m-13-2.south west)--(m-13-2.south east);
%	\draw[shorten <=-1cm,shorten >=-1cm,-latex] (m-16-2.south east)--(m-16-2.south west);
%	\draw[shorten <=-1cm,shorten >=-1cm,-latex] (m-19-2.south west)--(m-19-2.south east);
%	%\draw (m-10-2.north west) rectangle (m-19-2.south east);
%	\node[fit={(m-10-1.north west) (m-18-1.north west) (m-21-3.south east) (m-15-3.south east)},
%      ultra thick, inner sep=0pt, rounded corners=1mm,
%      draw=cyan, label={[cyan,align=center]270:WBPS Execution}]{};
%\end{tikzpicture}


\subsubsection{Security}

\begin{theorem}
If the ledger rules include the \emph{Idempotent Doublespends} property and $\WBPSP$ is unforgeable, the protocol of Section \ref{sec:mainproto}
 is safe.
\end{theorem}

\begin{proof}The safety property is derived from two factors: first, the unforgeability property of $\WBPSP$ (Theorem \ref{thm:unforge}) will ensure that only transactions that satisfy the given predicate will be signed. Second, our assumptions on ledger rules ensure that signed transactions involving spent UTXOs will have no effect (as opposed to something undesirable to the user).  
\end{proof}

\begin{theorem}
If  $\WBPSP$ scheme is weakly blind, the protocol of Section \ref{sec:mainproto} is private until posted.
\end{theorem}

\begin{proof}The Private until posted property of our protocol is derived from Theorem \ref{thm:blind} and the protocol structure. Given an adversary against $\A$ PuP, we can build  an adversary against the weak blindness of the $\WBPSP$ scheme by running the SP once on $\LState_0$ and once on $\LState_1$ to produce two challenge messages for the blindness security game and simulate the role of the challenge towards $\A$ by passing on the rest of the protocol messages (the experiment enforces random padding internally). If $A$ can distinguish which ledger was used in producing the messages we can use the same index as our guess in the PuP game. \end{proof}


\subsection{Discussion}

\label{sec:threats}


%\textbf{Erroneous User Balance.} A common threat vector is convincing the user of an erroneous balance on their account. This can lead a user to
%complete a transaction outside the blockchain based on wrong information (e.g. accept an invalid transaction as payment for goods, or exchange his private key
%in the understanding that is controls $x$ stake instead of $y$.)

%\textbf{Invalid Transactions.}  In response to an LC query, it may be desirable for an SP to construct a transaction that does not meet the intent specified by the user. For example, the created transaction may pay the wrong recipient, or pay an amount over and above the one specified by the user.

%\textbf{Protocol Failures.}  In a multi-node protocol, a malicious user may wish to force the protocol to abort even though there exist a number of honest nodes who would otherwise ensure the protocol completes successfully. In the single node, a malicious node can always cause a protocol to fail by not replying.

\textbf{Suboptimal Transactions.}  In response to an LC query, it is possible for an SP to construct a
\emph{sub-optimal} transaction (in terms of cost) which nevertheless matches the LC's specification. For example, the LC may
include a larger-than-necessary system fee in the transaction, or not provide the LC with the best
available exchange price for a specific token. We consider cost optimizations an orthogonal issue that can be addressed by e.g. 
a market system for SPs.

%Our model assumes that an LC is able to submit their transaction to multiple SPs, and will have pick one
%transaction to sign from the resulting set $S$ of responses received within some pre-defined amount of
%waiting time. Alternatively, the LC may simply choose the first valid transaction sent by an SP.
%Assuming such a competitive SP environment, the suboptimal transaction threat is mitigated by
%the incentive, in the form of a service fee for the SP, to provide both quickest and the most optimal
%response. Note that the SP is not a-priori aware of how the LC will make
%the choice of what transaction to sign, and how long their waiting period is.

\textbf{Liveness against Malicious SPs.}
\label{sec:stale-state}
Malicious SPs are able to force delays for the user by completing the signing and never posting the transaction. This forces the user to wait until the signed transaction is no longer valid before retrying. Otherwise, as intents are not necessarily satisfied by a unique transaction, it is possible that the first (malicious) node may be able to post the first transaction (which was purposefully delayed) after the posting of the second one. This can cause the user to e.g. double pay for a service as well as the full node fee. Even so, given enough retries the user will reach an honest SP and the transaction will be posted.


%Arbitrary delay caused by SP: SP can offer to execute an intent and provide the crafted TX, but never publish it on the chain.
%    For this reason, SP should provide the hashes of the UTxO references (or similar) to the light client, s.t. the LC can reuse those references in another transaction. This avoids the case where SP hangs on to the TX indefinitely or posts it at an arbitrary point in time causing two or more TXs being executed if the LC resubmits its intent/TX.

%\textbf{Non-submission of Transactions.} A class of protocols deals with the joint creation of transactions by the light client and node, with the transaction submission being left to the full node. It is possible for the full node to not post such a transaction in a timely manner, potentially leading the user to retry at a different time. 
%\textbf{Falsified or Unsafe Transaction Data}. SP lies to client about the UTxOs being spent by the tx, and tricks them into doing something they did not want to do (that is why 0-knowledge proof that outputs were correctly specified)

%\textbf{Data Leakage.} The 2-PC protocol we propose is one that has leakage. In particular,
%the co-construction of a transaction by the LC and the SP allows the LC to learn certain information
%about the chain state, such as, e.g. the upper and lower bound of the interval containing the current blockchain
%time. In fact, under certain circumstances, after several iterations of our SP-LC protocol, the LC
%may gain the capacity to construct certain kinds of transactions themselves. As noted above, this
%information can also be gained from blockchain explorer services. The
%reason an LC may engage in our protocol that it allows all transaction-construction,
%and state-tracking and querying logic to be off-loaded to the SP. If the LC can do all this themselves,
%engaging in the protocol does not benefit them in any way.
%
%However, there is a specific kind of
%data leakage threat we have have built the protocol to be resilient against : leakage of $\UTxO$ entry
%identifiers, i.e. the inputs of a transaction. It is impossible to construct a valid transaction
%without knowing the the inputs that need to be included in it for it to match the specification,
%and this information cannot be guessed without knowledge of blockchain state. If this information
%is somehow leaked to the LC along with the modified transaction, it makes it possible to for the
%LC to not pay the SP for their work.
%
%Specifically, the full transaction (including the correct inputs) can be modified locally to
%replace the output containing the SP's fee with one returning it as change to the LC, signed, and
%submitted to the network by the LC. As part of our protocol, the SP replaces the inputs of a
%transaction with outputs that correspond to those inputs in the UTxO set.
%We have demonstrated that guessing inputs without access to either the inputs themselves, or
%the transactions that have placed them into the UTxO set, is sufficiently difficult to not be considered a
%vector of attack.
%This data may also be leaked by getting recorded in a script, datum, or redeemer. E.g., a script
%may require that its redeemer also list the inputs of the transaction. The responsibility of mitigating
%this type of leakage rests on the SP, who can address this by limiting the kinds of specifications
%they agree to fulfill.

%\todobox{TODO : make sure we have shown that "We have demonstrated that guessing inputs
%without access to either the inputs themselves, or
%the transactions that have placed them into the UTxO set,
 %is sufficiently difficult to not be considered a
%vector of attack."} --see assumtion 2



\subsubsection{Multi-SP Protocols and optimality }

 However, we can also formulate a version of this
protocol where the LC instead sends the specification to multiple SPs, selecting the best
response, and engaging in the rest of the protocol only with that SP.

To do this, let $\fun{opt} : \TxAbs \to \Z$ be a function that rates abstract
transactions to express LC's preferences. For example, the total amount of primary tokens
spent by the transaction can be such a function :
$\fun{opt}~\var{tx} = \fun{coinValue}~(\sum_{o \in~\var{txd}.\fun{spentOuts}} \var{o}.\val)$.
A transaction $\var{tx}$ is \emph{optimal} in a set $S \in \type{Set}~\TxAbs$ when

 \[ \fun{opt}~\var{tx} = \fun{min}~\{~\fun{opt}~\var{tx'}~\vert~\var{tx'}~\in~S~\} \]

To get a multi-SP protocol, the first step of the single-SP protocol in Figure \ref{fig:protocol}
must be augmented, so that the pair $(\fun{opt}, \var{int_{post}}$ is sent to each SP instead of just
sending $\var{int_{post}}$.
Upon receiving transactions $\var{tx}_{A,i}$, with $0\leq i < k$, from each of the $k$ SPs responding
to LC's query, LC will engage in the rest of the protocol only with the sender of $\var{tx}_{A,i}$.

 Note that the $\fun{opt}$ function and the specification serve different purposes in the protocol.
 The specification is checked, and any response transaction that does not satisfy it is discarded.
 On the other hand, it is not required that a transaction be optimal across all possible
 specification-satisfying transactions.  The kind of
 sophisticated optimization (such as what is required, e.g., for optimized order-matching) would require
 an entirely distinct set of tools for demonstrating the optimality result, such as ZK proofs about
 the full blockchain state (rather than just the associated transaction), together with evidence that
 the proof is about state that is \emph{sufficiently current} (see Section \ref{sec:stale-state}).
 For example, a proof that
 that there were no better offers for a specific token available on the ledger at the time the SP
 produced a response to the LC. We do not assume that
 either the SP or the LC are necessarily capable of performing or verifying (resp.) optimality according to the
 function LC requested be optimized, but an LC is capable to comparing transactions using $\fun{opt}$.


% !TEX root = ../lc_main.tex

 \section{Blind Signatures for Abstract Transactions}
 \label{sec:blind}
 In order to allow the light to sign an abstract transaction we implement blind signatures on (partially) blinded transaction objects.
 At the minimum the SP hides the inputs to the transaction, i.e., the references to UTxO objects which are present in the ledger and required to cover the transaction.
 This ensures that the light client cannot simply obtain the transaction from the SP, then edit it to remove the SPs compensation.

% In Bitcoin, for example, UTxOs are captured in the ledger as follows:
 %For Cardano, the structure looks as follows:

% \todobox{ Describe UTxO briefly for all chains we support?  }.


 %The specific ledger implementation might require slight adaptations in terms of signature type and used hash functions---in this work, we give a very general construction that can be tailored to specific blockchains.

 Our construction is inspired by the predicate blind signature mechanism in~\cite{blindsigs}.
 It realizes a concurrently secure blind and partially blind signing protocol resulting in standard Schnorr signatures. This is a core property as it allows our protocol to be compatible with popular blockchains such as Bitcoin and Cardano without any modifications.
%The signature scheme builds upon the original protocol proposed in~\cite{10.1007/3-540-48071-4_7} with the addition of a commitment phase that binds the signer to her secrets (blinding value and message), preventing a forgery attack (described in~\cite{bibid}) and making the scheme unforgeable under concurrent sessions.

Given our application, the unlinkability property of blind signatures is not of much benefit: signed transactions are somewhat infrequent, often have different payees or payment sums (i.e different predicates) and re indirectly timestamped due to their time of posting on the blockchain. Further, there seems to be little benefit in preventing the light client from knowing which protocol section produced which payment. For this reason, we do away with the requirement of the signer being unable to distinguish between messages signed on different sessions. We only require that the singer is unable to distinguish the message \emph{before}  the signature-message pair is posted. 

This weakening of the blindness property brings about the benefit that we can remove the blinding operations from the protocol (for a small efficiency increase) and also from the proof of correctness of the challenge generation (where the gain is more significant as it removes group exponentiations using foreign field arithmetic). The resulting scheme is similar in operation to the Signatures over Blocks of Committed Messages construction of Bobolz et. al. \cite{10.1007/978-3-031-78679-2_7} with the addition of predicate checking.
 
%Once the SP publishes final (unblinded) transactions on the chain, the LC can in theory download recent blocks and observe the transaction that was posted by the SP on behalf of the light client. If downloading is too resource-intensive, the LC might also consult an online block explorer for said purpose. As a consequence, abstract transactions, published transactions and signatures are assumed to be ``linkable'' and unlinkability provided by the signature scheme is redundant in our scenario.



%  \todobox{
%     \begin{itemize}
%  \item Do we consider TLS connection between LC and SP? A network adversary could otherwise link/localize abstract transactions and transactions posted on the ledger.
%  \item specify the signing algorithm : $\fun{sign} : (\H, \privkey) \to \H$
%  \item specify the function to make real sigs from blind sigs $\fun{mkRealSig} : (\pubkey \times \TxId \times \TxAbs \times \H) \to \H$
% \item specify the blind signing algorithm $\fun{blindSign} : \H \times \TxAbs \times \privkey \to \H$
% \end{itemize} }.



\subsection{Weakly blind predicate signatures }

We adapt the definitions of \cite{blindsigs} to account for the weak variant of blindness.

A WBPS scheme is parameterized by a family of polynomial-time-computable predicates,
which are implemented by a p.t. algorithm $P$, the predicate compiler: on input a predicate
description $prd \in \{0, 1\}^*$ and a message $m \in \{0, 1\}^*$, $P$ returns 1 or 0 indicating whether $m$
satisfies $prd$.

 A WBPS scheme \WBPSP{} for predicate $P$ is defined by the following algorithms. We focus on
schemes with 2-round (i.e., 4-message) signing protocols for concreteness.
\begin{itemize}
\item $Setup(1^\lambda) \to par$: the setup algorithm, on input the security parameter, outputs public
parameters $par$, which define a message space $\mathcal{M}_{par}$.
\item $KeyGen(par) \to (sk, vk)$: the key generation algorithm, on input the parameters $par$, outputs
a signing/verification key pair $(sk, vk)$, which implicitly contain $par$, i.e., $vk = (par, vk')$.
–\item $\langle Sign(sk, prd),User(vk, prd, m) \rangle \to (b, \sigma)$: an interactive protocol with shared input $par$
(implicit in $sk$ and $vk$) and a predicate $prd$ is run between the signer and user. The signer
takes a secret key $sk$ as private input, the user’s private input is a verification key $vk$ and
a message $m$. The signer outputs $d = 1$ if the interaction succeeds and $d = 0$
otherwise, while the user outputs a signature $\sigma$ if it succeeds, and $\bot$ otherwise.
\item $Ver(vk, m, \sigma) \to 0/1$: the (deterministic) verification algorithm, on input a verification key
$vk$, a message $m$ and a signature $\sigma$, outputs 1 if $\sigma$ is valid on $m$ under  $vk$ and 0 otherwise.
\end{itemize}

For a 2-round protocol the interaction $\langle Sign(sk, prd),User(vk, prd, m) \rangle \to (d, \sigma)$ can be realized by the following algorithms:
\begin{align*}
(txt_{U,0},st_{U,0}) \gets& User_0(vk, prd, m) & (txt_{S,1},st_{S}) \gets& Sign_1(sk, prd, txt_{U,0}) \\
(txt_{U,1},st_{U,1}) \gets& User_1(st_{U,0}, txt_{S,1}) & (txt_{S,2}, d) \gets& Sign_2(st_S, txt_{U,1}) & \\
\sigma \gets& User_2(st_{U,1}, txt_{S,2}) &&
\end{align*}
We write $(d, \sigma) \gets  \langle Sign(sk, prd),User(vk, prd, m) \rangle$  as shorthand for the above sequence.



\begin{definition} A WBPS scheme $\WBPSP$ satisfies weak blindness if for all p.p.t.
adversaries $\A$:

$$Adv^{\mathsf{BLD}}_{\WBPSP,\mathcal{A}}(\lambda) := \Pr[
\BLD^{\A,1}_{\WBPSP}(\lambda) - \BLD^{\A,0}_{\WBPSP}(\lambda)] \mbox{ is negligible in $\lambda$.}
$$


\end{definition}

We note that the $\mathsf{BLD}$ experiment mauls the message by appending a random bitstring $r$ of length $\lambda$. This is appropriate to our setting, where the SP is explicitly allowed to use the  ``notes'' field of a transaction to add a randomizer. Without this mauling, any indistinguishability-based definition would fail.

\begin{figure}[ht]
%\footnotesize
    \fbox{

 \begin{minipage}[t]{0.40\linewidth}
\textbf{\uline{$\BLD^{\A,b}_{\WBPSP}(\lambda)$}}
%\hrule

$par \gets Setup(1^\lambda) $\\
$(m_0,m_1,prd,vk',st_\A) \gets \A_1(par)$\\
\textbf{if} $P(prd,m_0) = 0$ \textbf{or} $P(prd,m_1) = 0$
\textbf{then}  \quad \textbf{return} 0\\
$\nt \gets \{0,1\}^\lambda $\\
$m \gets m_b||\nt$\\
$vk \gets (par,vk')$
$sess\gets \mathsf{init}$\\
$b^*\gets\A_2^{\mathsf{ChalUser}}(st_\A)$\\
$ \textbf{return} \quad  b=b^* $
\end{minipage}%
\hspace{0.5cm}%
\begin{minipage}[t]{0.45\linewidth}
\textbf{\uline{$\mathsf{ChalUser}(msg=\emptyset)$}}
%\hrule





\textbf{if} $sess=\mathsf{await}$\\
\null\quad $sess\gets\mathsf{closed}$\\
\null\quad $\sigma \gets User_2(st_u, msg)$



\textbf{if} $sess=\mathsf{closed}$\\
\null\quad $msg' \gets Ver(vk, m, \sigma)$

\textbf{if} $sess=\mathsf{open}$\\
\null\quad $sess\gets\mathsf{await}$\\
\null\quad $(msg',st_u) \gets User_1(st_u, msg)$


\textbf{if} $sess=\mathsf{init}$\\
$\null\quad sess\gets\mathsf{open}$\\
$\null\quad (msg',st_u) \gets User_0(vk, prd, m)$





\textbf{return} $msg'$

\end{minipage}
}
    \caption{Weak Blindness experiment $\BLD^{\A,b}_{\WBPSP}(\lambda)$ for a $\WBPS{}$ scheme with predicate compiler $P$, adversary $\A$ and parameter $b$.}
\end{figure}
\begin{figure}[ht]
\footnotesize
    \fbox{
 \begin{minipage}[t]{0.50\linewidth}
\textbf{\uline{$\textsf{CMA}^{\A}_{\WBPSP}(\lambda)$}}
%\hrule

$par \gets Setup(1^\lambda) $\\
$ (sk, vk) \gets KeyGen(par) $\\
$Q \gets 0$\\
$S \gets 0$\\
$P \gets 0$\\
$(\vec{m^*},{\vec\sigma^*}, \vec{prd^*}) \gets \A^{SigInit,SigComplete}(vk)$\\
$n \gets \left |{\vec{m^*}}\right |$\\
\mbox{$ \textbf{if} \quad \prod_i Ver(vk,m_i^*,\sigma_i^*) \neq 1 \textbf{ return 0} $}\\
\mbox{$ \textbf{if} \quad \prod_i prd_i^*(m_i^*) \neq 1 \textbf{ return 0} $}\\
\mbox{$ \textbf{if} \quad \exists i, j: (i\neq j) \land (m_i^*,\sigma_i^*)=(m_j^*,\sigma_j^*) \ \textbf{ return 0} $}\\
\mbox{$ \textbf{if} \quad  n > Q \textbf{ return 1}$}\\
\textbf{if} $\nexists \rho \in Perm(S):$\\
\mbox{$\quad \forall i \leq n: (prd_{\rho(i)}(m_i)=1) \land (st_{\rho(i)}=\bot)$}\\
\textbf{then} \quad \textbf{return 1} \\
\quad \textbf{return 0}

\end{minipage}%
\hspace{0.5cm}%
\begin{minipage}[t]{0.40\linewidth}
\textbf{\uline{$\mathsf{SigInit}(prd, msg_{U,0})$}}
%\hrule
%$(msg_{U,0},st_{U,0}) \gets User_0(vk, prd, m)$\\
\mbox{$S\gets S+1; P\gets P+1$}\\
\mbox{$(msg,st_{S}) \gets Sign_1(sk, prd, msg_{U,0})$}\\
\mbox{$prd_S \gets prd$}\\
\textbf{return} $msg$\\

\textbf{\uline{$\mathsf{SigComplete}(s,msg_{U,1})$}}
%\hrule


\textbf{if } $s>S$ \textbf{or} $st_s=\bot$ \\ \textbf{ then } \textbf { return } $\bot$\\
%$(msg_{U,1},st_{U,1}) \gets User_1(st_{U,0}, msg_{S,1})$ \\
$(msg, d) \gets Sign_2(st_s, msg_{U,1})$ \\
\textbf{if } $d=1$ \textbf{ then } $Q\gets Q+1$\\
$P \gets P-1$\\
$st_S \gets \bot$\\
\textbf{return} $msg$

%Todo: change OSig1 / OSig2

\end{minipage}
}
    \caption{Chosen Message Unforgeability Experiment  $\mathsf{CMA}^{\A}_{\WBPSP}(\lambda)$ for a $\WBPS{}$ scheme with predicate compiler $P$, adversary $\A$}
\end{figure}

\begin{definition}. A WBPS scheme $\WBPSP$ satisfies unforgeability  if for all p.p.t.
adversaries $\A$
$$Adv^{\mathsf{EUF-CMA}}_{\WBPSP,\mathcal{A}}(1^\lambda) := \Pr[
\mathsf{CMA}^{\A}_{\WBPSP}(1^\lambda)=1] \mbox{\quad is negligible in $\lambda$.}
$$
\end{definition}


\subsection{Cavefish scheme}
\label{sect:ourscheme}
Our Scheme operates as follows: given an intent (i.e. predicate predicate description) $prd:= \var{int_{post}}$ and predicate compiler $P$ s.t. $(P(prd))(\tx):=\fun{chkSpec}(\var{int_{post}},\fun{mkAbs}(\tx))$ the requestor (i.e. the SP) encrypts the message, in our case the full transaction $m:=\tx$ and sends the commitment $C$ to the signer, who returns a random group element $R$ used in the final Schnorr signature. The requestor replies with the Schnorr challenge $c$ as well as a proof $\pi$ of its correct construction, and of the fact that $\tx$ satisfies $\var{int_{post}}$. 

We instantiate the scheme with a parametrisable $\NArg$ for the following relation, \begin{equation}
\label{narg}
\mathrm{R}_{\mathrm{Cavefish}}
	(\underbrace{(q, \mathbb{G}, G, \mathsf{H})}_{\text{parameters}\ par},\
	\overbrace{(X, R, com_\mathrm{tx}, \\TxAbs, c, \var{int_{post}} )}^{\text{known statement}\ \theta}\ ,\
	 \underbrace{(\tx||\nt,\ \rho)}_{\text{witness}\ \omega}\ )
\end{equation}
checking that  $\fun{chkSpec}(\var{int_{post}},\TxAbs) = 1 \land m=\tx||\nt \land C = \PKE.\Enc(m;\rho) \land c= \hash(R,X,m) \land \TxAbs=\fun{mkAbs}(\tx)$. For the extended version, we alter the predicate compiler to $(P_{\text{ext}}$ such that $(P_{\text{ext}}(prd))(\tx):=\fun{chkSpec}(\var{int_{post}},\tx)$ (i.e. we check $\tx$ rather than $\fun{mkAbs}(\tx)$.

The signer completes the signing only if the proof verifies correctly. We present the full protocol in Figure \ref{fig:WBSP}.
\begin{figure}[ht]
%\centering
%\footnotesize
\fbox{%
  \begin{minipage}{0.95\textwidth}
    % Top minipage
    
    \begin{minipage}[t]{0.09\textwidth}
    \phantom{a}
    \end{minipage}
    \begin{minipage}[t]{0.48\textwidth}
 \textbf{\uline{\bm{$\WBPS.\Setup$}}}\\[0.5ex]
$(\G,g,q) \gets \GGen(1^\lambda)\\$
$\hash\gets  \HGen(q)\\$
$sp \gets (\G,g,q,\hash)\\$
$(\crs,\tau)\gets \NArg.\Setup(sp)\\$
$(ek,dk)\gets \PKE.\Keygen(1^\lambda)\\$
$par \gets (\crs,ek)\\$
$\textbf{return } par\\$
    \end{minipage}
    \begin{minipage}[t]{0.02\textwidth}
    \phantom{a}
    \end{minipage}
    \begin{minipage}[t]{0.30\textwidth}
      \textbf{\uline{\bm{$\WBPS.\Keygen (par)$}}}\\[0.5ex]
$(\G,g,q) :\subseteq par\\$
$x\gets \Z_q; X\gets g^x \\$
$sk\gets(par,x)\\ vk\gets (par,X)\\$
$\textbf{return } (sk,vk)\\$\\


\textbf{\uline{\bm{$\WBPS.\Ver(vk,m,\sigma)$}}}\\[0.5ex]
$(\G,g,q,\hash,X):\subseteq vk$\\
$(R,s)\gets \sigma$\\
$c\gets \hash(R,X,m)$\\
$\textbf{return } (g^s=R\cdot X^c)$

    \end{minipage}
    \hfill
    \begin{minipage}[t]{0.1\textwidth}
    \end{minipage}

    \vspace{0.05em}

    % Bottom minipage with invisible table
    \begin{center}
      \begin{tabular}{>{\raggedright\arraybackslash}p{0.3\textwidth} p{0.15\textwidth} >{\raggedright\arraybackslash}p{0.3\textwidth}}
        \textbf{\uline{\bm{$\WBPS.\Sign(sk,prd)$}}} & & \textbf{\uline{\bm{$\WBPS.\Request(vk,prd,m)$}}} \\
        $(\G,g,q,\crs,ek,x):\subseteq sk$ &  &  $(\G,g,q,\hash,\crs,ek,X):\subseteq vk$ \\
         & $\xleftarrow{\quad C \quad}$  & $C\gets \PKE.\Enc(ek,m;\rho)$ \\
         $r\gets \Z_q; R\gets g^r $& $\xrightarrow{\quad R \quad}$ &$c\gets \hash(R,X,m)$\\
         && $ \theta \gets (X,R,c,C,prd,ek)$\\
         && $w\gets (m,\rho)$\\
         & $\xleftarrow{\quad c,\pi \quad}$ & $\pi \gets \NArg.\Prove(\crs,\theta,w)$\\
         \textbf{if } $\NArg.\Ver(\crs,\theta,\pi) = 0:$ &&\\
         \textbf{return } $0$ &&\\
         $s\gets (r+c\cdot x) \mod q$ & $\xrightarrow{\quad s \quad}$ & $\textbf{if }g^s \neq R\cdot X^c : \textbf{return } \bot$ \\
          \textbf{return } $1$ && $ \textbf{return } \sigma\gets(R,s)$
      \end{tabular} 
    \end{center}
  \end{minipage}%
}

\caption{The weakly blind predicate Schnorr signature scheme \WBPSP}\label{fig:WBSP}
\end{figure}



\subsection{Security}

%Blindness (initial sketch): If the commitments are perfectly hiding, then the first round does not reveal anything to the adversary. Likewise, the proof  can be simulated with no contradiction (there will always exist a correct witness to proceed with). However, the oracle query cannot be adversarially controlled: $(R,X,m_0)$ and $(R,X,m_1)$ are both known to the adversary. We thus need to replace $m\gets m_b$ with a weaker form.

\begin{theorem} The $\WBPSP{}$  scheme of Figure \ref{fig:WBSP} achieves weak blindness when the encryption scheme $\PKE$ is IND-CPA secure, $\NArg$ has the zero knowledge property, assumption \ref{ass:hash} holds w.r.t. the hash generator $\HGen{}$.\label{thm:blind}
\end{theorem}
\begin{proof}
We structure our proof as a sequence of games so that (1) two consecutive games are computationally indistinguishable to the adversary and (2) in the final game, the view of the adversary is independent of the challenge $b$. We define the initial game $G_0$ to be $G_0:=\BLD^{\A,b}_{\WBPSP}(\lambda)$. For $G_1$ we replace proofs with simulated ones. This is not detectable by the adversary due to the zero knowledge property of $\NArg$. Second, for game $G_2$ we replace $C$ with an encryption of a fixed message $\bar{m}$ rather than $m_b$, this too is undetectable\footnote{This will also change the input of the simulator, but this is not an obstacle: we simply consider the simulator to be part of the adversary we are constructing against the IND-CPA security of $\PKE$.} due to the IND-CPA security of $\PKE$. For game $G_3$ we replace the challenge $c$ to also use  $\bar{m}$ rather than $m_b$ in the hash. This is computationally indistinguishable due to assumption \ref{ass:hash}.
 %  We replace $m\gets m_b$ with the weaker form $m\gets m_b||m_r$. The commitments are hiding, so  the first round does not reveal anything to the adversary. Likewise, the proof  can be simulated with no contradiction (there will always exist a correct witness to proceed with). Finally, the oracle result $c$ cannot be predicted. It is either  $c=(R,X,m_0||m_r)$ or $c=(R,X,m_1||m_r)$, but neither value can be obtained without correctly guessing the value of $m_r$ (Event $H$, which only happens with negligible probability). Conditioning on $\neg H$, $c$ is independent of the challenge bit $b$.
\end{proof}

\begin{theorem} The $\WBPSP{}$  scheme of Figure \ref{fig:WBSP} achieves unforgeability when the encryption scheme $\PKE$ is IND-CPA secure, $\NArg$ is sound, and assumption \ref{ass:schnorr} holds w.r.t. the hash generator $\HGen{}$\label{thm:unforge}.
\end{theorem}

\begin{proof}
We follow the proof of \cite{blindsigs} with little changes. As before, we structure our proof as a sequence of games so that the probability of adversarial success changes negligibly from one game to the next. We define the initial game $G_0$ to be $G_0:=\mathsf{CMA}^{\A}_{\WBPSP}(\lambda)$.

For game $G_1$, the experiment holds the decryption key for $\PKE$, so that we extract the message from $C$ ahead of time, and thus predict the $c$ value once the value of $R$ has been determined. If the value sent by the adversary is different from the predicted one, and the accompanying proof $\pi$ verifies we abort early, diverging from the original game. This only happens with negligible probability though, due to the adaptive soundness of $\NArg$ (if $\pi$ does not verify, $G_0$ would abort as well).

For game $G_2$, we check to see if any of the adversarial signatures uses a message that belongs to a session that was not closed. If so, we abort early, diverging from $G_1$. However, this implies that the second part of the signature $s$ along with the secret key $x$ can be used to calculate the discrete log of $R$. Thus, if the early abort occurs (i.e. adversary manages to win using unfinished sessions)  with non negligible probability  we can build a discrete log calculator by embedding a discrete log challenge in one of the R values. Otherwise, we can assume that almost all of the adversarial wins use at least one ``new'' message, and thus the success probability for $G_1$ and $G_2$ differ only negligibly .

For $G_3$ we build a reduction to the EUF-CMA security of Schnorr (Assumption \ref{ass:schnorr}). We remove key generation from the challenger, and instead obtain a $vk$ from the Schnorr EUF-CMA security game.  When we decrypt a message $m$, we query it on our signing oracle, obtaining $R,s$ as the signature. The reduction forwards $R$ to the adversary anticipating $c,\pi$. The checks introduced in $G_1$ ensure the $c$ value matches the predicted one, and thus $s$ correctly completes the signature. This ensures the simulation can complete the game. If the adversary wins the security game  $\WBPSP{}$ security game, there must be a signature on a new message that the reduction uses to also win the Schnorr EUF-CMA game with the same probability.
\end{proof}


% !TEX root = ../lc_main.tex

\section{Analysis \& Applications}

\textbf{Storage and Communications Requirements.} We point out that our protocol is optimal in terms of light client storage and light client \& SP communication. The light client (LC) only needs to store information about which wallet subkeys have been active. In terms of communication, our protocol transmits only a constant number of elements when using a succinct proof system. This compares favorably to solutions such as SPV which incur linear costs or even solutions based on succinct proof systems for the chain tip, but where the LC needs to verify transaction inclusion into blocks via Merkle proofs.

\noindent \textbf{Transaction discovery via HD Wallets.}
\label{sec:xclient}
Assuming that the client only uses a single address can be unrealistic, as is requiring the LC to send the SP a list of addresses presumed active. Our protocol is compatible with BIP-32 \cite{bip32} hierarchical deterministic (HD) wallets.
This enables child addresses to be deterministically created from a parent address by hashing the parent key $X$, a chaincode value $\text{cc}$ and an address index $i$. Concretely, $o_i:=\hash(X,\text{cc},i)$ is the private key offset of child key $i$, i.e. the $i$-th child keypair is $x_i=x+o_i$ and $X_i=X\cdot g^{o_i}$. Child chaincodes are determined in the same way. 

Thus, an LC can simply transmit the chaincode corresponding to its public key $X$, and the SP will be able to derive a list of child addresses (bounding indexes can be done heuristically, or with hints from the client). The protocol then proceeds with minor changes:
\begin{enumerate}
\item The challenge $c$ is now derived as $c=\hash(R,X_c,m)$
\item The witness $w$ now includes the index of the child key $i$ (or vector of indices $\bm{i}$ and chaincodes $\textbf{cc}$ if the child is further down the tree).
\item The relation now checks that the hashed value is of the form $X_c=X\cdot g^{o_c}$ where $o_c=
\textsf{HDDerive}(X,\bm{i},\textbf{cc})$.
\item The SP modifies the  component $s:=xc+r$ to $s'=s+ o_c\cdot c$, adjusting for the child offset.
\end{enumerate}
Notably, the LC does not learn which child address was used, it can simply sign with regards to the sent public key $X$ and the SP can maul the signature as needed.

\noindent \textbf{Asset and Non-Asset Compensation.} In our model, we have so far assumed that an SP performs its
services in exchange for a fee. This fee may be either flat, or calculated on the basis of transaction size,
or complexity of transaction specification, etc.---the details of this fee calculation are dependent on the
specifics of the implementation of our protocol, the marketplace, and SPs preferences.
The fee may also be specified in either an amount of primary asset tokens,
or some other user-defined tokens. A fee requirement in user-defined tokens could be useful in the case
of the SP service for LCs being associated with another type of service trading in such tokens, such as a
videogame token marketplace. Yet another option for SPs is to request non-monetary compensation
for their services, such as requiring engagement with a specific smart contract (i.e. SP does DApp fee
sponsorship), voting for a specific update,
delegating stake, etc. All these actions can be performed by the very transaction that SP constructed
according to LC's specification.

% !TEX root = ../lc_main.tex

\section{Implementation}
To assess the efficiency of our construction, we implement and benchmark the $\NArg$ component of the Cavefish protocol. Given today's implementations of zk-SNARKs, running the $\NArg$ is expected to be the most time and resource intensive part of our light client protocol.

We base our tests on the trusted-setup zk-SNARK system \emph{Groth16}~\cite{cryptoeprint:2016/260} implemented by \emph{Iden3}~\cite{circom}.
The circuits we construct and benchmark are inspired by the implementation of~\cite{blindsigs} and written in the domain-specific language of Circom~2.1.

The \NArg we implement captures the relation $\mathrm{R}_{\mathrm{Cavefish}}$ in the protocol in Section~\ref{sect:ourscheme}.
We implement the ``light'' version of Cavefish, i.e., the intent is given by\\
$int_{\text{light}}(\fun{tx}):=\{\textbf{return } (\fun{mkAbs}(\var{tx})=\fun{mkAbs}(\var{tx_0}))\}$. The light client effectively specifies the transaction, except for the input UTxOs and note field $\textsf{aux}$. 


%We note that the abstract transaction $\var{tx}_A$ is part of the known statement $\theta$ in $\mathrm{R}_{\mathrm{Cavefish}}$ as it constitutes an input to the predicate $P$. If full blindness is desired, then $P:= \var{int_{post}}$ and otherwise $P$ is a composite predicate that evaluates to true if $\var{tx}$ has been partially blinded according to $\var{tx}_A$ and all the specifications stated in $\var{int_{post}}$ are met.

% proving: $\fun{chkSpec}(\var{int_{post}},\var{tx}) = 1$ $\land \var{Com}(\var{tx};\rho)$ from %\begin{equation}
%\label{narg}
%\mathrm{R}_{\mathrm{LCP}}
%	(\underbrace{(q, \mathbb{G}, G, \mathsf{H})}_{\text{parameters}\ par},\
%	\overbrace{(X_{\mathrm{client}}, R, com_\mathrm{tx}, \var{tx}_A, c, \var{int_{post}} )}^{\text{known statement}\ \theta}\ ,\
%	 \underbrace{(\var{tx}||\var{nt}, \rho)}_{\text{witness}\ \omega}\ )
%\end{equation}

\subsection{Implementation choices}
To implement and measure the arithmetic complexity of the relation $\mathrm{R}_{\mathrm{Cavefish}}$,
we use BN254 as the curve for \emph{Groth16} to operate on, i.e., the curve group has 254 bits and the relation is instantiated over an arithmetic circuit with modulus of 254 bits. %, i.e., the order of the group given by  BN254 has 254 bits.
The BN254 curve is one of the standard choices in Circom and forces the inputs to the circuit to be elements of the field given by BN254. 
We capture transaction $\var{tx}$ and abstract transaction $\var{tx_A}$ as bit strings of some length $n$ encoded as field elements, which allows us to instantiate a circuit that can handle transactions of length up to $n$.

In order to implement the encryption scheme $\PKE$ that is needed to encrypt $tx$ as $C$, 
 we use ElGamal~\cite{elgamal1985public} public key encryption over the Baby-JubJub curve~\cite{whitehat2020baby}.
 The reason for choosing Baby-JubJub curve is that the field given by BN254 is the base field of the Baby-JubJub curve and thus, any element can be represented as two elements of the BN254 field. As a consequence, the group operation is efficiently arithmetizable in the circuit~\cite{blindsigs}.
 Furthermore, we encapsulate key and encryption of ElGamal using DHIES~\cite{abdalla1999dhaes}, i.e., the shared secret in ElGamal serves as a seed to the PRF generating random group elements for an additive one-time-pad that encrypts $tx$. The PRF is instantiated as a Poseidon hash~\cite{263850} that is efficiently arithmetizable by design.

Unfortunately, most common cryptographic hash functions are not ``circuit-friendly''~\cite{263850} and thus it would be beneficial to optimize the hash function used to create the Schnorr challenge in the blind signature protocol. However, to be compatible with existing blockchain platforms and create standard Schnorr signatures, we have to adopt the exact hash function specified by the respective ledger. We use the library of hash functions in~\cite{komuves2025hashcircuits} which provides Circom implementations for many popular hash functions. The authors put effort into the optimization, but 
more efficient implementations might be possible.
Despite potential further improvements, the complexity of our resulting circuit is mainly governed by the number of rounds the hash function has to execute when producing the Schnorr challenge.


\subsection{Benchmarks}
We test and benchmark partially blind signatures for both Bitcoin and Cardano.
Partial blindness can be interpreted as a predicate itself (see Sections~\ref{sec:mainproto} and~\ref{sect:ourscheme}).
The ``light'' version of Cavefish uses $int_{\text{light}}$ which can be implemented with substring equality checks, provided the transaction to be signed can be unambiguously separated into blinded and nonblinded parts. 
Cardano transactions are CBOR-encoded and contain the transaction-id and index for every UTxO serving as input~\cite{cardano-ledger}.
Similarly, Bitcoin transactions (after Taproot update~\cite{bip340}) feature an input count followed by a list of inputs in binary format, each consisting of a transaction hash and output index.
We assume for our benchmarks that one UTxO serves as input to the transaction and needs to be blinded.

%For partial blindness, Cavefish constructs the following predicate for the \WBPSP \ scheme:
%$$P(m_\mathrm{known},m):= \mathsf{return}\ (m_\mathrm{pub} = m_\mathrm{known})\ \mathrm{where}\ (m_\mathrm{pub},m_\mathrm{blind}) := m$$
%asserts that transaction $\var{tx}$ corresponds to the abstract transaction $\var{tx_A}$.
%We implement $P(m_\mathrm{known},m)$ as part of the arithmetic circuit describing $\mathrm{R}_{\mathrm{Cavefish}}$.

We measure the number of contraints, the proving key size, and the proof size that $\mathrm{R}_{\mathrm{Cavefish}}$ requires in Circom.
The number of constraints are obtained from the arithmetization when given as a R1CS relation.
We also keep track of the time it takes to create the resulting circuit, the proving time and the proof verification time.
 
 The results are summarized in Table~\ref{table_results} for Bitcoin and Cardano as the target platform. The experiments were executed on commodity hardware based on an Intel(R) Core(TM) i7-8750H CPU operating at 2.20~GHz with 12 cores and 16~GB of RAM.


\begin{table}[h!]
\centering
\caption{Benchmark of $\mathrm{R}_{\mathrm{Cavefish}}$ for Bitcoin and Cardano implementing the ``light'' version of Cavefish.}
\label{table_results}
{\footnotesize
\begin{tabular}{@{} lccc @{}}
\toprule
  & \multicolumn{1}{c}{\textbf{Bitcoin}} & \multicolumn{1}{c}{\textbf{Cardano}} \\
\midrule
Signature scheme & \multicolumn{1}{c}{Schnorr} & \multicolumn{1}{c}{EdDSA} \\
Curve        & \multicolumn{1}{c}{Secp256k1} & \multicolumn{1}{c}{Ed25519}           \\
Hash          & \multicolumn{1}{c}{SHA-256} & \multicolumn{1}{c}{SHA-512}          \\
\midrule
%\multirow{2}{*}{Blindness type}     & & partial & & & & partial \\
Transaction size & 254 B & 285 B\\
 & 288 b blinded & 333 b blinded\\
\midrule
Proving key size & 115 MB & 116 MB\\
Proving key verification time & 18.6 s & 20.5 s \\
Verification key size & 67 kB & 93 kB\\
Proving time & 5.1 s & 5.8 s\\
Proof size & 806 B & 805 B\\
Proof verification time & 0.57 s & 0.59 s\\
Number of constraints & 226509 & 245181 \\
\bottomrule
\end{tabular}
}
\end{table}

%Transaction input fields in Cardano and Bitcoin take up space proportional to the number of UTxO inputs.


Bitcoin has one major conceptual difference from Cardano as the message being signed by the Schnorr signature scheme is the hash of the transaction $\mathsf{H}(tx)$ instead of the transaction itself.
Therefore, \WBPSP\ is executed with $m := \mathsf{H}(\var{tx}||\nt )$.
A straightforward adaption to \WBPSP\ that supports hashed transactions is to allow the predicate $P$ to accept an additional input which is a witness attesting to the signed message $m$ satisfying $P$.
This additional witness must be included in the witness $\omega$ for $\mathrm{R}_{\mathrm{Cavefish}}$~\cite{blindsigs}. Therefore, compared to Cardano, the circuit for Bitcoin has to perform an additional invocation of the hash function (SHA-256). On the other hand, the hash function used by Cardano (SHA-512) has a larger codomain and higher complexity, which requires more constraints.

We remark that the projects our implementation builds upon are in development, such as the hash function library, and lack certain optimizations.
Nevertheless, the introduction of a weak variant of blindness in our light client protocol
allowed us to obtain results within practical bounds and shows that the Cavefish protocol is deployable in the real world.


%The most obvious choice is to blind/redact the UTxO references serving as the inputs to the transaction preventing a light client from constructing the transaction on its own, without the inclusion of the output that represents the tip/reimbursement to the service provider. If the light client posts the modified transaction on chain and thereby circumvents the service provider, no tip is due and the light client is able to obtain information that constitutes a valid transaction for free.


% !TEX root = ../lc_main.tex

%\section{Conclusion}





%%
%% Bibliography
%%

%% Please use bibtex, 

\bibliography{lcbib}


\appendix
\section{Expanded Technical Backgroud}
\label{app:techbg}

\subsection{Discrete Log Groups}
\begin{definition} A group generator \GGen{} is a probabilistic polynomial time (p.p.t.) algorithm with input a security parameter $\lambda$ and outputs a group description $\G,g,q$ such that $\G$ is a group of prime order $q\approx 2^\lambda$ with generator $g$. We say that the discrete logarithm problem is hard w.r.t. \GGen{} if for all p.p.t $\A$ we have that

$$\Pr[(\G,g,q) \gets \GGen(1^\lambda); t\gets \Z_q; h\gets g^t: t=A(\G,g,q,h) ] \mbox{ is negligible in }\lambda.$$
\end{definition}

\subsection{ Public Key encryption and Schnorr Signatures}

A public key encryption scheme $\PKE$ comprises a set of polynomial time algorithms $\Keygen,\Enc,\Dec$ with the following syntax:
\begin{itemize}
\item $\Keygen(1^\lambda) \to (ek,dk)$. Creates a public/private keypair $ek,dk$. The encryption key $ek$ also defines the message space $\mathcal{M}$
\item $\Enc(ek,m;\rho) \to C$. Encrypts a message $m\in \mathcal{M}$ under the public encryption key $ek$. \item $\Dec(dk,C) \to m$. Decrypts a ciphertext $C$ using the private decryption key $dk$.
\end{itemize}

\begin{definition} A public key encryption scheme $\PKE$ is \emph{correct} if 
$$\Pr[ (ek,dk) \gets \PKE.\Keygen(1^\lambda); m\gets \mathcal{M}:\PKE.\Dec(sk,\PKE.\Enc(pk,m))=m ] =1.$$
\end{definition}

\begin{definition} A public key encryption scheme $\PKE$ is \emph{IND-CPA secure} if for all stateful p.p.t. adversaries \A, the difference
$$\left| \Pr[ (ek,dk) \gets \PKE.\Keygen(1^\lambda); (m_0,m_1)\gets A(ek);d\gets\{0,1\}; c^*\gets \PKE.\Enc(ek,m_d):A(c^*)=d ] - {1\over2} \right|$$
is negligible in $\lambda$.
\end{definition}

\begin{definition} The Schnorr digital signature scheme $\SDS$ is defined for a group generator $\GGen$ and a hash function generator $\HGen$ and operates as shown on Figure \ref{fig:schnorr}.\end{definition}


\begin{figure}[ht]
\fbox{%
  \begin{minipage}{0.95\textwidth}
    % Top minipage
    
    \begin{minipage}[t]{0.05\textwidth}
    \end{minipage}\hfill
    \begin{minipage}[t]{0.4\textwidth}
    
 \textbf{\uline{\bm{$\SDS.\Setup(1^\lambda)$}}}\\[0.5ex]
\mbox{$(\G,g,q) \gets \GGen(1^\lambda)$}\\
\mbox{$\hash\gets  \HGen(q)$}\\
\mbox{$sp \gets (\G,g,q,\hash)$}\\
$\textbf{return } sp\\$


      \textbf{\uline{\bm{$\SDS.\Sign (sk,m)$}}}\\[0.5ex]
\mbox{$(\G,g,q,\hash,x):\subseteq sk$}\\
\mbox{$r\gets \Z_q;\ R\gets g^x ;\ c\gets \hash(R,X,m)$}\\
\mbox{$s\gets (r+c\cdot x) \mod q$}\\
\mbox{$\textbf{return } \sigma \gets(R,s)$}\\
    \end{minipage}
    \hfill
    \begin{minipage}[t]{0.40\textwidth}
      \textbf{\uline{\bm{$\SDS.\Keygen (sp)$}}}\\[0.5ex]
\mbox{$(\G,g,q) :\subseteq sp$}\\
\mbox{$x\gets \Z_q;\ X\gets g^x$}\\
\mbox{$sk,vk \gets (par,x),(par,X)$}\\
$\textbf{return } (sk,vk)\\$\\


%\vspace{.5cm}
       \textbf{\uline{\bm{$\SDS.\Ver(vk,m,\sigma)$}}}\\[0.5ex]
$(\G,g,q,\hash,X):\subseteq vk$\\
$(R,s)\gets \sigma; c\gets \hash(R,X,m)$\\
$\textbf{return } (g^s=R\cdot X^c)$

    \end{minipage}
    \hfill
    \begin{minipage}[t]{0.1\textwidth}
    \end{minipage}

  \end{minipage}%
}

\caption{The  Schnorr signature scheme \SDS, with group and hash generators \GGen,   \HGen.}\label{fig:schnorr}

\end{figure}


\begin{definition} A digital signature scheme $\DS$ is \emph{correct} if 
$$\Pr[sp\gets \DS.\Setup(1^\lambda); (sk,vk) \gets \DS.\Keygen(sp); m\gets \mathcal{M_{S}}:\DS.\Ver(vk,m,Sign(sk,m))=1 ] =1.$$
\end{definition}

\begin{definition} A digital signature scheme $\DS$ is \emph{strongly existentially unforgeable against chosen message attacks}  (sEUF-CMA)  if for all p.p.t.  adversaries \A, we have  
$$\Pr[\textsf{sEUF-CMA}^{\A}_{\DS}(1^\lambda)] $$
is negligible in $\lambda$ where the game $\textsf{GsEUF-CMA}^{\A}_{\DS}$ is defined in Figure \ref{fig:seufcma}. \label{def:seufcma}
\end{definition}


\begin{figure}[ht]
\fbox{%
  \begin{minipage}{0.95\textwidth}
    % Top minipage
    
    \begin{minipage}[t]{0.05\textwidth}
    \end{minipage}\hfill
    \begin{minipage}[t]{0.5\textwidth}

      \textbf{\uline{\bm{$\textsf{GsEUF-CMA}^{\A}_{\DS}(1^\lambda)$}}}\\[0.5ex]
$sp \gets \DS.\Setup(1^\lambda); Q\gets \emptyset$\\
$(sk,vk)\gets \DS.\Keygen(sp)\\ (m^*,\sigma^*)\gets \A^{\textsf{OSig}}(vk)$\\
$\textbf{return } (m^*,\sigma^*)\neq Q \land \DS.\Ver(vk,m^*,\sigma^*)$
    \end{minipage}
    \hfill
    \begin{minipage}[t]{0.35\textwidth}
      \textbf{\uline{\bm{$\textsf{OSig}(m)$}}}\\[0.5ex]
$\sigma \gets \DS.\Sign(skk,m)$\\
$Q \gets Q\cup \{(m,s)\}$\\
$\textbf{return } \sigma$
    \end{minipage}
    \hfill
    \begin{minipage}[t]{0.1\textwidth}
    \end{minipage}

  \end{minipage}%
}

\caption{The security experiment $\textsf{sEUF-CMA}^{\A}_{\DS}$ and supporting oracle $\textsf{OSig}$.}\label{fig:seufcma}

\end{figure}

\subsection{Parametrized Non-Interactive Zero Knowledge Arguments}

Following \cite{blindsigs}, we define non-interactive zero knowledge arguments with regards to \emph parametrized polynomial relations $\mathcal{P}: \{0,1\}^* \times \{0,1\}^* \times \{0,1\}^*  \to \{0,1\}$, where the first argument represents a parameter set $par$ (e.g. a group description). Given a value of $par$, we say that $w$ is a witness for statement $\theta$ if $\mathcal{P}(par,\theta,w)=1$, i.e. $R=R_{par}(\theta,w):=\mathcal{P}(par,\theta,w)$ is an NP-relation, and $\Lang=\Lang_{par}$ is an NP-language. A NIZK for a relation  $\mathcal{P}$ operates as follows:
\begin{itemize}

\item $\mathsf{Rel}(1^\lambda) \to par$. Generates a parameter set $par$ that defines $R$ and $\Lang$.
\item $\Setup(par)\to (\crs,\tau)$. Generates a common reference string (CRS) and trapdoor $\tau$ used by the simulator. We assume the CRS contains a description of $R$.
\item $\Prove(\crs,\theta,w)\to \pi$. Given a CRS $\crs$, statement $\theta$ and witness $w$ for $\theta$, produces a proof $\pi$.  
\item $\Ver(\crs,\theta,\pi)\to \{0,1\}$. Given a CRS $\crs$, a statement $\theta$ and a  proof $\pi$ outputs 1 or 0, accepting or rejecting the proof.
\item $\SimProve(\crs,\theta,\tau)$. Given a CRS $\crs$, statement $\theta$ and traproor $\tau$ for $\crs$, produces a simulated proof $\pi$.
\end{itemize}


\begin{definition} A system $\NArgr$ is perfectly correct if for all unbounded adversaries $\A$ 
\begin{align*}
\Pr [par \gets \NArg.\mathsf{Rel}(1^\lambda);  (\crs,\tau)\gets \NArg.\Setup(par); (\theta,w) \gets A(crs):  \\
\neg{}R(\theta,w) \lor \NArg.\Ver(\crs,\theta,\NArg.\Prove(\crs,\theta,w))]=1 \\
\end{align*}
\end{definition}

\begin{definition} A system $\NArgr$ is adaptably computationally sound if for all p.p.t. adversaries $\A$ 
\begin{align*}
\Pr [par \gets \NArg.\mathsf{Rel}(1^\lambda);  (\crs,\tau)\gets \NArg.\Setup(par); (\theta,\pi) \gets A(crs):  \\
\neg{}L(\theta) \land {}\NArg.\Ver(\crs,\theta,\pi)] \mbox{ is negligible in $\lambda$.}
\end{align*}
\end{definition}

\begin{definition} A system $\NArgr$ is  computationally zero-knowledge if for all p.p.t. adversaries $\A$ 
\begin{align*}
|\Pr [par \gets \NArg.\mathsf{Rel}(1^\lambda);  (\crs,\tau)\gets \NArg.\Setup(par); d\gets\{0,1\}:  \\
d = A^{\mathsf{OProve}_d}(crs)] - {1\over2}|\mbox{ is negligible in $\lambda$, where}\\
\mathsf{OProve}_0(\theta,w):= \textbf{if } \neg R(\theta,w) \textbf{ return }\bot; \textbf{ return } \NArg.\Prove(\theta,w), \mbox{and} \\
\mathsf{OProve}_1(\theta,w):= \textbf{if } \neg R(\theta,w) \textbf{ return }\bot; \textbf{ return } \NArg.\SimProve(\theta,\tau).
  \end{align*}
\end{definition}

\newpage

\section{Additional Ledger Syntax}
\label{sec:appendix}

In Figure~\ref{fig:notation:nonstandard} we introduce the standard ledger syntax that we use throughout.

\begin{figure}[h!tb]
  \begin{align*}
    \H{}
    & =~\bigcup_{n=0}^{\infty}\{0,1\}^{8n}
    & \mbox{the type of bytestrings }
    \\
    (a, b)
    & :~\Interval{A}
    & \text{intervals over a totally-ordered set $A$}
    \\
    \var{Key} \mapsto \var{Value}
    & \subseteq \{~ k \mapsto v ~\mid~ k \in \var{Key},~v \in \var{Value}~ \}
    & \text{finite map with unique keys}
    \\
    [a1 ; ...; ak]
    & :~[C]
    & \text{finite list with terms of type $C$}
    \\
    h :: t
    & :~[C]
    & \text{list with head $h$ and tail $t$}
    \\
    x \cup \fun{nothing}
    & :~A^?
    & \text{maybe type over $A$}
    \\
    a ~\{ ~\fun{field} = ~x~\}
    & :~A
    & \text{record of type A with $\fun{field}$ changed to $x$}
  \end{align*}
  \caption{Notation}
  \label{fig:notation:nonstandard}
\end{figure}

Figure~\ref{fig:eutxo-types} lists the primitives and derived types
that comprise the foundations of the EUTxO model,
along with some ancillary definitions.
(Outputs normally refer to transaction IDs by hash,
but we simplify here for clarity.)

\begin{figure}[h!tb]

\textsc{Ledger primitives}
\begin{displaymath}
\begin{array}{rlll}
  \checkSig &:&
  \Tx \to \pubkey \to \H \to \B \\
  & &\mbox{\emph{checks that a given key signed a transaction}}
\end{array}
\end{displaymath}

\textsc{Helper functions}
    \begin{align*}
        &\fun{txid} : \Tx \to \TxId \\
        &\fun{txid}~\var{tx} = \fun{hash}~(\var{tx} ~\{ ~\fun{sigKeys} = ~\fun{dom}~(\var{tx} . \sigs) ,~\sigs = ~\emptyset~\})
        \nextdef
        &\fun{toMap} : \N \to [\TxOut] \to (\N \mapsto \TxOut) \\
        &\fun{toMap}(\_,~[~]) ~~~~~~~~~~~~~= [~] \\
        &\fun{toMap}(\var{ix},~u~::~\var{outs}) = \{~\var{ix}\mapsto u~\} \cup \fun{toMap}(\var{ix}+1,~\var{outs}) \\
        & \mbox{\emph{constructs a map from a list of outputs}}
        \nextdef
        &\fun{mkOuts} : \Tx \to \UTxO \\
        &\fun{mkOuts}(tx) = \{~(\var{tx},~\var{ix}) \mapsto o~ \mid~(\var{ix} \mapsto o)\in~\fun{toMap}(0,~\var{tx}.\outputs)~\} \\
        &\mbox{\emph{constructs a UTxO set from a list of outputs of a given transaction}}
        \nextdef
        &\fun{MOf} : \N \to \N \to (A \to \B) \to [A] \to \B \\
        &\fun{MOf}~k~m~f~[~] = m~\leq~k \\
        &\fun{MOf}~k~m~f~( h~ ::~ t) = \fun{if}~ (m~\leq~k)~\fun{then}~\true~\fun{else}~(\fun{MOf}~(k~+~a)~m~f~t) \\
        &~~\where~~a~=~\fun{if}~ (f~(h))~\fun{then}~1~\fun{else}~0 \\
        &\mbox{\emph{returns $\true$ if enough elements of a list satisfy given function}}
    \end{align*}
\caption{Primitives and basic types for the $\UTxO_{ma}$ model}
\label{fig:eutxo-types}
\end{figure}
\newpage 
\subsection{Transaction Validation Rules}

A transaction $\var{tx}$ is \emph{valid} if it follows the following rules. 

\label{sec:check-tx}


\begin{itemize}
    \item[(i)] \textbf{The transaction has at least one input:}
  \[
  \var{tx}.\inputs~\neq~\{\}
  \]
    \item[(ii)] \textbf{The current slot is within transaction validity interval:}
  \[
  \var{slot} \in \var{tx}.\fun{validityInterval}
  \]
    \item[(iii)] \textbf{All outputs have positive values:}
  \[
  \forall o \in \var{tx}.\outputs,~\var{o}.\val > \emptyset
  \]
    \item[(iv)] \textbf{All output references of transaction inputs exist in the UTxO:}
  \[
  \var{tx}.\inputs~ \subseteq~ \fun{dom}~\var{utxo}
  \]
    \item[(v)] \textbf{Value is preserved:}
  \[
  \var{tx}.\mint +
  \sum_{i \in~\var{tx}.\inputs,~(i~\mapsto~o) \in~\var{utxo}} \var{o}.\val =
  \sum_{o \in~\var{tx}.\outputs} \var{o}.\val ~+~\fun{toValue}~(\var{tx}.\fun{fee})
  \]
    \item[(vii)] \textbf{All inputs validate:}
  \[
  \forall~ i \in \var{tx}.\inputs,~i \mapsto (s, v) \in \var{utxo},~
     \applyScript{s}(\fun{dom}~(\var{tx}.\sigs),~\var{tx}.\validityInterval) = \true
  \]
    \item[(ix)] \textbf{All minting scripts validate:}
  \[
  \forall~ p\mapsto \_ \in \var{tx}.\mint,~ \applyScript{p} (\fun{dom}~(\var{tx}.\sigs),~\var{tx}.\validityInterval) = \true
  \]
    \item[(x)] \textbf{All signatures are correct:}
  \[
  \forall~ (pk \mapsto s) \in \var{tx}.\sigs,~ \checkSig (\var{tx}, pk, s) = \true
  \]
  \item[(i)] \textbf{The fee is sufficient:}
  \[
    \var{s}.\fun{minfee} \leq \var{tx}.\fun{fee} 
  \]
  \end{itemize}

\subsection{Script construction and Evaluation}
\label{sec:check-script}

In Figure \ref{fig:script} we present the constructors and evaluation rules for scripts, and in Figure \ref{fig:mktospec} we explain the transaction building function \fun{mkToSpec}.

\begin{figure}
    \textsc{Constructors of $\Script$} 
\begin{displaymath}
\begin{array}{rlll}
    \fun{RequireMOf}         &: \N \to [\Script] &\to~ \Script \\
    \fun{RequireSig}         &: \pubkey      &\to~ \Script \\
    \fun{RequireTimeStart}   &: \Slot        &\to~ \Script \\
    \fun{RequireTimeExpire}  &: \Slot        &\to~ \Script \\
\end{array}
\end{displaymath}
\nextdef
\textsc{Evaluation of $\Script$} 
\begin{displaymath}
\begin{array}{rlll}
  \applyScript{\_} &:& \Script \to ((\type{Set} \pubkey) \times (Slot \times Slot)) \to \B \\
  \applyScript{\fun{RequireMOf}~n~ls} (\var{khs}, (t1, t2)) &=&  \fun{MOf}~0~ n ~(\applyScript{\_}~(\var{khs}, (t1, t2))) ~ls \\
  \applyScript{\fun{RequireSig}~k} (\var{khs}, (t1, t2)) &=& k~\in~\var{khs} \\
  \applyScript{\fun{RequireTimeStart}~t1'} (\var{khs}, (t1, t2)) &=& t1'~\leq~t1 \\
  \applyScript{\fun{RequireTimeExpire}~t2'} (\var{khs}, (t1, t2)) &=& t2~\leq~t2' 
\end{array}
\end{displaymath}
\caption{$\Script$ constructors and evaluation}
\label{fig:script}
\end{figure}

\begin{figure}
      \textsc{Trivial transaction $\fun{initTx}_{a,mf}$} 
\begin{displaymath}
\begin{array}{rlll}
  \fun{initTx}_{a,mf}  &=& \{ ~\inputs = \emptyset,\\
  & &\ \outputs = [ (a , \fun{tip})],\\
  & &\ \fun{validityInterval} = [\fun{nothing} , \fun{nothing}],\\
  & &\ \mint = 0,\\
  & &\ \fun{fee} = mf \\
  & &\ \fun{aux} = []~\\
  & &\ \sigs = \emptyset~ \}
\end{array}
\end{displaymath}
\nextdef
\textsc{Input selection function} 
\begin{displaymath}
\begin{array}{rlll}
  \fun{mkIns}  &:& (\LState \times (\type{Set} \TxIn) \times \Value \times \Script) \to (\type{Set} \TxIn) \\
  \fun{mkIns} ~(l,~i,~v,~s) &=& \fun{if}~\neg~(v \leq 0) ~ \fun{then} \\
  & & ~~~~ \fun{if}~(v > 0)~\fun{then}~\\
  & & ~~~~ ~~~~\fun{mkIns}~(l~\setminus~(\{j \mapsto \_\}),~i \cup j,~v~-~(u(j) . \val) ,~s) \\
  & & ~~~~ \fun{else}~i\\
  & & \fun{else } \\
  & & ~~~~\fun{nothing} \\
  & & \fun{where}~\\
  & & ~~~j = \fun{pickInput}~l~s \\
  & & ~~~u = \fun{utxo}~l
\end{array}
\end{displaymath}
\nextdef
\textsc{Auxiliary $\fun{mkToSpec'}$ definition} 
\begin{displaymath}
\begin{array}{rlll}
\fun{mkToSpec'} &:& \LState \to \mathcal{I}_\mathsf{post} \to \Tx \to \Tx^? \\
\fun{mkToSpec'}~l~(\fun{MustMint}~v) (\var{tx}) &=&  \var{tx}~\{ \mint~=~v ~+~ \var{tx} . \mint , \fun{sigs} = tx.\fun{sigs} \cup \fun{getSigsVal}~v ,\\
& & ~~~~\fun{validityInterval} = \fun{restrictIntervalVal}~tx.\fun{validityInterval}~v\} \\
\fun{mkToSpec'}~l~(\fun{SpendFrom}~s) (\var{tx}) &=& \var{tx}~\{ \outputs~=~ \var{tx} . \inputs~\\
& & ~~~~\cup~\fun{newIns} , \fun{sigs} = tx.\fun{sigs} \cup \fun{getSigsUTxO}~\fun{newIns}~l ,\\ 
& & ~~~~ \fun{validityInterval} = \\ 
& &~~~~~~~~\fun{restrictIntervalUTxO}~tx.\fun{validityInterval}~\fun{newIns}~l\} \\
& & ~~~~\where \\
& & ~~~~\fun{newIns} = \fun{mkIns}~l~(\var{tx} . \inputs)~(\fun{produced} - \fun{consumed})~ s \\
\fun{mkToSpec'}~l~ (\fun{MaxInterval}~i) (\var{tx}) &=& \var{tx}~\{ \validityInterval~=~(l . \fun{slot}, \\
& & ~~~~\fun{min}~\{l . \fun{slot}~+~i , \var{tx} . \fun{validityInterval}_2)\} \}  \\
\fun{mkToSpec'}~l~ (\fun{PayTo}~(s, v)) (\var{tx}) &=& \var{tx}~\{ \outputs~=~ \var{tx} . \outputs~\cup~ (s, v) \}   \\
\fun{mkToSpec'}~l~ (\fun{ChangeTo}~s) (\var{tx}) &=& \fun{if}~ \fun{consumed} - \fun{produced} ~>~0 ~\\ 
& & ~~~~\fun{then}~ \var{tx}~\{ \outputs~=~ \var{tx} . \outputs~\\
& & ~~~~\cup~ \{(s,~ \fun{consumed} - \fun{produced})\} \} ~\fun{else}~\fun{nothing}   \\
\fun{mkToSpec'}~l~ (\fun{MaxFee}~f)~ (\var{tx}) &=&  \fun{if}~ \var{tx} . \fun{fee} \leq f~\\ 
& & ~~~~\fun{then}~\var{tx}~\fun{else}~\fun{nothing} \\
\fun{mkToSpec'}~l~ (\fun{AndExps}~[a1 ; a2 ; ... ; ak]) (\var{tx}) &=& \fun{mkToSpec'}~l~ak~(... ~(\fun{mkToSpec'}~l~a2~(\fun{mkToSpec'}~l~a1~\var{tx})))
\end{array}
\end{displaymath}
\nextdef 
\textsc{$\fun{mkToSpec}$ definition} 
\begin{displaymath}
\begin{array}{rlll}
\fun{mkToSpec} &:& (\LState \times \mathcal{I}_\mathsf{post}) \to \Tx^? \\
\fun{mkToSpec}~(l , i) &=& \fun{mkToSpec'}~l~i~\fun{initTx}_{a,mf}
\end{array}
\end{displaymath}
\caption{Building transactions according to the specific intent}
\label{fig:mktospec}
\end{figure}




\begin{figure}
\begin{displaymath}
\begin{array}{rlll}
  \fun{txi}_1 &=&(\outputs = \{~(s, t)~\},\\
  & &\ \fun{validityInterval} = [ \fun{slot}~l , (\fun{slot}~l) + j ],\\
  & &\ \mint = t,\\
  & &\ \fun{fee} = \fun{minfee}~l \\
  & &\ \fun{sigKeys} = \fun{getSignersVal}~t) 
  \nextdef 
  \fun{txi}_2 &=& (\outputs = \{ (\fun{RequireSig}~k2,~x)~,~(\fun{RequireSig}~k1~,\\& & ~(\fun{balance}~(\fun{mkIns}~l~\{\}~x~ (\fun{RequireSig}~k1))) - x) \} ,\\
  & &\ \fun{validityInterval}  = [~ \fun{nothing}~ , \fun{nothing}~],\\
  & &\ \mint  = \{~\} ,\\
  & &\ \fun{fee}  = \fun{minfee}~l \\
  & &\ \fun{sigKeys}  = \{~k1~\} )
\end{array}
\end{displaymath}
\caption{Abstract transaction examples}
\label{fig:txs}
\end{figure}



\end{document}
