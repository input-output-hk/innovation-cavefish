\documentclass[runningheads]{llncs}
%
\usepackage[T1]{fontenc}
\usepackage{amssymb}
% T1 fonts will be used to generate the final print and online PDFs,
% so please use T1 fonts in your manuscript whenever possible.
% Other font encondings may result in incorrect characters.
%
\usepackage{graphicx}
\usepackage{stmaryrd}
\usepackage{amsmath}
\usepackage{caption}
% Used for displaying a sample figure. If possible, figure files should
% be included in EPS format.
%
% If you use the hyperref package, please uncomment the following two lines
% to display URLs in blue roman font according to Springer's eBook style:
%\usepackage{color}
%\renewcommand\UrlFont{\color{blue}\rmfamily}
%\urlstyle{rm}

%formatting
\newcommand{\var}[1]{\ensuremath{\mathit{#1}}}
\newcommand{\fun}[1]{\ensuremath{\mathsf{#1}}}
\newcommand{\type}[1]{\ensuremath{\mathsf{#1}}}
\renewcommand\H{\mathbb{H}}
\newcommand{\Interval}[1]{\type{Interval}[#1]}
\newcommand{\nextdef}{\ensuremath{\\[1em]}}
\newcommand{\where}{\ensuremath{~ ~ \mathbf{where}~ ~ }}

% Macros for eutxo things.
\newcommand{\tx}{\fun{tx}}
\newcommand{\TxId}{\type{TxId}}
\newcommand{\TxIn}{\type{TxIn}}
\newcommand{\Header}{\type{Header}}
\newcommand{\Block}{\type{Block}}
\newcommand{\LState}{\type{LState}}
\newcommand{\Tx}{\type{Tx}}
\newcommand{\Slot}{\type{Slot}}
\newcommand{\B}{\mathbb{B}}
\newcommand{\TxOut}{\type{TxOut}}
\newcommand{\UTxO}{\type{UTxO}}
\newcommand{\txId}{\msf{txId}}
\newcommand{\txrefid}{\fun{id}}
\newcommand{\Address}{\ensuremath{\s{Address}}}
\newcommand{\DataHash}{\ensuremath{\s{DataHash}}}
\newcommand{\hashData}{\fun{dataHash}}
\newcommand{\idx}{\fun{index}}
\newcommand{\inputs}{\fun{inputs}}
\newcommand{\outputs}{\fun{outputs}}
\newcommand{\Out}{\type{Output}}
\newcommand{\validityInterval}{\fun{validityInterval}}
\newcommand{\scripts}{\fun{scripts}}
\newcommand{\mint}{\fun{mint}}
\newcommand{\mintScripts}{\fun{mintScripts}}
\newcommand{\mintScsRdmrs}{\fun{mintRdmrs}}
\newcommand{\mintRdmrs}{\fun{mintRdmrs}}
\newcommand{\sigs}{\fun{sigs}}
\newcommand{\fee}{\fun{fee}}
\newcommand{\addr}{\fun{addr}}
\newcommand{\pubkey}{\fun{PubKey}}
\newcommand{\val}{\fun{value}}  %% \value is already defined
\newcommand{\Value}{\type{Value}}
\newcommand{\Redeemer}{\type{Redeemer}}
\newcommand{\TxOutRef}{\type{TxIn}}
\newcommand{\ScriptContext}{\type{ScriptContext}}
\newcommand{\ScriptPurpose}{\type{ScriptPurpose}}
\newcommand{\Datum}{\type{Datum}}
\newcommand{\DCert}{\type{DCert}}
\newcommand{\LCTx}{\type{LCTx}}
\newcommand{\TxInInfo}{\type{TxInInfo}}


\newcommand{\validator}{\fun{validator}}
\newcommand{\redeemer}{\fun{redeemer}}
\newcommand{\datum}{\fun{datum}}
\newcommand{\datumHash}{\fun{datumHash}}
\newcommand{\datumWits}{\fun{datumWitnesses}}
\newcommand{\Data}{\ensuremath{\s{Data}}}
\newcommand{\Input}{\ensuremath{\s{Input}}}
\newcommand{\Output}{\type{Output}}
\newcommand{\OutputRef}{\fun{OutputRef}}
\newcommand{\Signature}{\type{Signature}}
\newcommand{\Ledger}{\ensuremath{\s{Ledger}}}

\newcommand{\outputref}{\fun{outputRef}}
\newcommand{\outputrefs}{\fun{outputRefs}}
\newcommand{\txin}{\fun{in}}
\newcommand{\id}{\fun{id}}
\newcommand{\lookupTx}{\msf{lookupTx}}
\newcommand{\getSpent}{\msf{getSpentOutput}}

% \newcommand{\Tick}{\ensuremath{\s{Tick}}}
\newcommand{\Tick}{\type{Slot}}
\newcommand{\Script}{\type{Script}}
\newcommand{\spent}{\msf{spentOutputs}}
\newcommand{\unspent}{\msf{unspentOutputs}}
\newcommand{\txunspent}{\msf{unspentTxOutputs}}
\newcommand{\eutxotx}{\msf{Tx}}

\newcommand{\consumes}[1]{\msf{consumes(#1)}}
\newcommand{\consumesOne}[1]{\msf{consumesOne(#1)}}
\newcommand{\cid}{\fun{cid}}
\newcommand{\inputValue}{\fun{inputValue}}
\newcommand{\rMin}{r_{\fun{min}}}
\newcommand{\rMax}{r_{\fun{max}}}

\newcommand{\utxotx}{\msf{Tx}}

\newcommand{\N}{\mathbb{N}}

\newcommand{\Quantity}{\type{Quantity}}
\newcommand{\TokenName}{\type{TokenName}}
\newcommand{\Asset}{\ensuremath{\s{Asset}}}
\newcommand{\AssetID}{\type{AssetID}}
\newcommand{\Policy}{\type{Policy}}
\newcommand{\Quantities}{\ensuremath{\s{Quantities}}}
\newcommand{\nativeCur}{\ensuremath{\mathrm{nativeC}}}
\newcommand{\nativeTok}{\ensuremath{\mathrm{nativeT}}}
\newcommand{\valC}{\mkValidator{\mathcal{C}}}
\newcommand{\polC}{\mkPolicy{\mathcal{C}}}
\newcommand\mkValidator[1]{\msf{validator}_#1}
\newcommand\mkPolicy[1]{\msf{policy}_#1}

\newcommand{\PublicKey}{\ensuremath{\s{PubKey}}}
\newcommand{\PubKey}{\ensuremath{\s{PubKey}}}
\newcommand{\PrivateKey}{\ensuremath{\s{PrivateKey}}}

\newcommand{\mkContext}{\ensuremath{\s{mkContext}}}
\newcommand{\mkTxInfo}{\ensuremath{\s{mkTxInfo}}}
\newcommand{\mkVlContext}{\ensuremath{\s{mkValidatorContext}}}
\newcommand{\mkMpsContext}{\ensuremath{\s{mkPolicyContext}}}
\newcommand{\checkSig}{\type{checkSig}}

\newcommand{\applyScript}[1]{\ensuremath{\llbracket#1\rrbracket}}
\newcommand{\applyMPScript}[1]{\ensuremath{\llbracket#1\rrbracket}}

\newcommand{\true}{\type{True}}
\newcommand{\false}{\type{False}}
\newcommand{\True}{\type{True}}
\newcommand{\False}{\type{False}}

\newcommand{\leteq}{:=}

\begin{document}
%
\title{Light Clients for Building UTxO Ledger Transactions}
%
%\titlerunning{Abbreviated paper title} 
% If the paper title is too long for the running head, you can set
% an abbreviated paper title here
%
\author{Pyrros Chaidos\inst{1}\orcidID{0000-1111-2222-3333} \and
Aggelos Kiayias\inst{1}\orcidID{1111-2222-3333-4444} \and
Marc Roeschlin\inst{1}\orcidID{0000-1111-2222-3333} \and
Polina Vinogradova\inst{1}\orcidID{0000-0003-3271-3841}}
%
\authorrunning{P. Chaidos et al.}
% First names are abbreviated in the running head.
% If there are more than two authors, 'et al.' is used.
%
\institute{Input Output, Global
\email{firstname.lastname@iohk.io}
\url{iohk.io}}
%
\maketitle              % typeset the header of the contribution
%
\begin{abstract}
The abstract should briefly summarize the contents of the paper in
150--250 words.

\keywords{First keyword  \and Second keyword \and Another keyword.}
\end{abstract}
%
%
%
\section{Introduction}

What sets us apart?

\begin{itemize}
    \item Atomicity of payment+service
    
    \item permission-less, decentralized

    \item Model for (trustless) 2-party transaction construction rather than proving things about chain/ledger state

    \item Do not require establishing a relationship with SP or any other set-up
    
    \item inherent timeliness of transaction construction incentivized by SPs desired to earn 
    their tip. This is in contrast with the possibility of stale info provided from old Mithril snapshots in other LC models
\end{itemize}

We claim that our work can be used by any UTxO or EUTxO blockchain (with some adjustments to the details of intent specification). 
We use blind signatures \cite{blindsigs}

\section{Related Work}


Our approach provides facilitates transaction submission for a light client that is not aware of the entire history of a blockchain. 
Therefore, we compare our work with existing solutions for light clients as well as mechanisms that allow the interaction with a blockchain when in resource constrained operation.
Our approach is different from a more ``traditional'' (i.e., what commonly is referred to) light client that has the primary goal of syncing to the blockchain in order to acquire the information necessary to interact with a smart contract such as to submit a transaction. We summarize here the most important concepts and works in the area of light clients below:



Most related works and ideas are \emph{intents and solver networks} that attempt to establish a relationship between solvers and users via their (light) clients~\cite{ethresearch-solvers}.
A light client issues an intent via as an abstracted transaction object and solvers process the intent incentivized by transaction fee or intent execution reward. The users are then free to accept or reject a solver’s proposal.
Outsourcing transaction creation to a solver does not need for user to bridge assets cross-chain, as chains are abstracted away. In case the solvers are required to provide a deposit, slashing guarantees honest behavior of a rational actor.
While concepts around solver networks are relatively new there is currently no universal standard governing the specification of intents processing of 

We also briefly cover \emph{payment channels}, which also constitute a concept similar to our approach:



Compare our approach  with :
\begin{itemize}
    \item "Free" websites monitoring the chain -- mention they lack long-term sustainability.
    \item Bridges (trustless and trusted)
    \item Payment channels 
    \item LCs that operate on single-prover model (eg. with an established relationship via deposit)
    \item LCs that operate on multi-prover model
    \item LC SoK
    \item Solver networks

\subsection{Technical Background}
UTxO model
\end{itemize}

\section{Ledger Model}
\label{sec:model}

The ledger model to which we tailor our light client design is 
a UTxO ledger with multi-asset support ($\mathsf{UTxO}_{ma}$), first introduced in \cite{utxoma}. 
The UTxO ledger model, such as the one used by BitCoin \cite{bitcoin}, Ergo \cite{ergo},
and Cardano~\cite{full-cardano}, maintains a record, called the \emph{UTxO set}, 
of transaction outputs added 
by transactions that have been applied throughout its history, but not yet spent by 
subsequent transactions. We chose the $\mathsf{UTxO}_{ma}$ ledger because it allows us to demonstrate 
relevant usecases of our light client design (which would also work for a single-asset UTxO ledger),
without introducing unnecessarily complexity of the Extended UTxO ledger. 
For completeness, and in order to establish notation, 
we include an overview of the $\mathsf{UTxO}_{ma}$ model. For additional notation 
explanation, see Figure \ref{fig:notation:nonstandard}.

\textbf{Blocks and Ledger States.} A block $\Block = \Header \times [\Tx]$ is a data 
structure used to update the state of the ledger
by applying a list of transactions $\var{lstx} \in [\Tx]$ contained in the block, 
as well as doing some other 
checks and updates we do not model here. 
Among other data, the block header contains a slot number field $\fun{slot}~:~\Header \to \Slot$,
which represents the blockchain time at which the block is produced. 

The ledger state is a
data structure which is updated by applying blocks incoming on the network.
The ledger state $\LState$ contains (among other data we do not model here)
the UTxO set field, $\fun{utxo}~:~\LState \to \UTxO$.
It also contains the parameter $\fun{minfee}~:~\LState \in \N$, which is the minimum 
fee a transaction must pay.
A block $b$ can extend the blockchain (i.e. update the current state $s \in \LState$) whenever the function 

\[ \fun{checkBlock}~:~\LState \times \Block \to \B \]

applied as $\fun{checkBlock}~(s, b)$ returns $\true$. Then, the updated block state 
is computed by $\fun{updateState}~(s, b)$. We do not give full specifications of 
$\fun{checkBlock}$ or $\fun{updateState}$, as they are not required to model our light client approach.

\textbf{Full Nodes.}
Let $s_0$ be some verified state, e.g. a genesis state, 
or a verified checkpoint state, and suppose $[b_0, ..., b_k]$ is a list of blocks 
that have been disseminated across the network since the time slot of $s_0$. 
We assume that a \emph{full node} is one that is to able to (1) compute the current state $s_{k}$ by applying 
them in sequence, i.e. computing $s_{i+1} = \fun{updateState}~(s_i, b_i)$ for $0 \leq i < k$,
(2) is able to determine if all of the blocks in the list are valid
(i.e. $\fun{checkBlock}~(s_i, b_i) = \true$), and (3) is able to listen on the network for 
new blocks. If all the blocks are valid, we say 
that $[b_1, ..., b_k]$ forms a valid blockchain.


\textbf{Multi-asset Support}. A ledger which supports transacting with not only the 
primary currency, e.g. BitCoin on the BitCoin 
platform, or Ada on Cardano, but also other types of currencies, is called 
a \emph{multi-asset ledger}. 
Each asset is uniquely identified by an $\AssetID \leteq \Policy \times \TokenName$.
Arbitrary combinations of assets are specified using a finitely-supported map 
$\Value \leteq \AssetID \mapsto \Quantity$, where, for a given $v \in \Value$,
all assets whose IDs are not 
included in the domain of $v$ are assumed to have quantity $0 \in \N$. 
This data structure 
has a partial order $\leq$, and forms a group under addition $+$, with zero being 
the empty map $\emptyset$. In our model, all assets are user-defined, meaning that a
user can introduce any asset into circulation so long as minting of this asset is allowed 
by its minting policy (which may be defined by the user themselves).

\textbf{Ledger State and the UTxO Set.} 
The state of a UTxO-based ledger 
necessarily contains a UTxO set. While realistic ledgers often contain additional 
information in their state, in our model, the ledger state is just the UTxO set.
The UTxO set is a finite map, $\UTxO \leteq \TxIn \mapsto \TxOut$. 
A transaction updates the UTxO set by either adding and removing entries. 

\textbf{Transactions}. A transaction is the following data structure :

\begin{displaymath}
    \begin{array}{rll}
        \Tx &=&(\inputs: \type{Set}~{\TxIn},\\
        & &\ \outputs: [\TxOut],\\
        & &\ \fun{validityInterval}: \Interval{\Tick},\\
        & &\ \mint: \Value,\\
        & &\ \fun{fee}: \N \\
        & &\ \sigs: \Signature)
    \end{array}
\end{displaymath}

An input $(\var{txid}, \var{ix}) \in \TxIn \leteq \TxId \times \N$ is a pair 
of a transaction ID and a natural number. When a transaction is applied to a UTxO 
set, its set of $\inputs$ is used to identify the entries
which the transaction is removing from the set. In each input,
$\var{txid} \in \TxId \leteq \H{}$ 
is the hash of a (previous) transaction that added that entry to the UTxO, and $\var{ix}$ 
is the index of that output in the list of outputs of that transaction.

An output $(s, v) \in \TxOut \leteq \Script \times \Value$ is a pair of a script $s$
which specifies some constraints that are checked when the output is spent, 
and the assets $v$ contained in the output. The list $\outputs$ of outputs of a transaction 
$\var{tx}$ is used to construct a set of UTxO entries that will be added to the UTxO set, such 
that the unique identifier $\var{txin}$ of each output $o \in \outputs~\var{tx}$ consists of the 
transaction hash $\fun{txid}~\var{tx}$, and the index of $o$ in the list $\outputs~\var{tx}$.
The entires added to the UTxO set by $\var{tx}$ are computed in this way by $\fun{mkOuts}$, 
see Figure \ref{fig:eutxo-types}. 

The interval $\fun{validityInterval}$ specifies the range of slot numbers for which a transaction
can be valid. The field $\sigs : \Signature \leteq \pubkey \mapsto \H $
is a set of public keys, associated with their 
signatures on the the transaction (excluding $\sigs$ itself). 
The $\fun{fee}$ is the amount of primary currency a transaction pays as a system fee,
which is checked to be at least the required fee $\fun{minfee}$.
The $\mint$ field represents the assets being minted or burned by the transaction. 
Assets with positive quantities are said to be minted, while those with negative quantities 
are burned. When a transaction is applied, the constraints specified by every 
$p \in \Policy \leteq \Script$ of each type of asset specified in this field are 
checked to make sure minting/burning of this type and quantity of asset is allowed. 

\textbf{Ledger State Update. } Given a UTxO set $\var{utxo}$ and a transaction $\var{tx}$, the function 
$\fun{updateUTxO} : \UTxO \times \Tx \to \UTxO$
computes the updated UTxO set by adding and removing the appropriate entries :

\[\fun{updateUTxO}~(\var{utxo},~\var{tx})~=~\{~i\mapsto o \in \var{utxo} ~\mid~ i \notin \fun{inputs}~(tx)~\} \cup \fun{mkOuts}(tx)\]
    
While an update to the UTxO set can be computed for any transaction, only transactions that are
\emph{valid} for a given set are allowed to perform an update to the ledger state. For a given
$\var{utxo}$ set, a transaction $\var{tx}$ 
is valid whenever the function $\fun{checkTx} : (\Slot \times \N) \times \UTxO \times \Tx \to \B$,
applied as $\fun{checkTx}~((\var{slot}, \var{fee}), \var{utxo},~\var{tx})$, returns $\True$. The $\fun{checkTx}$
function is the conjunction of the constraints specified in Section \ref{sec:check-tx}.
For a given block $b$ and state $s$, the function $\fun{updateState}~(s, b)$ updates 
$s$ with the list of transactions $\pi_2~b = [\var{tx}_1 ; ... ; \var{tx}_k ]$ in such a way that 
the update to the UTxO set contained in $s$ is computed by applying the transactions in sequence, i.e.

\[ \fun{updateUTxO}~(\fun{updateUTxO}~((... \fun{utxo}~s ...),~\var{tx}_{k-1}),~\var{tx}_k) = \fun{utxo}~(\fun{updateState}~(s, b)) \]

For each transaction in the list, $\fun{checkTx}~((\fun{slot}~b, \fun{fee}~s), \var{utxo}_i,~\var{tx}_i) = \true$ is first checked, 
and the entire block is considered invalid if this check fails.
As part of checking transaction validity, constraints of every $\Script$ run by the transaction are checked. 
The constructors and evaluation of $\Script$ is given in Figure \ref{fig:script}, 
and $\fun{MOf}$ is given in \ref{fig:eutxo-types}. A script, for a given set of 
signer keys $\var{khs}$ and slot numbers $s1, s2$, can be defined to check that 
(some specific) $m$ of them have signed the transaction (are included in the domain of $\Signature$), 
and/or that the validity interval of the transaction starts after $s1$ and/or ends before $s2$.








\section{Light Client Specification}
 
 What is a light client?
 \begin{itemize}
 
 \item User - LC interface
 
 \item LC characteristics/ limitations
   \begin{itemize}
\item What are the capabilities of a light client?

\item Does it remember all addresses it has been paid at (tx history)?

\item Do we assume that viewing keys exist? Can we assume LC can generate first x addresses from its private key in a
deterministic way? 

\item Is light client allowed to maintain state, and what state can they maintain if so? Secret key is the minimum state.
\item Sanity test: if we dismiss all requirements we should recover a full client.
\end{itemize}

 \item LC - Full Node interactions
  \begin{itemize}
  \item Protocols to support user requests
  \item Security reqs (protecting integrity/ privacy/ SPO revenue?)
\end{itemize}
 \end{itemize}
 
Can we describe the differences between light wallets, bridges and light nodes in our framework? (Probably yes). 

How can we formalize the intent of a light client without revealing secret key?

Can we have viewing keys?


\section{Threat Model}

\begin{itemize}
    \item Client finds out too much from SP answer and can submit tx without payment to SP (that’s why the inputs must be hidden)

    \item SP lies to client about the UTxOs being spent by the tx, and tricks them into doing something they didn’t want to do (that’s why 0-knowledge proof that outputs were correctly specified)

    \item Suboptimal transactions possibility : 
    Do we want to let users specify what they want optimized for in their intents (e.g. minimize fee or find best price on exchange offer)? 
    Can we assume that competition across SPs will enable client to get a better response?
\end{itemize}


\section{Intent Specification}
Intent (DSL or predicate to describe intent of the client, ie what they want to do to the ledger state).

Give example of :

\begin{itemize}
    \item intent to mint some tokens before a deadline
    \item intent to pay x from key k1 to k2 (ie script $\fun{RequireSig}~k1$ to $\fun{RequireSig}~k2$)
\end{itemize}

\section{Light Client Protocols}

\section{Analysis}


\section{Conclusion}



\begin{credits}
\subsubsection{\ackname} A bold run-in heading in small font size at the end of the paper is
used for general acknowledgments, for example: This study was funded
by X (grant number Y).

\subsubsection{\discintname}
It is now necessary to declare any competing interests or to specifically
state that the authors have no competing interests. Please place the
statement with a bold run-in heading in small font size beneath the
(optional) acknowledgments\footnote{If EquinOCS, our proceedings submission
system, is used, then the disclaimer can be provided directly in the system.},
for example: The authors have no competing interests to declare that are
relevant to the content of this article. Or: Author A has received research
grants from Company W. Author B has received a speaker honorarium from
Company X and owns stock in Company Y. Author C is a member of committee Z.
\end{credits}
%
% ---- Bibliography ----
%
% BibTeX users should specify bibliography style 'splncs04'.
% References will then be sorted and formatted in the correct style.
%
\bibliographystyle{splncs04}
\bibliography{lcbib}

\appendix
\section{Expanded Technical Backgroud}
\label{app:techbg}

\subsection{Discrete Log Groups}
\begin{definition} A group generator \GGen{} is a probabilistic polynomial time (p.p.t.) algorithm with input a security parameter $\lambda$ and outputs a group description $\G,g,q$ such that $\G$ is a group of prime order $q\approx 2^\lambda$ with generator $g$. We say that the discrete logarithm problem is hard w.r.t. \GGen{} if for all p.p.t $\A$ we have that

$$\Pr[(\G,g,q) \gets \GGen(1^\lambda); t\gets \Z_q; h\gets g^t: t=A(\G,g,q,h) ] \mbox{ is negligible in }\lambda.$$
\end{definition}

\subsection{ Public Key encryption and Schnorr Signatures}

A public key encryption scheme $\PKE$ comprises a set of polynomial time algorithms $\Keygen,\Enc,\Dec$ with the following syntax:
\begin{itemize}
\item $\Keygen(1^\lambda) \to (ek,dk)$. Creates a public/private keypair $ek,dk$. The encryption key $ek$ also defines the message space $\mathcal{M}$
\item $\Enc(ek,m;\rho) \to C$. Encrypts a message $m\in \mathcal{M}$ under the public encryption key $ek$. \item $\Dec(dk,C) \to m$. Decrypts a ciphertext $C$ using the private decryption key $dk$.
\end{itemize}

\begin{definition} A public key encryption scheme $\PKE$ is \emph{correct} if 
$$\Pr[ (ek,dk) \gets \PKE.\Keygen(1^\lambda); m\gets \mathcal{M}:\PKE.\Dec(sk,\PKE.\Enc(pk,m))=m ] =1.$$
\end{definition}

\begin{definition} A public key encryption scheme $\PKE$ is \emph{IND-CPA secure} if for all stateful p.p.t. adversaries \A, the difference
$$\left| \Pr[ (ek,dk) \gets \PKE.\Keygen(1^\lambda); (m_0,m_1)\gets A(ek);d\gets\{0,1\}; c^*\gets \PKE.\Enc(ek,m_d):A(c^*)=d ] - {1\over2} \right|$$
is negligible in $\lambda$.
\end{definition}

\begin{definition} The Schnorr digital signature scheme $\SDS$ is defined for a group generator $\GGen$ and a hash function generator $\HGen$ and operates as shown on Figure \ref{fig:schnorr}.\end{definition}


\begin{figure}[ht]
\fbox{%
  \begin{minipage}{0.95\textwidth}
    % Top minipage
    
    \begin{minipage}[t]{0.05\textwidth}
    \end{minipage}\hfill
    \begin{minipage}[t]{0.4\textwidth}
    
 \textbf{\uline{\bm{$\SDS.\Setup(1^\lambda)$}}}\\[0.5ex]
\mbox{$(\G,g,q) \gets \GGen(1^\lambda)$}\\
\mbox{$\hash\gets  \HGen(q)$}\\
\mbox{$sp \gets (\G,g,q,\hash)$}\\
$\textbf{return } sp\\$


      \textbf{\uline{\bm{$\SDS.\Sign (sk,m)$}}}\\[0.5ex]
\mbox{$(\G,g,q,\hash,x):\subseteq sk$}\\
\mbox{$r\gets \Z_q;\ R\gets g^x ;\ c\gets \hash(R,X,m)$}\\
\mbox{$s\gets (r+c\cdot x) \mod q$}\\
\mbox{$\textbf{return } \sigma \gets(R,s)$}\\
    \end{minipage}
    \hfill
    \begin{minipage}[t]{0.40\textwidth}
      \textbf{\uline{\bm{$\SDS.\Keygen (sp)$}}}\\[0.5ex]
\mbox{$(\G,g,q) :\subseteq sp$}\\
\mbox{$x\gets \Z_q;\ X\gets g^x$}\\
\mbox{$sk,vk \gets (par,x),(par,X)$}\\
$\textbf{return } (sk,vk)\\$\\


%\vspace{.5cm}
       \textbf{\uline{\bm{$\SDS.\Ver(vk,m,\sigma)$}}}\\[0.5ex]
$(\G,g,q,\hash,X):\subseteq vk$\\
$(R,s)\gets \sigma; c\gets \hash(R,X,m)$\\
$\textbf{return } (g^s=R\cdot X^c)$

    \end{minipage}
    \hfill
    \begin{minipage}[t]{0.1\textwidth}
    \end{minipage}

  \end{minipage}%
}

\caption{The  Schnorr signature scheme \SDS, with group and hash generators \GGen,   \HGen.}\label{fig:schnorr}

\end{figure}


\begin{definition} A digital signature scheme $\DS$ is \emph{correct} if 
$$\Pr[sp\gets \DS.\Setup(1^\lambda); (sk,vk) \gets \DS.\Keygen(sp); m\gets \mathcal{M_{S}}:\DS.\Ver(vk,m,Sign(sk,m))=1 ] =1.$$
\end{definition}

\begin{definition} A digital signature scheme $\DS$ is \emph{strongly existentially unforgeable against chosen message attacks}  (sEUF-CMA)  if for all p.p.t.  adversaries \A, we have  
$$\Pr[\textsf{sEUF-CMA}^{\A}_{\DS}(1^\lambda)] $$
is negligible in $\lambda$ where the game $\textsf{GsEUF-CMA}^{\A}_{\DS}$ is defined in Figure \ref{fig:seufcma}. \label{def:seufcma}
\end{definition}


\begin{figure}[ht]
\fbox{%
  \begin{minipage}{0.95\textwidth}
    % Top minipage
    
    \begin{minipage}[t]{0.05\textwidth}
    \end{minipage}\hfill
    \begin{minipage}[t]{0.5\textwidth}

      \textbf{\uline{\bm{$\textsf{GsEUF-CMA}^{\A}_{\DS}(1^\lambda)$}}}\\[0.5ex]
$sp \gets \DS.\Setup(1^\lambda); Q\gets \emptyset$\\
$(sk,vk)\gets \DS.\Keygen(sp)\\ (m^*,\sigma^*)\gets \A^{\textsf{OSig}}(vk)$\\
$\textbf{return } (m^*,\sigma^*)\neq Q \land \DS.\Ver(vk,m^*,\sigma^*)$
    \end{minipage}
    \hfill
    \begin{minipage}[t]{0.35\textwidth}
      \textbf{\uline{\bm{$\textsf{OSig}(m)$}}}\\[0.5ex]
$\sigma \gets \DS.\Sign(skk,m)$\\
$Q \gets Q\cup \{(m,s)\}$\\
$\textbf{return } \sigma$
    \end{minipage}
    \hfill
    \begin{minipage}[t]{0.1\textwidth}
    \end{minipage}

  \end{minipage}%
}

\caption{The security experiment $\textsf{sEUF-CMA}^{\A}_{\DS}$ and supporting oracle $\textsf{OSig}$.}\label{fig:seufcma}

\end{figure}

\subsection{Parametrized Non-Interactive Zero Knowledge Arguments}

Following \cite{blindsigs}, we define non-interactive zero knowledge arguments with regards to \emph parametrized polynomial relations $\mathcal{P}: \{0,1\}^* \times \{0,1\}^* \times \{0,1\}^*  \to \{0,1\}$, where the first argument represents a parameter set $par$ (e.g. a group description). Given a value of $par$, we say that $w$ is a witness for statement $\theta$ if $\mathcal{P}(par,\theta,w)=1$, i.e. $R=R_{par}(\theta,w):=\mathcal{P}(par,\theta,w)$ is an NP-relation, and $\Lang=\Lang_{par}$ is an NP-language. A NIZK for a relation  $\mathcal{P}$ operates as follows:
\begin{itemize}

\item $\mathsf{Rel}(1^\lambda) \to par$. Generates a parameter set $par$ that defines $R$ and $\Lang$.
\item $\Setup(par)\to (\crs,\tau)$. Generates a common reference string (CRS) and trapdoor $\tau$ used by the simulator. We assume the CRS contains a description of $R$.
\item $\Prove(\crs,\theta,w)\to \pi$. Given a CRS $\crs$, statement $\theta$ and witness $w$ for $\theta$, produces a proof $\pi$.  
\item $\Ver(\crs,\theta,\pi)\to \{0,1\}$. Given a CRS $\crs$, a statement $\theta$ and a  proof $\pi$ outputs 1 or 0, accepting or rejecting the proof.
\item $\SimProve(\crs,\theta,\tau)$. Given a CRS $\crs$, statement $\theta$ and traproor $\tau$ for $\crs$, produces a simulated proof $\pi$.
\end{itemize}


\begin{definition} A system $\NArgr$ is perfectly correct if for all unbounded adversaries $\A$ 
\begin{align*}
\Pr [par \gets \NArg.\mathsf{Rel}(1^\lambda);  (\crs,\tau)\gets \NArg.\Setup(par); (\theta,w) \gets A(crs):  \\
\neg{}R(\theta,w) \lor \NArg.\Ver(\crs,\theta,\NArg.\Prove(\crs,\theta,w))]=1 \\
\end{align*}
\end{definition}

\begin{definition} A system $\NArgr$ is adaptably computationally sound if for all p.p.t. adversaries $\A$ 
\begin{align*}
\Pr [par \gets \NArg.\mathsf{Rel}(1^\lambda);  (\crs,\tau)\gets \NArg.\Setup(par); (\theta,\pi) \gets A(crs):  \\
\neg{}L(\theta) \land {}\NArg.\Ver(\crs,\theta,\pi)] \mbox{ is negligible in $\lambda$.}
\end{align*}
\end{definition}

\begin{definition} A system $\NArgr$ is  computationally zero-knowledge if for all p.p.t. adversaries $\A$ 
\begin{align*}
|\Pr [par \gets \NArg.\mathsf{Rel}(1^\lambda);  (\crs,\tau)\gets \NArg.\Setup(par); d\gets\{0,1\}:  \\
d = A^{\mathsf{OProve}_d}(crs)] - {1\over2}|\mbox{ is negligible in $\lambda$, where}\\
\mathsf{OProve}_0(\theta,w):= \textbf{if } \neg R(\theta,w) \textbf{ return }\bot; \textbf{ return } \NArg.\Prove(\theta,w), \mbox{and} \\
\mathsf{OProve}_1(\theta,w):= \textbf{if } \neg R(\theta,w) \textbf{ return }\bot; \textbf{ return } \NArg.\SimProve(\theta,\tau).
  \end{align*}
\end{definition}

\newpage

\section{Additional Ledger Syntax}
\label{sec:appendix}

In Figure~\ref{fig:notation:nonstandard} we introduce the standard ledger syntax that we use throughout.

\begin{figure}[h!tb]
  \begin{align*}
    \H{}
    & =~\bigcup_{n=0}^{\infty}\{0,1\}^{8n}
    & \mbox{the type of bytestrings }
    \\
    (a, b)
    & :~\Interval{A}
    & \text{intervals over a totally-ordered set $A$}
    \\
    \var{Key} \mapsto \var{Value}
    & \subseteq \{~ k \mapsto v ~\mid~ k \in \var{Key},~v \in \var{Value}~ \}
    & \text{finite map with unique keys}
    \\
    [a1 ; ...; ak]
    & :~[C]
    & \text{finite list with terms of type $C$}
    \\
    h :: t
    & :~[C]
    & \text{list with head $h$ and tail $t$}
    \\
    x \cup \fun{nothing}
    & :~A^?
    & \text{maybe type over $A$}
    \\
    a ~\{ ~\fun{field} = ~x~\}
    & :~A
    & \text{record of type A with $\fun{field}$ changed to $x$}
  \end{align*}
  \caption{Notation}
  \label{fig:notation:nonstandard}
\end{figure}

Figure~\ref{fig:eutxo-types} lists the primitives and derived types
that comprise the foundations of the EUTxO model,
along with some ancillary definitions.
(Outputs normally refer to transaction IDs by hash,
but we simplify here for clarity.)

\begin{figure}[h!tb]

\textsc{Ledger primitives}
\begin{displaymath}
\begin{array}{rlll}
  \checkSig &:&
  \Tx \to \pubkey \to \H \to \B \\
  & &\mbox{\emph{checks that a given key signed a transaction}}
\end{array}
\end{displaymath}

\textsc{Helper functions}
    \begin{align*}
        &\fun{txid} : \Tx \to \TxId \\
        &\fun{txid}~\var{tx} = \fun{hash}~(\var{tx} ~\{ ~\fun{sigKeys} = ~\fun{dom}~(\var{tx} . \sigs) ,~\sigs = ~\emptyset~\})
        \nextdef
        &\fun{toMap} : \N \to [\TxOut] \to (\N \mapsto \TxOut) \\
        &\fun{toMap}(\_,~[~]) ~~~~~~~~~~~~~= [~] \\
        &\fun{toMap}(\var{ix},~u~::~\var{outs}) = \{~\var{ix}\mapsto u~\} \cup \fun{toMap}(\var{ix}+1,~\var{outs}) \\
        & \mbox{\emph{constructs a map from a list of outputs}}
        \nextdef
        &\fun{mkOuts} : \Tx \to \UTxO \\
        &\fun{mkOuts}(tx) = \{~(\var{tx},~\var{ix}) \mapsto o~ \mid~(\var{ix} \mapsto o)\in~\fun{toMap}(0,~\var{tx}.\outputs)~\} \\
        &\mbox{\emph{constructs a UTxO set from a list of outputs of a given transaction}}
        \nextdef
        &\fun{MOf} : \N \to \N \to (A \to \B) \to [A] \to \B \\
        &\fun{MOf}~k~m~f~[~] = m~\leq~k \\
        &\fun{MOf}~k~m~f~( h~ ::~ t) = \fun{if}~ (m~\leq~k)~\fun{then}~\true~\fun{else}~(\fun{MOf}~(k~+~a)~m~f~t) \\
        &~~\where~~a~=~\fun{if}~ (f~(h))~\fun{then}~1~\fun{else}~0 \\
        &\mbox{\emph{returns $\true$ if enough elements of a list satisfy given function}}
    \end{align*}
\caption{Primitives and basic types for the $\UTxO_{ma}$ model}
\label{fig:eutxo-types}
\end{figure}
\newpage 
\subsection{Transaction Validation Rules}

A transaction $\var{tx}$ is \emph{valid} if it follows the following rules. 

\label{sec:check-tx}


\begin{itemize}
    \item[(i)] \textbf{The transaction has at least one input:}
  \[
  \var{tx}.\inputs~\neq~\{\}
  \]
    \item[(ii)] \textbf{The current slot is within transaction validity interval:}
  \[
  \var{slot} \in \var{tx}.\fun{validityInterval}
  \]
    \item[(iii)] \textbf{All outputs have positive values:}
  \[
  \forall o \in \var{tx}.\outputs,~\var{o}.\val > \emptyset
  \]
    \item[(iv)] \textbf{All output references of transaction inputs exist in the UTxO:}
  \[
  \var{tx}.\inputs~ \subseteq~ \fun{dom}~\var{utxo}
  \]
    \item[(v)] \textbf{Value is preserved:}
  \[
  \var{tx}.\mint +
  \sum_{i \in~\var{tx}.\inputs,~(i~\mapsto~o) \in~\var{utxo}} \var{o}.\val =
  \sum_{o \in~\var{tx}.\outputs} \var{o}.\val ~+~\fun{toValue}~(\var{tx}.\fun{fee})
  \]
    \item[(vii)] \textbf{All inputs validate:}
  \[
  \forall~ i \in \var{tx}.\inputs,~i \mapsto (s, v) \in \var{utxo},~
     \applyScript{s}(\fun{dom}~(\var{tx}.\sigs),~\var{tx}.\validityInterval) = \true
  \]
    \item[(ix)] \textbf{All minting scripts validate:}
  \[
  \forall~ p\mapsto \_ \in \var{tx}.\mint,~ \applyScript{p} (\fun{dom}~(\var{tx}.\sigs),~\var{tx}.\validityInterval) = \true
  \]
    \item[(x)] \textbf{All signatures are correct:}
  \[
  \forall~ (pk \mapsto s) \in \var{tx}.\sigs,~ \checkSig (\var{tx}, pk, s) = \true
  \]
  \item[(i)] \textbf{The fee is sufficient:}
  \[
    \var{s}.\fun{minfee} \leq \var{tx}.\fun{fee} 
  \]
  \end{itemize}

\subsection{Script construction and Evaluation}
\label{sec:check-script}

In Figure \ref{fig:script} we present the constructors and evaluation rules for scripts, and in Figure \ref{fig:mktospec} we explain the transaction building function \fun{mkToSpec}.

\begin{figure}
    \textsc{Constructors of $\Script$} 
\begin{displaymath}
\begin{array}{rlll}
    \fun{RequireMOf}         &: \N \to [\Script] &\to~ \Script \\
    \fun{RequireSig}         &: \pubkey      &\to~ \Script \\
    \fun{RequireTimeStart}   &: \Slot        &\to~ \Script \\
    \fun{RequireTimeExpire}  &: \Slot        &\to~ \Script \\
\end{array}
\end{displaymath}
\nextdef
\textsc{Evaluation of $\Script$} 
\begin{displaymath}
\begin{array}{rlll}
  \applyScript{\_} &:& \Script \to ((\type{Set} \pubkey) \times (Slot \times Slot)) \to \B \\
  \applyScript{\fun{RequireMOf}~n~ls} (\var{khs}, (t1, t2)) &=&  \fun{MOf}~0~ n ~(\applyScript{\_}~(\var{khs}, (t1, t2))) ~ls \\
  \applyScript{\fun{RequireSig}~k} (\var{khs}, (t1, t2)) &=& k~\in~\var{khs} \\
  \applyScript{\fun{RequireTimeStart}~t1'} (\var{khs}, (t1, t2)) &=& t1'~\leq~t1 \\
  \applyScript{\fun{RequireTimeExpire}~t2'} (\var{khs}, (t1, t2)) &=& t2~\leq~t2' 
\end{array}
\end{displaymath}
\caption{$\Script$ constructors and evaluation}
\label{fig:script}
\end{figure}

\begin{figure}
      \textsc{Trivial transaction $\fun{initTx}_{a,mf}$} 
\begin{displaymath}
\begin{array}{rlll}
  \fun{initTx}_{a,mf}  &=& \{ ~\inputs = \emptyset,\\
  & &\ \outputs = [ (a , \fun{tip})],\\
  & &\ \fun{validityInterval} = [\fun{nothing} , \fun{nothing}],\\
  & &\ \mint = 0,\\
  & &\ \fun{fee} = mf \\
  & &\ \fun{aux} = []~\\
  & &\ \sigs = \emptyset~ \}
\end{array}
\end{displaymath}
\nextdef
\textsc{Input selection function} 
\begin{displaymath}
\begin{array}{rlll}
  \fun{mkIns}  &:& (\LState \times (\type{Set} \TxIn) \times \Value \times \Script) \to (\type{Set} \TxIn) \\
  \fun{mkIns} ~(l,~i,~v,~s) &=& \fun{if}~\neg~(v \leq 0) ~ \fun{then} \\
  & & ~~~~ \fun{if}~(v > 0)~\fun{then}~\\
  & & ~~~~ ~~~~\fun{mkIns}~(l~\setminus~(\{j \mapsto \_\}),~i \cup j,~v~-~(u(j) . \val) ,~s) \\
  & & ~~~~ \fun{else}~i\\
  & & \fun{else } \\
  & & ~~~~\fun{nothing} \\
  & & \fun{where}~\\
  & & ~~~j = \fun{pickInput}~l~s \\
  & & ~~~u = \fun{utxo}~l
\end{array}
\end{displaymath}
\nextdef
\textsc{Auxiliary $\fun{mkToSpec'}$ definition} 
\begin{displaymath}
\begin{array}{rlll}
\fun{mkToSpec'} &:& \LState \to \mathcal{I}_\mathsf{post} \to \Tx \to \Tx^? \\
\fun{mkToSpec'}~l~(\fun{MustMint}~v) (\var{tx}) &=&  \var{tx}~\{ \mint~=~v ~+~ \var{tx} . \mint , \fun{sigs} = tx.\fun{sigs} \cup \fun{getSigsVal}~v ,\\
& & ~~~~\fun{validityInterval} = \fun{restrictIntervalVal}~tx.\fun{validityInterval}~v\} \\
\fun{mkToSpec'}~l~(\fun{SpendFrom}~s) (\var{tx}) &=& \var{tx}~\{ \outputs~=~ \var{tx} . \inputs~\\
& & ~~~~\cup~\fun{newIns} , \fun{sigs} = tx.\fun{sigs} \cup \fun{getSigsUTxO}~\fun{newIns}~l ,\\ 
& & ~~~~ \fun{validityInterval} = \\ 
& &~~~~~~~~\fun{restrictIntervalUTxO}~tx.\fun{validityInterval}~\fun{newIns}~l\} \\
& & ~~~~\where \\
& & ~~~~\fun{newIns} = \fun{mkIns}~l~(\var{tx} . \inputs)~(\fun{produced} - \fun{consumed})~ s \\
\fun{mkToSpec'}~l~ (\fun{MaxInterval}~i) (\var{tx}) &=& \var{tx}~\{ \validityInterval~=~(l . \fun{slot}, \\
& & ~~~~\fun{min}~\{l . \fun{slot}~+~i , \var{tx} . \fun{validityInterval}_2)\} \}  \\
\fun{mkToSpec'}~l~ (\fun{PayTo}~(s, v)) (\var{tx}) &=& \var{tx}~\{ \outputs~=~ \var{tx} . \outputs~\cup~ (s, v) \}   \\
\fun{mkToSpec'}~l~ (\fun{ChangeTo}~s) (\var{tx}) &=& \fun{if}~ \fun{consumed} - \fun{produced} ~>~0 ~\\ 
& & ~~~~\fun{then}~ \var{tx}~\{ \outputs~=~ \var{tx} . \outputs~\\
& & ~~~~\cup~ \{(s,~ \fun{consumed} - \fun{produced})\} \} ~\fun{else}~\fun{nothing}   \\
\fun{mkToSpec'}~l~ (\fun{MaxFee}~f)~ (\var{tx}) &=&  \fun{if}~ \var{tx} . \fun{fee} \leq f~\\ 
& & ~~~~\fun{then}~\var{tx}~\fun{else}~\fun{nothing} \\
\fun{mkToSpec'}~l~ (\fun{AndExps}~[a1 ; a2 ; ... ; ak]) (\var{tx}) &=& \fun{mkToSpec'}~l~ak~(... ~(\fun{mkToSpec'}~l~a2~(\fun{mkToSpec'}~l~a1~\var{tx})))
\end{array}
\end{displaymath}
\nextdef 
\textsc{$\fun{mkToSpec}$ definition} 
\begin{displaymath}
\begin{array}{rlll}
\fun{mkToSpec} &:& (\LState \times \mathcal{I}_\mathsf{post}) \to \Tx^? \\
\fun{mkToSpec}~(l , i) &=& \fun{mkToSpec'}~l~i~\fun{initTx}_{a,mf}
\end{array}
\end{displaymath}
\caption{Building transactions according to the specific intent}
\label{fig:mktospec}
\end{figure}




\begin{figure}
\begin{displaymath}
\begin{array}{rlll}
  \fun{txi}_1 &=&(\outputs = \{~(s, t)~\},\\
  & &\ \fun{validityInterval} = [ \fun{slot}~l , (\fun{slot}~l) + j ],\\
  & &\ \mint = t,\\
  & &\ \fun{fee} = \fun{minfee}~l \\
  & &\ \fun{sigKeys} = \fun{getSignersVal}~t) 
  \nextdef 
  \fun{txi}_2 &=& (\outputs = \{ (\fun{RequireSig}~k2,~x)~,~(\fun{RequireSig}~k1~,\\& & ~(\fun{balance}~(\fun{mkIns}~l~\{\}~x~ (\fun{RequireSig}~k1))) - x) \} ,\\
  & &\ \fun{validityInterval}  = [~ \fun{nothing}~ , \fun{nothing}~],\\
  & &\ \mint  = \{~\} ,\\
  & &\ \fun{fee}  = \fun{minfee}~l \\
  & &\ \fun{sigKeys}  = \{~k1~\} )
\end{array}
\end{displaymath}
\caption{Abstract transaction examples}
\label{fig:txs}
\end{figure}

%

\end{document}
