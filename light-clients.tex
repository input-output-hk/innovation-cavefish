\documentclass[runningheads]{llncs}
%
\usepackage[T1]{fontenc}
% T1 fonts will be used to generate the final print and online PDFs,
% so please use T1 fonts in your manuscript whenever possible.
% Other font encondings may result in incorrect characters.
%
\usepackage{graphicx}
% Used for displaying a sample figure. If possible, figure files should
% be included in EPS format.
%
% If you use the hyperref package, please uncomment the following two lines
% to display URLs in blue roman font according to Springer's eBook style:
%\usepackage{color}
%\renewcommand\UrlFont{\color{blue}\rmfamily}
%\urlstyle{rm}
%
\begin{document}
%
\title{Light Clients for Building UTxO Ledger Transactions}
%
%\titlerunning{Abbreviated paper title} 
% If the paper title is too long for the running head, you can set
% an abbreviated paper title here
%
\author{Pyrros Chaidos\inst{1}\orcidID{0000-1111-2222-3333} \and
Aggelos Kiayias\inst{1}\orcidID{1111-2222-3333-4444} \and
Marc Roeschlin\inst{1}\orcidID{0000-1111-2222-3333} \and
Polina Vinogradova\inst{1}\orcidID{0000-0003-3271-3841}}
%
\authorrunning{P. Chaidos et al.}
% First names are abbreviated in the running head.
% If there are more than two authors, 'et al.' is used.
%
\institute{Input Output, Global
\email{firstname.lastname@iohk.io}
\url{iohk.io}}
%
\maketitle              % typeset the header of the contribution
%
\begin{abstract}
The abstract should briefly summarize the contents of the paper in
150--250 words.

\keywords{First keyword  \and Second keyword \and Another keyword.}
\end{abstract}
%
%
%
\section{Introduction}

What sets us apart?

\begin{itemize}
    \item Atomicity of payment+service

    \item Model for (trustless) 2-party transaction construction rather than proving things about chain/ledger state

    \item Do not require establishing a relationship with SP or any other set-up
\end{itemize}

We use blind signatures \cite{blindsigs}

\section{Threat Model}

\begin{itemize}
    \item Client finds out too much from SP answer and can submit tx without payment to SP (that’s why the inputs must be hidden)

    \item SP lies to client about the UTxOs being spent by the tx, and tricks them into doing something they didn’t want to do (that’s why 0-knowledge proof that outputs were correctly specified)

\end{itemize}

\section{Light Client Specification}
 
What are the capabilities of a light client?

Does it remember all addresses it has been paid at (tx history)?

Is light client allowed to maintain state, and what state can they maintain if so?
Secret key is the minimum state.

How can we formalize the intent of a light client without revealing secret key?

Can we have viewing keys?

\section{Intent Specification}
Intent (DSL or predicate to describe intent of the client, ie what they want to do to the ledger state)

\section{Protocol}

\section{Analysis}

\section{Related Work}

Compare our approach  with :
\begin{itemize}
    \item "Free" websites monitoring the chain
    \item Bridges (trustless and trusted)
    \item Payment channels 
    \item LCs that operate on single-prover model (eg. with an established relationship via deposit)
    \item LCs that operate on multi-prover model
\end{itemize}

\section{Conclusion}



\begin{credits}
\subsubsection{\ackname} A bold run-in heading in small font size at the end of the paper is
used for general acknowledgments, for example: This study was funded
by X (grant number Y).

\subsubsection{\discintname}
It is now necessary to declare any competing interests or to specifically
state that the authors have no competing interests. Please place the
statement with a bold run-in heading in small font size beneath the
(optional) acknowledgments\footnote{If EquinOCS, our proceedings submission
system, is used, then the disclaimer can be provided directly in the system.},
for example: The authors have no competing interests to declare that are
relevant to the content of this article. Or: Author A has received research
grants from Company W. Author B has received a speaker honorarium from
Company X and owns stock in Company Y. Author C is a member of committee Z.
\end{credits}
%
% ---- Bibliography ----
%
% BibTeX users should specify bibliography style 'splncs04'.
% References will then be sorted and formatted in the correct style.
%
\bibliographystyle{splncs04}
\bibliography{lcbib}
%

\end{document}
