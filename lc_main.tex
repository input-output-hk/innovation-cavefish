
\documentclass[a4paper,USenglish,cleveref, autoref,anonymous, thm-restate]{lipics-v2021}
%This is a template for producing LIPIcs articles. 
%See lipics-v2021-authors-guidelines.pdf for further information.
%for A4 paper format use option "a4paper", for US-letter use option "letterpaper"
%for british hyphenation rules use option "UKenglish", for american hyphenation rules use option "USenglish"
%for section-numbered lemmas etc., use "numberwithinsect"
%for enabling cleveref support, use "cleveref"
%for enabling autoref support, use "autoref"
%for anonymousing the authors (e.g. for double-blind review), add "anonymous"
%for enabling thm-restate support, use "thm-restate"
%for enabling a two-column layout for the author/affilation part (only applicable for > 6 authors), use "authorcolumns"
%for producing a PDF according the PDF/A standard, add "pdfa"

%\pdfoutput=1 %uncomment to ensure pdflatex processing (mandatatory e.g. to submit to arXiv)
%\hideLIPIcs  %uncomment to remove references to LIPIcs series (logo, DOI, ...), e.g. when preparing a pre-final version to be uploaded to arXiv or another public repository


\usepackage{stmaryrd}
\usepackage{bm}

%\usepackage{amsmath}
\usepackage{tikz}
\usetikzlibrary{matrix,shapes,arrows,positioning,chains,calc,fit}
\usepackage{xcolor}
\usepackage{booktabs}
\usepackage{multirow}
\usepackage[normalem]{ulem}
\usepackage{amsthm}

\theoremstyle{definition} % For upright text
\newtheorem{assumption}{Assumption}


%\graphicspath{{./graphics/}}%helpful if your graphic files are in another directory

\bibliographystyle{plainurl}% the mandatory bibstyle

\title{Communication-Optimal Light Client Protocol for UTxO Ledgers in the Rational Model}

%\titlerunning{Dummy short title} %TODO optional, please use if title is longer than one line

\author{Jane {Open Access}}{Dummy University Computing Laboratory, [optional: Address], Country \and My second affiliation, Country \and \url{http://www.myhomepage.edu} }{johnqpublic@dummyuni.org}{https://orcid.org/0000-0002-1825-0097}{(Optional) author-specific funding acknowledgements}%TODO mandatory, please use full name; only 1 author per \author macro; first two parameters are mandatory, other parameters can be empty. Please provide at least the name of the affiliation and the country. The full address is optional. Use additional curly braces to indicate the correct name splitting when the last name consists of multiple name parts.


\authorrunning{J. Open Access and J.\,R. Public} %TODO mandatory. First: Use abbreviated first/middle names. Second (only in severe cases): Use first author plus 'et al.'

\Copyright{Jane Open Access and Joan R. Public} %TODO mandatory, please use full first names. LIPIcs license is "CC-BY";  http://creativecommons.org/licenses/by/3.0/

\ccsdesc[100]{\textcolor{red}{Replace ccsdesc macro with valid one}} %TODO mandatory: Please choose ACM 2012 classifications from https://dl.acm.org/ccs/ccs_flat.cfm 

\keywords{Dummy keyword} %TODO mandatory; please add comma-separated list of keywords

\category{} %optional, e.g. invited paper

\relatedversion{} %optional, e.g. full version hosted on arXiv, HAL, or other respository/website
%\relatedversiondetails[linktext={opt. text shown instead of the URL}, cite=DBLP:books/mk/GrayR93]{Classification (e.g. Full Version, Extended Version, Previous Version}{URL to related version} %linktext and cite are optional

%\supplement{}%optional, e.g. related research data, source code, ... hosted on a repository like zenodo, figshare, GitHub, ...
%\supplementdetails[linktext={opt. text shown instead of the URL}, cite=DBLP:books/mk/GrayR93, subcategory={Description, Subcategory}, swhid={Software Heritage Identifier}]{General Classification (e.g. Software, Dataset, Model, ...)}{URL to related version} %linktext, cite, and subcategory are optional

%\funding{(Optional) general funding statement \dots}%optional, to capture a funding statement, which applies to all authors. Please enter author specific funding statements as fifth argument of the \author macro.


%\nolinenumbers %uncomment to disable line numbering



%Editor-only macros:: begin (do not touch as author)%%%%%%%%%%%%%%%%%%%%%%%%%%%%%%%%%%
\EventEditors{John Q. Open and Joan R. Access}
\EventNoEds{2}
\EventLongTitle{42nd Conference on Very Important Topics (CVIT 2016)}
\EventShortTitle{CVIT 2016}
\EventAcronym{CVIT}
\EventYear{2016}
\EventDate{December 24--27, 2016}
\EventLocation{Little Whinging, United Kingdom}
\EventLogo{}
\SeriesVolume{42}
\ArticleNo{23}
%%%%%%%%%%%%%%%%%%%%%%%%%%%%%%%%%%%%%%%%%%%%%%%%%%%%%%


%formatting
\newcommand{\var}[1]{\ensuremath{\mathit{#1}}}
\newcommand{\fun}[1]{\ensuremath{\mathsf{#1}}}
\newcommand{\type}[1]{\ensuremath{\mathsf{#1}}}
\renewcommand\H{\mathbb{H}}
\newcommand\A{\ensuremath{\mathcal{A}}}

\newcommand\BLD{\ensuremath{\mathsf{BLD}}}
\newcommand\WBPS{\ensuremath{\mathsf{WBPS}}}
\newcommand\WBPSP{\ensuremath{\mathsf{WBPS}[P]}}


\newcommand\hash{\ensuremath{\mathsf{H}}}
\newcommand\HGen{\ensuremath{\mathsf{HGen}}}

\newcommand\Setup{\ensuremath{\mathsf{Setup}}}
\newcommand\Keygen{\ensuremath{\mathsf{KeyGen}}}
\newcommand\Ver{\ensuremath{\mathsf{Ver}}}
\newcommand\Sign{\ensuremath{\mathsf{Sign}}}
\newcommand\Request{\ensuremath{\mathsf{Request}}}

\newcommand\GGen{\ensuremath{\mathsf{GGen}}}
\newcommand\NArg{\ensuremath{\mathsf{NArg}}}
\newcommand\Prove{\ensuremath{\mathsf{Prove}}}
\newcommand\PKE{\ensuremath{\mathsf{PKE}}}
\newcommand\SDS{\ensuremath{\mathsf{SDS}}}
\newcommand\DS{\ensuremath{\mathsf{DS}}}
\newcommand\Enc{\ensuremath{\mathsf{Enc}}}
\newcommand\Dec{\ensuremath{\mathsf{Dec}}}


\newcommand{\Interval}[1]{\type{Interval}[#1]}
\newcommand{\nextdef}{\ensuremath{\\[1em]}}
\newcommand{\where}{\ensuremath{~ ~ \mathbf{where}~ ~ }}

\newcommand{\todobox}[1]{ \fcolorbox{black}{gray!50} {\textbf{TODO:} \parbox{.7\textwidth}{#1}}}


% Macros for eutxo things.
\newcommand{\tx}{\fun{tx}}
\newcommand{\TxId}{\type{TxId}}
\newcommand{\TxIn}{\type{TxIn}}
\newcommand{\Header}{\type{Header}}
\newcommand{\Block}{\type{Block}}
\newcommand{\LState}{\type{LState}}
\newcommand{\Tx}{\type{Tx}}
\newcommand{\Slot}{\type{Slot}}
\newcommand{\TxAbs}{\type{TxAbs}}
\newcommand{\B}{\mathbb{B}}
\newcommand{\TxOut}{\type{TxOut}}
\newcommand{\UTxO}{\type{UTxO}}
\newcommand{\txId}{\msf{txId}}
\newcommand{\txrefid}{\fun{id}}
\newcommand{\Address}{\ensuremath{\s{Address}}}
\newcommand{\DataHash}{\ensuremath{\s{DataHash}}}
\newcommand{\hashData}{\fun{dataHash}}
\newcommand{\idx}{\fun{index}}
\newcommand{\inputs}{\fun{inputs}}
\newcommand{\outputs}{\fun{outputs}}
\newcommand{\Out}{\type{Output}}
\newcommand{\validityInterval}{\fun{validityInterval}}
\newcommand{\scripts}{\fun{scripts}}
\newcommand{\mint}{\fun{mint}}
\newcommand{\mintScripts}{\fun{mintScripts}}
\newcommand{\mintScsRdmrs}{\fun{mintRdmrs}}
\newcommand{\mintRdmrs}{\fun{mintRdmrs}}
\newcommand{\sigs}{\fun{sigs}}
\newcommand{\fee}{\fun{fee}}
\newcommand{\addr}{\fun{addr}}
\newcommand{\pubkey}{\fun{PubKey}}
\newcommand{\privkey}{\fun{PrivKey}}
\newcommand{\val}{\fun{value}}  %% \value is already defined
\newcommand{\Value}{\type{Value}}
\newcommand{\Redeemer}{\type{Redeemer}}
\newcommand{\TxOutRef}{\type{TxIn}}
\newcommand{\ScriptContext}{\type{ScriptContext}}
\newcommand{\ScriptPurpose}{\type{ScriptPurpose}}
\newcommand{\Datum}{\type{Datum}}
\newcommand{\DCert}{\type{DCert}}
\newcommand{\LCTx}{\type{LCTx}}
\newcommand{\TxInInfo}{\type{TxInInfo}}


\newcommand{\validator}{\fun{validator}}
\newcommand{\redeemer}{\fun{redeemer}}
\newcommand{\datum}{\fun{datum}}
\newcommand{\datumHash}{\fun{datumHash}}
\newcommand{\datumWits}{\fun{datumWitnesses}}
\newcommand{\Data}{\ensuremath{\s{Data}}}
\newcommand{\Input}{\ensuremath{\s{Input}}}
\newcommand{\Output}{\type{Output}}
\newcommand{\OutputRef}{\fun{OutputRef}}
\newcommand{\Signature}{\type{Signature}}
\newcommand{\Ledger}{\ensuremath{\s{Ledger}}}

\newcommand{\outputref}{\fun{outputRef}}
\newcommand{\outputrefs}{\fun{outputRefs}}
\newcommand{\txin}{\fun{in}}
\newcommand{\id}{\fun{id}}
\newcommand{\lookupTx}{\msf{lookupTx}}
\newcommand{\getSpent}{\msf{getSpentOutput}}

% \newcommand{\Tick}{\ensuremath{\s{Tick}}}
\newcommand{\Tick}{\type{Slot}}
\newcommand{\Script}{\type{Script}}
\newcommand{\spent}{\msf{spentOutputs}}
\newcommand{\unspent}{\msf{unspentOutputs}}
\newcommand{\txunspent}{\msf{unspentTxOutputs}}
\newcommand{\eutxotx}{\msf{Tx}}

\newcommand{\consumes}[1]{\msf{consumes(#1)}}
\newcommand{\consumesOne}[1]{\msf{consumesOne(#1)}}
\newcommand{\cid}{\fun{cid}}
\newcommand{\inputValue}{\fun{inputValue}}
\newcommand{\rMin}{r_{\fun{min}}}
\newcommand{\rMax}{r_{\fun{max}}}

\newcommand{\utxotx}{\msf{Tx}}

\newcommand{\N}{\mathbb{N}}
\newcommand{\Z}{\mathbb{Z}}
\newcommand{\G}{\mathbb{G}}


\newcommand{\Quantity}{\type{Quantity}}
\newcommand{\TokenName}{\type{TokenName}}
\newcommand{\Asset}{\ensuremath{\s{Asset}}}
\newcommand{\AssetID}{\type{AssetID}}
\newcommand{\Policy}{\type{Policy}}
\newcommand{\Quantities}{\ensuremath{\s{Quantities}}}
\newcommand{\nativeCur}{\ensuremath{\mathrm{nativeC}}}
\newcommand{\nativeTok}{\ensuremath{\mathrm{nativeT}}}
\newcommand{\valC}{\mkValidator{\mathcal{C}}}
\newcommand{\polC}{\mkPolicy{\mathcal{C}}}
\newcommand\mkValidator[1]{\msf{validator}_#1}
\newcommand\mkPolicy[1]{\msf{policy}_#1}

\newcommand{\PublicKey}{\ensuremath{\s{PubKey}}}
\newcommand{\PubKey}{\ensuremath{\s{PubKey}}}
\newcommand{\PrivateKey}{\ensuremath{\s{PrivateKey}}}

\newcommand{\mkContext}{\ensuremath{\s{mkContext}}}
\newcommand{\mkTxInfo}{\ensuremath{\s{mkTxInfo}}}
\newcommand{\mkVlContext}{\ensuremath{\s{mkValidatorContext}}}
\newcommand{\mkMpsContext}{\ensuremath{\s{mkPolicyContext}}}
\newcommand{\checkSig}{\type{checkSig}}

\newcommand{\applyScript}[1]{\ensuremath{\llbracket#1\rrbracket}}
\newcommand{\chkSpec}[1]{\ensuremath{\llbracket#1\rrbracket}_{\type{DSL}}}
\newcommand{\applyMPScript}[1]{\ensuremath{\llbracket#1\rrbracket}}

\newcommand{\true}{\type{True}}
\newcommand{\false}{\type{False}}
\newcommand{\True}{\type{True}}
\newcommand{\False}{\type{False}}

\newcommand{\leteq}{:=}

\begin{document}

\maketitle

\begin{abstract}
Blockchain light clients (LCs) are users with limited resources who cannot maintain a fully validated local copy of the ledger state. Consequently, they rely on service providers (SPs), typically full nodes, to access the data needed for tasks such as transaction construction or interacting with off-chain applications.
In this work, we propose a protocol and model for UTxO-based platforms that enables LCs to submit transactions with minimal local storage and computation. Our solution is a two-party computation (2PC) protocol between an LC and an SP. The LC instructs the SP to create a transaction according to its specifications, but only a blinded version of the result is shared. The blinded form prevents the LC from altering the transaction or constructing a new (valid) transaction, while still allowing the LC to verify that the transaction satisfies their original specification.
To achieve this, we introduce a secure predicate mechanism and a weakly blind signature scheme. This enables the SP to obtain a valid signature on the original (unblinded) transaction from the LC, which can then be submitted to the network. The result is a trustless interaction where the LC achieves their transaction goal, and the SP receives compensation for their work.
To optimize communication and computational overhead, we design an extension of Secure Blind Schnorr Signatures called Weakly Blind Predicate Signatures, which relaxes the unlinkability requirement in standard blind signatures.
Lastly, we implement and benchmark the Non-interactive Argument of Knowledge (NArg) component of our protocol on two major UTxO-based blockchain platforms. Despite this component being the most computationally intensive part, our evaluation demonstrates that proving and verification times, as well as associated circuit sizes, are well within practical bounds for real-world deployment.



%Blockchain light clients (LCs) are platform users that do not have the capacity to locally maintain current,
%fully-validated ledger state. For this reason, light clients rely on service providers (SPs, which are full nodes) for obtaining the blockchain data they require, e.g. for transaction construction or running off-chain apps.
%In this paper, we propose protocol and model in which a light client of a UTxO-style platform has the goal of submitting a transaction with minimal local state and minimal local processing. We propose a 2-party computation protocol between the LC and the SP. First, the light client creates a specification for the transaction they would like the SP to construct, then the SP builds such a transaction, while including a small payment to cover the cost of computation.
%The SP sends the LC a blinded version of the constructed transaction for signing and checking. It is modified such that it is impossible for the LC to use the provided data to construct a distinct new (valid) transaction without guessing pre-images of transaction hashes, but has enough information to allow the LC to
%check that it meets their specification. Using a secure predicate and partially blind signature scheme, the SP is able to obtain a signature on the original (unmodified) transaction,
%which is submitted to the network for inclusion in a block. This 2-PC protocol constitutes a trustless
%interaction between the LC and the SP resulting in the LC's desired transaction being applied
%to the ledger state, and the SP receiving compensation for their work.

%In order to realize our light client protocol in a communication optimal way, we develop an adaptation of Secure Blind Schnorr Signatures that we call Weakly Blind Predicate Signatures, alleviating the unlinkability requirement of standard Blind Schnorr Signatures.

%Finally, we implement and benchmark the Non-interactive Argument of Knowledge as part of our protocol and constitutes the most time and resource intensive component of our construction.
\end{abstract}



\section{Introduction}
\todobox{
What sets us apart?

\begin{itemize}
    \item Atomicity of payment+service

    \item permission-less, decentralized

    \item Model for (trustless) 2-party transaction construction rather than proving things about chain/ledger state

    \item Do not require establishing a relationship with SP or any other set-up

    \item inherent timeliness of transaction construction incentivized by SPs desired to earn
    their tip. This is in contrast with the possibility of stale info provided from old Mithril snapshots in other LC models
\end{itemize}}

The core of our design is an answer to the question of "how can a user request and approve payment (e.g. from their wallet) without knowing anything about the chain except what the transaction spending their money is doing with their money"?

We claim that our work can be used by any UTxO or EUTxO blockchain (with some adjustments to the details of intent specification).
We use blind signatures~\cite{blindsigs}.

\section{Related Work}


Our approach facilitates transaction submission for a light client that is not aware of the history of a blockchain.
Therefore, we compare our work with existing solutions for light clients as well as mechanisms that allow the interaction with a blockchain when in resource constrained operation.
Our approach is different from a more ``traditional'' (i.e., what commonly is referred to) light client that has the primary goal of syncing to the blockchain in order to acquire the information necessary to interact with a smart contract or to submit a transaction.
We summarize the most important concepts and works in the area of light clients below.

We first describe the common idea of a light client and then outline concepts related to our work that complement light clients.

A summary of \textbf{light client research} can be found in~\cite{soklc}, listing the main functionalities a light client is expected to perform: (1) issue queries, such as for the balance of an account, or the state of a transaction, and (2) safeguard secrecy information and submit transactions to the blockchain.
In order to implement these functionalities, light clients use several generic techniques, most notably: header verification and consensus evolution verification. Compared to a \emph{full node} a light client only verifies the headers of blocks and skips the verification of transactions and account balances, a technique made popular with SPV in Bitcoin~\cite{spv} and nearly adopted by most practical approaches for light clients, e.g.,~\cite{Ethereumlc}.
In addition to that, consensus evolution verification is needed for proof-of-stake based blockchains as the validator set can change.
Another common technique is to compress blockchain and/or ledger state to reduce the information a light client has to download to verify proofs and synchronize to the blockchain (see, e.g.,~\cite{flyclient}).
Some light clients use game-theoretic assumptions to implement the slashing of previously deposited collateral in case of misbehavior, for example~\cite{superlight}.
The main cryptographic building blocks that are used to realize those techniques are succinct representation and proofs, such as data accumulators (often Merkle trees) and commitments and SNARKs.
Furthermore, suitable signatures and hash functions are needed, for example, aggregate signatures and threshold signatures.

Most related works and ideas are \textbf{intents} and \textbf{solver networks} that attempt to establish a relationship between solvers and users via their (light) clients~\cite{ethresearch-solvers}.
A light client issues an intent via as an abstracted transaction object and solvers process the intent incentivized by transaction fee or intent execution reward. The users are then free to accept or reject a solver’s proposal.
Outsourcing transaction creation to a solver does not need for user to bridge assets cross-chain, as chains are abstracted away. In case the solvers are required to provide a deposit, slashing guarantees honest behavior of a rational actor.
While concepts around solver networks are relatively new there is currently no universal standard governing the specification for intents and abstract transaction objects. To the best of our knowledge, this is the first work describing

\textbf{Payment channels}.
We also briefly cover \textbf{payment channels} (also called \textbf{state channels}), which also constitute
a concept similar to our approach:
payment channels describe a type of off-chain mechanism for blockchains. They allow to establish a private payment
channel between two parties. The channel can be used to conduct a series of (similar) transactions without interacting
with the main blockchain. The creation of a payment channel requires locking funds in a smart contract, which is used
to conduct the transactions~\cite{payment-channel}. Some state channels, such as Hydra \cite{hydra}, not only allow
for simple payments between users, but simulate the majority of the on-chain transaction processing mechanism internally.

Participation in a payment channel usually presumes that a user is able to predict, to some degree, the nature of the
anticipated payments they will send or receive, and accepts that interactions within the channel will be limited
to users, contracts, and tokens, available in that channel (i.e. locked by the channel's smart contract).
Due to the specifics of the channel setup process, and the limited nature of these interactions, payment channels
are capable of greatly improving scalability
and reducing transaction processing times. The target user base for our light client design, however, are ones
that do not wish to engage in establishing relationships with other chain users, participating
in an on-chain setup process, or intend to make repeat or scheduled interactions with the same
users. The on-chain space- and cost-saving benefits of payment channels,
which are due to not posting every single transaction, are unlikely to be accessible to a user base which
we target with our low-commitment design. For these reasons, it does not make sense to limit themselves
to a specific payment channel.

Similarly, our design should be general enough to support any transaction or on-chain operation, irregardless of its monetary value or physical size.
One potential problem in payment channels is liquidity. If one of the parties does not have sufficient liquidity in the channel, payments might fail or need to be routed through multiple channels\footnote{Payment networks attempt to address this issue by using multi-path payments, where the original payment is split into smaller parts, each routed through different channels, allowing for greater liquidity}, adding complexity and risk.
Our construction does not suffer from those shortcommings while at the same time provinding the same level of atomicity expected from payment channels, i.e., payments are either fully successful or completely fail.

Some existing proposals\cite{todo} suggest to augment payment channels with the ability to process any off-chain interaction, not just payments. Often those constructions are called state channels. Off-chain interactions that go beyond transaction do however entail increased complexity. The base/main chain needs to offer smart contract support. Hash Time-Locked contracts (as used in the Lightning Network) are not sufficient for complex off-chain interactions.

Availability/Being online: Payment channels (and certain payment networks) require that the participants are ``always'' online and observe the main chain for updates (e.g., an updated smart contract).
A possibly remedy for this inconvenience are watchers. They make sure that channels are fair by monitoring the state of the payment channels. They check if either participant tries to broadcast an invalid transaction--- such as an old state---in order to claim more funds than they are entitled to. If a participant tries to cheat, the watcher can detect this and, trigger a penalty transaction that punishes the dishonest participant by forfeiting their funds.
Being always online is not mandatory for the user in our light client protocol. The user might observe the chain to verify if their transaction has been included and the funds have been transferred, however, not being online can never result in a scenario where a party wrongfully claims funds and active intervention on the user's part is required to prevent this.


\todobox{
some payment networks in the wild:\\
Lightning network (Bitcoin)
Raiden Network for Ethereum, which also utilizes payment channels for faster payments, or Liquid Network (a sidechain to Bitcoin)
Celer Network?
Stellar network?
RippleNet?
Celo network?
}


\textbf{API and Explorer Services}.
Many realistic UTxO ledger implementations (e.g. Ergo \footnote{\url{https://ergoplatform.org/}},
Cardano \footnote{\url{https://cardano.org/}}, and BitCoin \footnote{\url{https://bitcoin.org/}}) are set up
in a way that a badly constructed transaction will, in many cases, not result in any update to the ledger
state. So, even if a transaction is constructed by an LC based on ledger or block data that is incorrect or
old, the worst-case scenario is that it will be rejected. This makes
services such as blockchain explorers \footnote{\url{https://beta.explorer.cardano.org/}} or Blockfrost (API
as a service for accessing the Cardano blockchain) \footnote{\url{https://blockfrost.dev/}} are some of the
strongest competitors with our proposal, as they provide the data needed
for the LC to construct their transaction, often with a relatively high degree of reliability.
One aspect that sets our design apart is the SP compensation structure. Revenue from services such as those above is
either ad-based, requires users to create an account paid for with fiat currency, or is free in order to
promote use of the specific service/blockchain. We, on the other hand, propose that SPs are compensated
in assets on the same blockchain as the one to which their transaction gets applied, and the payment structure
consists of a one-time, no-setup atomic exchange of assets for services. Since the SP in our design gets paid
only once a transaction is successfully applied, all data for which payment they receive payment
is not just mostly reliable (as for existing services), but provably correct.

\subsection{Hierarchical Deterministic wallets and address discovery}
UTXO based blockchains, including Bitcoin and Cardano use the concept of Hierarchical Deterministic (HD) wallets \cite{bip32} where child addresses can be created in a deterministic way using a public key of a keypair $sk,pk$, and some short auxiliary data (the chain code). Given a public key $pk$, chain code $c$ and requested index $i$ for a child key, a hash function is used to produce a scalar $s_i$ enablind the derivation of the child key as $sk_i=sk+s_i$. It is also possible to use $s_i$ without knowing $sk$ to directly derive the corresponding public key $pk_i=pk\cdot g^{s_i}$. The same method is used to produce a child chain code $c_i$, so that multiple generations of keys are possible.

In our application, we can assume that a wallet can give the SPO a single public key and chain code, letting the SPO do address derivation and lookup on their side, using some agreed-upon bounds, or heuristics on the index depth. This keeps the communication cost from the light client to the SPO constant.


\todobox{
Compare our approach  with :
\begin{itemize}
    \item "Free" websites monitoring the chain -- mention they lack long-term sustainability. mention big APIs. blockfrost, etc.
    \item Bridges (trustless and trusted). only overview. maybe SOK paper. and describe how LC and bridge are related.
    \item Payment channels (require upfront capital) + Payment networks~\cite{cryptoeprint:2017/823}
    \item LCs that operate on single-prover model (eg. with an established relationship via deposit)
    \item LCs that operate on multi-prover model
	%\item Time lock micro payment channel~\cite{{10.1007/978-3-319-21741-3_1}
\end{itemize}}
\subsection{Technical Background}
UTxO model \textcolor{orange}{mr: do we still need this subsection?}


\subsubsection{Discrete Log Groups}
\begin{definition} A group generator \GGen is a probabilistic polynomial time (p.p.t.) algorithm with input a security parameter $\lambda$ and outputs a group description $\G,g,q$ such that $\G$ is a group of prime order $q\approx 2^\lambda$ with generator $g$. We say that the discrete logarithm problem is hard w.r.t. \GGen{} if for all p.p.t $\A$ we have that

$$\Pr[(\G,g,q) \gets \GGen(1^\lambda); t\gets \Z_q; h\gets g^t: t=A(\G,g,q,h) ] \mbox{ is negligible in }\lambda.$$
\end{definition}

\subsubsection{ Public Key encryption and Schnorr Signatures}

A public key encryption scheme $\PKE$ comprises a set of polynomial time algorithms $\Keygen,\Enc,\Dec$ with the following syntax:
\begin{itemize}
\item $\Keygen(1^\lambda) \to (ek,dk)$. Creates a public/private keypair $ek,dk$. The encryption key $ek$ also defines the message space $\mathcal{M}$
\item $\Enc(ek,m;\rho) \to C$. Encrypts a message $m\in \mathcal{M}$ under the public encryption key $ek$. We list the randomness $\rho$ used in the encryption as an input so that we can refer to it in other contexts (e.g. zero knowledge arguments).
\item $\Dec(dk,C) \to m$. Decrypts a ciphertext $C$ using the private decryption key $dk$.
\end{itemize}

\begin{definition} A public key encryption scheme $\PKE$ is \emph{correct} if 
$$\Pr[ (ek,dk) \gets \PKE.\Keygen(1^\lambda); m\gets \mathcal{M}:\PKE.\Dec(sk,\PKE.\Enc(pk,m))=m ] =1.$$
\end{definition}

\begin{definition} A public key encryption scheme $\PKE$ is \emph{IND-CPA secure} if for all stateful p.p.t. adversaries \A, the difference
$$\left| \Pr[ (ek,dk) \gets \PKE.\Keygen(1^\lambda); (m_0,m_1)\gets A(ek);d\gets\{0,1\}; c^*\gets \PKE.\Enc(ek,m_d):A(c^*)=d ] - {1\over2} \right|$$
is negligible in $\lambda$.
\end{definition}

The security of the Schnorr signature scheme has been well studied in the random oracle model, however in this work the need arises to instantiate the underlying hash function $\hash$ so that we may reason about it inside zero knowledge proof systems. This will render existing proofs ineffective necessitating an assumption on the security of the scheme. This does not differ significantly from typical implementation practices (where the hash function is drawn from a small pool for standardized options), and also from the treatment of this issue in the literature. +++refs  


\begin{definition} A hash function generator $\HGen(n) \to \hash$, on input $n\in \N$ generates a hash function $\hash : \{0,1\}^* \to \Z_n$.
\end{definition}


\begin{definition} The Schnorr digital signature scheme $\SDS$ is defined for a group generator $\GGen$ and a hash function generator $\HGen$ and operates as shown on figure \ref{fig:schnorr}.\end{definition}


\begin{definition} A digital signature scheme $\DS$ is \emph{correct} if 
$$\Pr[sp\gets \DS.\Setup(1^\lambda); (sk,vk) \gets \DS.\Keygen(sp); m\gets \mathcal{M_{S}}:\DS.\Ver(vk,m,Sign(sk,m))=1 ] =1.$$
\end{definition}

\begin{definition} A digital signature scheme $\DS$ is \emph{strongly existentially unforgeable against chosen message attacks}  (sEUF-CMA)  if for all p.p.t.  adversaries \A, we have  
$$\Pr[\textsf{GsEUF-CMA}^{\A}_{\DS}(1^\lambda)] $$
is negligible in $\lambda$ where the game $\textsf{GsEUF-CMA}^{\A}_{\DS}$ is defined in figure \ref{fig:seufcma}. \label{def:seufcma}
\end{definition}

\begin{assumption} \cite{XXX} There exists a group generator $\GGen$ and hash function generator $\HGen$ such that the Schnorr signature scheme $\SDS$ is strongly unforgeable under definition \ref{def:seufcma}. \end{assumption}

\section{Ledger Model}
\label{sec:model}

The ledger model to which we tailor our light client design is 
a UTxO ledger with multi-asset support ($\mathsf{UTxO}_{ma}$), first introduced in \cite{utxoma}. 
The UTxO ledger model, such as the one used by BitCoin \cite{bitcoin}, Ergo \cite{ergo},
and Cardano~\cite{full-cardano}, maintains a record, called the \emph{UTxO set}, 
of transaction outputs added 
by transactions that have been applied throughout its history, but not yet spent by 
subsequent transactions. We chose the $\mathsf{UTxO}_{ma}$ ledger because it allows us to demonstrate 
relevant usecases of our light client design (which would also work for a single-asset UTxO ledger),
without introducing unnecessarily complexity of the Extended UTxO ledger. 
For completeness, and in order to establish notation, 
we include an overview of the $\mathsf{UTxO}_{ma}$ model. For additional notation 
explanation, see Figure \ref{fig:notation:nonstandard}.

\textbf{Blocks and Ledger States.} A block $\Block = \Header \times [\Tx]$ is a data 
structure used to update the state of the ledger
by applying a list of transactions $\var{lstx} \in [\Tx]$ contained in the block, 
as well as doing some other 
checks and updates we do not model here. 
Among other data, the block header contains a slot number field $\fun{slot}~:~\Header \to \Slot$,
which represents the blockchain time at which the block is produced. 

The ledger state is a
data structure which is updated by applying blocks incoming on the network.
The ledger state $\LState$ contains (among other data we do not model here)
the UTxO set field, $\fun{utxo}~:~\LState \to \UTxO$.
It also contains the parameter $\fun{minfee}~:~\LState \in \N$, which is the minimum 
fee a transaction must pay.
A block $b$ can extend the blockchain (i.e. update the current state $s \in \LState$) whenever the function 

\[ \fun{checkBlock}~:~\LState \times \Block \to \B \]

applied as $\fun{checkBlock}~(s, b)$ returns $\true$. Then, the updated block state 
is computed by $\fun{updateState}~(s, b)$. We do not give full specifications of 
$\fun{checkBlock}$ or $\fun{updateState}$, as they are not required to model our light client approach.

\textbf{Full Nodes.}
Let $s_0$ be some verified state, e.g. a genesis state, 
or a verified checkpoint state, and suppose $[b_0, ..., b_k]$ is a list of blocks 
that have been disseminated across the network since the time slot of $s_0$. 
We assume that a \emph{full node} is one that is to able to (1) compute the current state $s_{k}$ by applying 
them in sequence, i.e. computing $s_{i+1} = \fun{updateState}~(s_i, b_i)$ for $0 \leq i < k$,
(2) is able to determine if all of the blocks in the list are valid
(i.e. $\fun{checkBlock}~(s_i, b_i) = \true$), and (3) is able to listen on the network for 
new blocks. If all the blocks are valid, we say 
that $[b_1, ..., b_k]$ forms a valid blockchain.


\textbf{Multi-asset Support}. A ledger which supports transacting with not only the 
primary currency, e.g. BitCoin on the BitCoin 
platform, or Ada on Cardano, but also other types of currencies, is called 
a \emph{multi-asset ledger}. 
Each asset is uniquely identified by an $\AssetID \leteq \Policy \times \TokenName$.
Arbitrary combinations of assets are specified using a finitely-supported map 
$\Value \leteq \AssetID \mapsto \Quantity$, where, for a given $v \in \Value$,
all assets whose IDs are not 
included in the domain of $v$ are assumed to have quantity $0 \in \N$. 
This data structure 
has a partial order $\leq$, and forms a group under addition $+$, with zero being 
the empty map $\emptyset$. In our model, all assets are user-defined, meaning that a
user can introduce any asset into circulation so long as minting of this asset is allowed 
by its minting policy (which may be defined by the user themselves).

\textbf{Ledger State and the UTxO Set.} 
The state of a UTxO-based ledger 
necessarily contains a UTxO set. While realistic ledgers often contain additional 
information in their state, in our model, the ledger state is just the UTxO set.
The UTxO set is a finite map, $\UTxO \leteq \TxIn \mapsto \TxOut$. 
A transaction updates the UTxO set by either adding and removing entries. 

\textbf{Transactions}. A transaction is the following data structure :

\begin{displaymath}
    \begin{array}{rll}
        \Tx &=&(\inputs: \type{Set}~{\TxIn},\\
        & &\ \outputs: [\TxOut],\\
        & &\ \fun{validityInterval}: \Interval{\Tick},\\
        & &\ \mint: \Value,\\
        & &\ \fun{fee}: \N \\
        & &\ \sigs: \Signature)
    \end{array}
\end{displaymath}

An input $(\var{txid}, \var{ix}) \in \TxIn \leteq \TxId \times \N$ is a pair 
of a transaction ID and a natural number. When a transaction is applied to a UTxO 
set, its set of $\inputs$ is used to identify the entries
which the transaction is removing from the set. In each input,
$\var{txid} \in \TxId \leteq \H{}$ 
is the hash of a (previous) transaction that added that entry to the UTxO, and $\var{ix}$ 
is the index of that output in the list of outputs of that transaction.

An output $(s, v) \in \TxOut \leteq \Script \times \Value$ is a pair of a script $s$
which specifies some constraints that are checked when the output is spent, 
and the assets $v$ contained in the output. The list $\outputs$ of outputs of a transaction 
$\var{tx}$ is used to construct a set of UTxO entries that will be added to the UTxO set, such 
that the unique identifier $\var{txin}$ of each output $o \in \outputs~\var{tx}$ consists of the 
transaction hash $\fun{txid}~\var{tx}$, and the index of $o$ in the list $\outputs~\var{tx}$.
The entires added to the UTxO set by $\var{tx}$ are computed in this way by $\fun{mkOuts}$, 
see Figure \ref{fig:eutxo-types}. 

The interval $\fun{validityInterval}$ specifies the range of slot numbers for which a transaction
can be valid. The field $\sigs : \Signature \leteq \pubkey \mapsto \H $
is a set of public keys, associated with their 
signatures on the the transaction (excluding $\sigs$ itself). 
The $\fun{fee}$ is the amount of primary currency a transaction pays as a system fee,
which is checked to be at least the required fee $\fun{minfee}$.
The $\mint$ field represents the assets being minted or burned by the transaction. 
Assets with positive quantities are said to be minted, while those with negative quantities 
are burned. When a transaction is applied, the constraints specified by every 
$p \in \Policy \leteq \Script$ of each type of asset specified in this field are 
checked to make sure minting/burning of this type and quantity of asset is allowed. 

\textbf{Ledger State Update. } Given a UTxO set $\var{utxo}$ and a transaction $\var{tx}$, the function 
$\fun{updateUTxO} : \UTxO \times \Tx \to \UTxO$
computes the updated UTxO set by adding and removing the appropriate entries :

\[\fun{updateUTxO}~(\var{utxo},~\var{tx})~=~\{~i\mapsto o \in \var{utxo} ~\mid~ i \notin \fun{inputs}~(tx)~\} \cup \fun{mkOuts}(tx)\]
    
While an update to the UTxO set can be computed for any transaction, only transactions that are
\emph{valid} for a given set are allowed to perform an update to the ledger state. For a given
$\var{utxo}$ set, a transaction $\var{tx}$ 
is valid whenever the function $\fun{checkTx} : (\Slot \times \N) \times \UTxO \times \Tx \to \B$,
applied as $\fun{checkTx}~((\var{slot}, \var{fee}), \var{utxo},~\var{tx})$, returns $\True$. The $\fun{checkTx}$
function is the conjunction of the constraints specified in Section \ref{sec:check-tx}.
For a given block $b$ and state $s$, the function $\fun{updateState}~(s, b)$ updates 
$s$ with the list of transactions $\pi_2~b = [\var{tx}_1 ; ... ; \var{tx}_k ]$ in such a way that 
the update to the UTxO set contained in $s$ is computed by applying the transactions in sequence, i.e.

\[ \fun{updateUTxO}~(\fun{updateUTxO}~((... \fun{utxo}~s ...),~\var{tx}_{k-1}),~\var{tx}_k) = \fun{utxo}~(\fun{updateState}~(s, b)) \]

For each transaction in the list, $\fun{checkTx}~((\fun{slot}~b, \fun{fee}~s), \var{utxo}_i,~\var{tx}_i) = \true$ is first checked, 
and the entire block is considered invalid if this check fails.
As part of checking transaction validity, constraints of every $\Script$ run by the transaction are checked. 
The constructors and evaluation of $\Script$ is given in Figure \ref{fig:script}, 
and $\fun{MOf}$ is given in \ref{fig:eutxo-types}. A script, for a given set of 
signer keys $\var{khs}$ and slot numbers $s1, s2$, can be defined to check that 
(some specific) $m$ of them have signed the transaction (are included in the domain of $\Signature$), 
and/or that the validity interval of the transaction starts after $s1$ and/or ends before $s2$.








\section{Light Client Specification}

 We will model light clients in terms of their functionality, restrictions and protocols for communication with full nodes. By functionality, we refer to the interaction between a light client and its user(s). To satisfy user requests a light client will need to connect to and query a number of full nodes via a set of protocols prescribed by its design.


 We describe functionality of a light client in terms of reading and posting \emph{intents} denoted as two sets of predicates $\mathcal{I}_\mathsf{read}$, $\mathcal{I}_\mathsf{post}$. When queried on a reading predicate $\mathsf{int_{read}} \in \mathcal{I}_\mathsf{read}$ at time $t_0$, the light client replies with all transactions $tx$ such that $\mathsf{int_{read}}(tx,\LState_t)=1$, where  $\LState_t$ is the state of the blockchain at time $t \geq t_0$. When queried on a posting predicate $\mathsf{int_{post}} \in \mathcal{I}_\mathsf{post}$ at time $t_0$, the light client prepares a transaction $tx_p$ such that $\mathsf{int_{post}}(tx_p,\LState_t)=1$, where $t \geq t_0$ and broadcasts it to the blockchain.

\todobox{ [TODO: revisit time]
 [TODO: can we do the LC part of bridges with these intents?]}

 In order to provide answers to queries by the user, a light client participates in a number of protocols between itself and a set of full nodes. These protocols may be one to one or one to many, depending on the design of the light client. We will expand on the syntax and operation of such protocols in the following subsection.

 Lastly, a light client is characterized in terms of restrictions placed on the resources it has available (without restrictions, 
 we could treat a full node of a chain as a light client). These restrictions can relate to the  
 communication cost to answer a series of queries, available storage, and assumptions about connectivity. On the other hand,
  a particular light client design might limit the intents contained in $\mathcal{I}_\mathsf{read}$, $\mathcal{I}_\mathsf{post}$
  to simplify its design and limit its functionality.


\subsection{Intent Specification}
\label{sec:dsl}

We define the type of \emph{abstract transaction}, $\TxAbs$,
which is the data structure a light client receives from the service provider instead of
the full plaintext transaction, see \ref{fig:abs-tx}. This type does not contain
the $\fun{auxData}$ or the $\outputs$ transaction fields, and instead of the $\Signature$
field, it contains only the signing keys $\fun{sigKeys}$ and not the corresponding signatures.
For the implementing some example the usecases of our design, we require the definition of the
following functions, given in Figure \ref{fig:dsl}, and Figure \ref{fig:mktospec}, respectively :

\begin{itemize}
    \item[(i)] The function that construct a transaction based on the specification
    \[ \fun{mkToSpec} : \LState \times \mathcal{I}_\mathsf{post} \to \Tx \]
    \item[(ii)] The function that checks that a given abstract transaction matches the specification
    \[\fun{chkSpec}~(\mathcal{I}_\mathsf{post} \times \TxAbs) \to \B\]
\end{itemize}

\begin{figure}
    \textsc{$\mathcal{I}_\mathsf{post}$ Constructors}
\begin{displaymath}
\begin{array}{rlll}
    \fun{MustMint}           &: \Value \to \mathcal{I}_\mathsf{post}\\
    \fun{SpendFrom}         &: \Script \to \mathcal{I}_\mathsf{post}  \\
    \fun{MaxInterval}   &: \Slot \to \mathcal{I}_\mathsf{post}  \\
    \fun{PayTo}  &: (\Value \times \Script)  \to \mathcal{I}_\mathsf{post}  \\
    \fun{ChangeTo} &: \Script \to \mathcal{I}_\mathsf{post}  \\
    \fun{MaxFee} &: \N  \to \mathcal{I}_\mathsf{post} \\
    \fun{AndExps}  &: [\mathcal{I}_\mathsf{post}] \to \mathcal{I}_\mathsf{post} \\
\end{array}
\end{displaymath}
\nextdef
\textsc{Evaluation of $\mathcal{I}_\mathsf{post}$}
\begin{displaymath}
\begin{array}{rlll}
  \chkSpec{\_} &:& \mathcal{I}_\mathsf{post} \to \TxAbs \to \B \\
  \chkSpec{\fun{MustMint}~v} (\var{tx}) &=&  v \leq \var{tx} . \mint \\
  \chkSpec{\fun{SpendFrom}~s} (\var{tx}) &=& \applyScript{s}(\fun{dom}~(\var{tx} . \sigs), \var{tx}.\validityInterval) \\
  \chkSpec{\fun{MaxInterval}~i} (\var{tx}) &=& (\var{tx}.\validityInterval)_2 - (\var{tx}.\validityInterval)_1 \leq i \\
  \chkSpec{\fun{PayTo}~(s, v)} (\var{tx}) &=& (s, v) \in \var{tx} . \outputs  \\
  \chkSpec{\fun{ChangeTo}~s} (\var{tx}) &=& (s, \fun{consumed}~-~\fun{produced}) \in \var{tx} . \outputs  \\
  \chkSpec{\fun{MaxFee}~f} (\var{tx}) &=&  \var{tx} . \fun{fee} \leq f \\
  \chkSpec{\fun{AndExps}~[a1 ; a2 ; ... ; ak]} (\var{tx}) &=& (\chkSpec{a1}~\var{tx}) \wedge (\chkSpec{a2}~\var{tx}) \wedge ... \wedge (\chkSpec{ak}~\var{tx})
\end{array}
\end{displaymath}
\caption{$\mathcal{I}_\mathsf{post}$ constructors and evaluation}
\label{fig:dsl}
\end{figure}

We these functions are defined such that for any $l,~i$, such that $\fun{mkToSpec}~l~i~\var{tx}_t \neq \fun{nothing}$,
necessarily $\chkSpec{i}~(\fun{mkAbs}~(\fun{mkToSpec}~l~i~\var{tx}_t))$,
see Appendix \ref{sec:chk-mk} for a proof sketch.
Note that in a realistic system, however, more constraints need to be
included in this DSL, e.g. a way to limits transaction size, or support for specifying
desired token exchanges. Let us define two examples of intent specifications :

\begin{itemize}
    \item[($i_1$)] Intent to mint some token $t$ within an interval of length $j$ and maximum fee $f$ :
    \[ i_1 =  \fun{AndExps}~[ \fun{MustMint}~t ; \fun{MaxInterval}~j ; \fun{MaxFee}~f ]\]

    \item[($i_2$)] Intent to pay $x$ from outputs locked by $\fun{RequireSig}~k1$ to  $\fun{RequireSig}~k2$
    \[ i_1 =  \fun{AndExps}~[ \fun{SpendFrom}~(\fun{RequireSig}~k1) ; \fun{PayTo}~(\fun{RequireSig}~k2,~x) ]\]
\end{itemize}

The resulting transactions are given in Figure \ref{fig:txs}.

\todobox{
    define fig:txs,  Appendix \ref{sec:chk-mk}
}


\todobox{Give example of :

Set of transactions - that satisfy the intent
Examples
Assume there is some number client can generate that corresponds to how many public keys server should check - high communication complexity for some users
Is the hash unpredictable?
SPs can respond to queries with no proof , because LC will have to ask for tx construction to use that info anyways
spv and Ethereumlc citations?
SP can send the client the completed transaction as soon as the sP observes it being included into a block
SPs can provide services for free in exchange for delegating to their stake pool

TxData - what is it?
DSL
Security - somehow sign a whole class of transactions?}


 \subsection{Light Client Interactions}

The light client interacts with a number of full nodes. We assume that all parties know the current time up to an allowable deviation $\delta_t$. We allow for protocols featuring a single or multiple full nodes, and (orthogonally) protocols that are supported by a smart contract at the cost of additional setup. An important difference between the single and multi-node model is that existential honesty trivialises the single-node case. As such, single-node protocols are usually required to operate under more stringent security assumptions.

The end result of such protocols are typically one of the following: an update to the light client's state or the creation of a transaction to be posted on the blockchain.  In Section \ref{sec:threats} we give an informal treatment of potential threats and in Section \ref{sec:protocols} we describe a number of such protocols and formalize the corresponding security notions.

\todobox{
 \begin{itemize}

\item User - LC interface

 \item LC characteristics/ limitations
   \begin{itemize}
\item What are the capabilities of a light client?

\item Does it remember all addresses it has been paid at (tx history)?

\item Do we assume that viewing keys exist? Can we assume LC can generate first x addresses from its private key in a
deterministic way?

\item Is light client allowed to maintain state, and what state can they maintain if so? Secret key is the minimum state.
\item Sanity test: if we dismiss all requirements we should recover a full client.
\end{itemize}

 \item LC - Full Node interactions
  \begin{itemize}
  \item Protocols to support user requests
  \item Security reqs (protecting integrity/ privacy/ SPO revenue?)
\end{itemize}
 \end{itemize}}


\todobox{Can we describe the differences between light wallets, bridges and light nodes in our framework? (Probably yes).

How can we formalize the intent of a light client without revealing secret key?

Can we have viewing keys?}


\section{Threat Model and Threat Mitigation}
\subsection{Model Requirements}
\paragraph{Rational Actors} For liveness, we assume that protocol participants are rational in the sense that they will opt to complete tasks that benefit them directly (i.e in the form of rewards, or in the form of a desired transaction being posted). For safety we assume parties can be fully malicious.

\paragraph{Transaction Expiry} For safety, we require that transactions can be set to be valid only within a certain time window. This is to prevent trivial attacks where a malicious SPO delays posting a transaction indefinitely. The user could contact a different SPO to create a second transaction with similar parameters (e.g. by paying the same recipient with different coins) with the risk that the first SPO would also post their transaction, making the user pay twice.

\paragraph{Idempotent Doublespends} For safety we also require that attempted double spends are idempotent, i.e. they are simply not valid transactions and do not penalize (slash) the user for signing them. 

\label{sec:threats}


\textbf{Erroneous User Balance.} A common threat vector is convincing the user of an erroneous balance on their account. This can lead a user to
complete a transaction outside the blockchain based on wrong information (e.g. accept an invalid transaction as payment for goods, or exchange his private key
in the understanding that is controls $x$ stake instead of $y$.)

\textbf{Invalid Transactions.}  In response to an LC query, it may be desirable for an SP to construct a transaction that does not meet the intent specified by the user. For example, the created transaction may pay the wrong recipient, or pay an amount over and above the one specified by the user.

\textbf{Protocol Failures.}  In a multi-node protocol, a malicious user may wish to force the protocol to abort even though there exist a number of honest nodes who would otherwise ensure the protocol completes successfully. In the single node, a malicious node can always cause a protocol to fail by not replying.

\textbf{Suboptimal Transactions.}  In response to an LC query, it is possible for an SP to construct a
\emph{sub-optimal} transaction which nevertheless matches the LC's specification. For example, the LC may
include a larger-than-necessary system fee in the transaction, or not provide the LC with the best
available exchange price for a specific token.
Our model assumes that an LC is able to submit their transaction to multiple SPs, and will have pick one
transaction to sign from the resulting set $S$ of responses received within some pre-defined amount of
waiting time. Alternatively, the LC may simply choose the first valid transaction sent by an SP.
Assuming such a competitive SP environment, the suboptimal transaction threat is mitigated by
the incentive, in the form of a service fee for the SP, to provide both quickest and the most optimal
response. Note that the SP is not a-priori aware of how the LC will make
the choice of what transaction to sign, and how long their waiting period is.

\textbf{Response Delays and Stale State.}
\label{sec:stale-state}

Arbitrary delay caused by SP: SP can offer to execute an intent and provide the crafted TX, but never publish it on the chain.
    For this reason, SP should provide the hashes of the UTxO references (or similar) to the light client, s.t. the LC can reuse those references in another transaction. This avoids the case where SP hangs on to the TX indefinitely or posts it at an arbitrary point in time causing two or more TXs being executed if the LC resubmits its intent/TX.

\textbf{Non-submission of Transactions.} A class of protocols deals with the joint creation of transactions by the light client and node, with the transaction submission being left to the full node. It is possible for the full node to not post such a transaction in a timely manner, potentially leading the user to retry at a different time. As intents are not necessarily satisfied by a unique transaction, it is possible that the first full node may be able to post the first transaction (which is purposefully delayed) after the posting of the second one. This can cause the user to e.g. double pay for a service as well as the full node fee.

\textbf{Falsified or Unsafe Transaction Data}. SP lies to client about the UTxOs being spent by the tx, and tricks them into doing something they did not want to do (that is why 0-knowledge proof that outputs were correctly specified)

\textbf{Data Leakage.} The 2-PC protocol we propose is one that has leakage. In particular,
the co-construction of a transaction by the LC and the SP allows the LC to learn certain information
about the chain state, such as, e.g. the upper and lower bound of the interval containing the current blockchain
time. In fact, under certain circumstances, after several iterations of our SP-LC protocol, the LC
may gain the capacity to construct certain kinds of transactions themselves. As noted above, this
information can also be gained from blockchain explorer services. The
reason an LC may engage in our protocol that it allows all transaction-construction,
and state-tracking and querying logic to be off-loaded to the SP. If the LC can do all this themselves,
engaging in the protocol does not benefit them in any way.

However, there is a specific kind of
data leakage threat we have have built the protocol to be resilient against : leakage of $\UTxO$ entry
identifiers, i.e. the inputs of a transaction. It is impossible to construct a valid transaction
without knowing the the inputs that need to be included in it for it to match the specification,
and this information cannot be guessed without knowledge of blockchain state. If this information
is somehow leaked to the LC along with the modified transaction, it makes it possible to for the
LC to not pay the SP for their work.

Specifically, the full transaction (including the correct inputs) can be modified locally to
replace the output containing the SP's fee with one returning it as change to the LC, signed, and
submitted to the network by the LC. As part of our protocol, the SP replaces the inputs of a
transaction with outputs that correspond to those inputs in the UTxO set.
We have demonstrated that guessing inputs without access to either the inputs themselves, or
the transactions that have placed them into the UTxO set, is sufficiently difficult to not be considered a
vector of attack.
This data may also be leaked by getting recorded in a script, datum, or redeemer. E.g., a script
may require that its redeemer also list the inputs of the transaction. The responsibility of mitigating
this type of leakage rests on the SP, who can address this by limiting the kinds of specifications
they agree to fulfill.

\todobox{TODO : make sure we have shown that "We have demonstrated that guessing inputs
without access to either the inputs themselves, or
the transactions that have placed them into the UTxO set,
 is sufficiently difficult to not be considered a
vector of attack."}

\section{Light Client Protocols}
\label{sec:protocols}
We given an overview on our light client protocol and its components.
As the protocol is uniquely suited for UTxO-based blockchains, we develop our approach
to support the general ledger model explained in Section~\ref{sec:model}
and the intents set forth in Section~\ref{sec:dsl}.

\subsection{Outline}
The following computations are performed in the protocol

\begin{itemize}
    \item[(i)] Intent specification is generated (see Section \ref{sec:dsl})
    \item[] \textcolor{orange}{mr: do we still need this extra signature?}
    \item[(ii)] Signature is generated as follows :
    \[\var{sig} = \fun{sign}~(\var{int_{post}}, sk_{\mathrm{client}})\]
    \item[(iii)] The signature is checked as follows :
    \[ \fun{checkSig} (\var{int_{post}}, vk_{\mathrm{client}}, \var{sig}) = \true \]
    \item[(iv)] The transaction is generated as follows, including a randomly generated bytestring
    in the $\fun{auxData}$ field for extra entropy :
    \[ \var{tx} = \fun{mkToSpec}~(\var{ls},~\var{int_{post}},~\var{tx}_t)~\{~\fun{auxData}~=~\fun{random}~\} \]
    \item[(v)] Abstract transaction is generated as :
    \[ \var{tx}_A = \fun{mkAbs}~\var{tx}  \]
    \item[(vi)] Blind Signature Commitment (see Section \ref{sec:blind} for details)

    \item[(vii)] Intent specification Proof (see Section \ref{sec:dsl}) :
    \[ \chkSpec{(\var{int_{post}})}(\var{tx}_A )= \true \]
    \item[(viii)] Proof Check. Blind signatures are computed by (see Section \ref{sec:blind} for details), with $\var{sks} = \fun{getSKs}~\fun{dom}~(\var{tx} . \fun{sigs})$ :
    \[ \var{sigs} = \{~(s,~\fun{blindSign}~(h,~\var{tx}_A,~ s)) \vert s \in sk_{\mathrm{client}} \cup ~\var{sks}~\} \]
    \item[(ix)] Signatures are computed from blind signatures (see Section \ref{sec:blind} for details) :
    \[ \var{sigs'} = \{~(s,~\fun{mkRealSig}~(s,~h,~\var{tx}_A,~k)) ~\vert~(s, k) \in \var{sigs}~\} \]
    \item[(ix)] Signatures are added to the transaction
    \[ \var{tx'} = \var{tx}~\{~sig = \var{sigs'}~\} \]
\end{itemize}

\subsection{Definitions and Requirements}

\begin{definition}[Correctness] A transaction building protocol is \textbf{correct} if for an honest light client, and service provider and any state $\LState$ and intent $\mathcal{I}_\mathsf{post}$ such that: if $\fun{mkToSpec}(\LState, \mathcal{I}_\mathsf{post}) \neq \bot$, then the protocol completes and the transaction $\Tx'$ produced by the SP is such that $\fun{\fun{checkTxL}~}(LState, \Tx')=1$. \end{definition}

\todobox{ Let ${\fun{checkTxL}~}(LState, \Tx)$  be shorthand for  $\fun{checkTx}~((\var{slot}, \var{fee}), \var{utxo},~\var{tx})$ }.


\begin{definition}[Safety] A transaction building protocol is \textbf{safe} if for an honest light client and any service provider SP we have that: if transaction $\Tx('$ is produced by the SP is such that  $\fun{\fun{checkTxL}~}(LState, \Tx')=1$
%against the client's key (XXX note: maybe we can ommit this)
, then it must be that $\fun{chkSpec}(\mathcal{I}_\mathsf{post} , \Tx')=1$. \end{definition}

\begin{definition}[Private until posted] A transaction building protocol has the  \textbf{private until posted} property if a PPT adversarial client $C_\mathcal{A}$ cannot win the following experiment with probability significantly higher than $1\over 2$.

For an honest service provider SP and any client $C_\mathcal{A}$ we have that: client provides 2 states $\LState_0, \LState_1$ and one intent $\mathcal{I}_\mathsf{post}$ such that for $i=0,1$ it is $ \fun{mkToSpec}(\LState_i, \mathcal{I}_\mathsf{post}) \neq \bot$.  The experiment flips a coin $d$ and runs the protocol between SP and C* using state $\LState_d$. $C_\mathcal{A}$ outputs $d^*$. The adversarial client wins iff $d=d^*$. \end{definition}



\textbf{Single-SP vs Multi-SP Protocols. }
We describe a basic protocol in which a single LC queries a single SP for constructing a
transaction matching LC's specification, and then eventually that transaction getting posted
after all protocol steps are completed. We assume that all specification-meeting
transactions are acceptable to the LC. However, we can also formulate a version of this
protocol where the LC instead sends the specification to multiple SPs, selecting the best
response, and engaging in the rest of the protocol only with that SP.

To do this, let $\fun{opt} : \TxAbs \to \Z$ be a function that rates abstract
transactions to express LC's preferences. For example, the total amount of primary tokens
spent by the transaction can be such a function :
$\fun{opt}~\var{tx} = \fun{coinValue}~(\sum_{o \in~\var{txd}.\fun{spentOuts}} \var{o}.\val)$.
A transaction $\var{tx}$ is \emph{optimal} in a set $S \in \type{Set}~\TxAbs$ when

 \[ \fun{opt}~\var{tx} = \fun{min}~\{~\fun{opt}~\var{tx'}~\vert~\var{tx'}~\in~S~\} \]

To get a multi-SP protocol, the first step of the single-SP protocol in Figure \ref{fig:protocol}
must be augmented, so that the pair $(\fun{opt}, \var{int_{post}}$ is sent to each SP instead of just
sending $\var{int_{post}}$.
Upon receiving transactions $\var{tx}_{A,i}$, with $0\leq i < k$, from each of the $k$ SPs responding
to LC's query, LC will engage in the rest of the protocol only with the sender of $\var{tx}_{A,i}$.

 Note that the $\fun{opt}$ function and the specification serve different purposes in the protocol.
 The specification is checked, and any response transaction that does not satisfy it is discarded.
 On the other hand, it is not required that a transaction be optimal across all possible
 specification-satisfying transactions.  The kind of
 sophisticated optimization (such as what is required, e.g., for optimized order-matching) would require
 an entirely distinct set of tools for demonstrating the optimality result, such as ZK proofs about
 the full blockchain state (rather than just the associated transaction), together with evidence that
 the proof is about state that is \emph{sufficiently current} (see Section \ref{sec:stale-state}).
 For example, a proof that
 that there were no better offers for a specific token available on the ledger at the time the SP
 produced a response to the LC. We do not assume that
 either the SP or the LC are necessarily capable of performing or verifying (resp.) optimality according to the
 function LC requested be optimized, but an LC is capable to comparing transactions using $\fun{opt}$.



\subsection{Protocol Description}
We show our light client protocol that leverages weakly blind signatures in Figure~\ref{fig:wbps_protocol}.

\dots


\tikzset{
  leftArrowBelow/.style={
    append after command={
      (\tikzlastnode.south east) ++(0,0) edge[->, thick] ([yshift=0pt]\tikzlastnode.south west)
    }
  },
  rightArrowBelow/.style={
    append after command={
      (\tikzlastnode.south east) ++(0,0) edge[<-, thick] ([yshift=0pt]\tikzlastnode.south west)
    }
  }
}
\begin{figure}[ht]
\begin{tikzpicture}[every node/.style={font=\small}]
    \label{fig:protocol-short}
	\matrix (m)[matrix of nodes, column  sep=0.1cm,row sep=0mm,
			nodes={draw=none, anchor=center, 
			minimum height=3mm},
			column 1/.style={nodes={minimum width=4.2cm}}, 
			column 3/.style={nodes={minimum width=5.2cm}}, 
			column 2/.style={nodes={minimum width=2.3cm}} 
			 ]{
		\textbf{Signer (LC)} & & \textbf{Service Provider (SP)}\\
    & & Obtain newest $\LState_t\ ls$\\
		Generate intent $\var{int_{post}} \in \mathcal{I}_\mathsf{post}$\\
		& \node[rightArrowBelow] {$\var{int_{post}}, vk_{\mathrm{client}}$};  \\
		& & Determine verif. key(s) $\var{X}_{\mathrm{client}}$ \\
		& & from $(\var{int_{post}}, vk_{\mathrm{client}})$\\
		& & Generate $\var{tx} = \fun{mkToSpec}(\var{ls},\var{int_{post}})$,\\
    & & note $\var{nt}$, $\var{tx}_A = \fun{mkAbs}(\var{tx})$,\\
    & & $com_\var{tx}$ = $\var{Com}(\var{tx}{\color{blue}||\var{nt}};\rho)$\\
		& &  \\
		& \node[leftArrowBelow] {$com_\var{tx},\var{tx}_A,X_{\mathrm{client}}$}; & \\
		%\ h=\fun{txid}~\var{tx}
		% %& ZK.proof $p$ of $P (h,\  \var{tx}_A) = \true$ & \\
		Produce blind sig. \\
		commitment $\var{R}=g^r$ \\
		% sign abstract transaction\\[-7mm]
		&  \node[rightArrowBelow] {$\var{R}$};\\
		& & Produce challenge \\
    & & $c=H(R,X_{\mathrm{client}},\var{tx}{\color{blue}||\var{nt}})$ and\\
		& & proof $\pi$ : $\fun{chkSpec}(\var{int_{post}},\var{tx}) = 1\ \land$\\
	 	& & \quad\quad\quad $com_\var{tx} = \var{Com}(\var{tx}{\color{blue}||\var{nt}};\rho)\ \land$\\
		& & \quad\quad\quad $c=H(R,X_{\mathrm{client}},\var{tx}{\color{blue}||\var{nt}})$ \\
		&  \node[leftArrowBelow] {$(c,\pi)$}; & \\
		% %& ZK.proof $p$ of $P (h,\  \var{tx}_A) = \true$ & \\
		 Check proof $\pi$ \\
		Produce  $s=r+cx$ \\
		% sign abstract transaction\\[-7mm]
		&  \node[rightArrowBelow] {$\var{s}$}; \\
		& & Produce signature $\sigma = (R,s)$\\
		& & Post $(\var{tx}{\color{blue}||\var{nt}},\sigma)$ on chain.\\
	};
	\draw[thick] (m-1-1.south east)--(m-1-1.south west);
	\draw[thick] (m-1-3.south east)--(m-1-3.south west);
	\node[fit={(m-8-3.south west) (m-25-3.south east) (m-12-1.north west)},
   ultra thick, dotted, inner xsep=11pt, inner ysep=0pt,
    rounded corners=1mm, draw=gray, label={[gray,anchor=north west]north west:WBPS Execution}]{};	
\end{tikzpicture}
\caption{WBPS protocol between light client and service provider.}
\label{fig:wbps_protocol}
\end{figure}

%\begin{tikzpicture}
%    \label{fig:protocol}
%	\matrix (m)[matrix of nodes, column  sep=1.5cm,row  sep=7mm, nodes={draw=none, anchor=center,text depth=0pt} ]{
%		\textbf{Light Client (LC)} & & \textbf{Service Provider (SP)}\\[-4mm]
%		Generate intent & & \\[-7mm]
%		(i) $\var{int_{post}} \in \mathcal{I}_\mathsf{post}$, & & \\[-7mm]
%		(ii) generate signature $\var{sig}$  & & \\[-7mm]
%		% and predicate $\fun{opt} : \TxAbs \to \B$\\
%		&  $(\var{int_{post}}, vk_{\mathrm{client}}, \var{sig})$ & \\[-7mm] %\fun{opt}(\ ),\
%		& & (iii) Check signature \\[-7mm]
%		& & (iv) Generate transaction  \\[-7mm]
%		& & and (v) initiate WBPS signing \\[-7mm]
%		&& with abstract $\var{tx}_A$  \\[-7mm]
%		\phantom{x}& $\var{tx}_A$ & \\[-7mm]
%		%\ h=\fun{txid}~\var{tx}
%		% %& ZK.proof $p$ of $P (h,\  \var{tx}_A) = \true$ & \\
%		(vi) Produce Blind.  \\[-7mm]
%			Sig. Commitment \\[-7mm]
%		% sign abstract transaction\\[-7mm]
%		& $\var{R}$\\[-7mm]
%		& & (vii) Produce proof   \\[-7mm]
%		& & $\fun{chkSpec}(\var{int_{post}},\var{tx}_A) = 1$ \\[-7mm]
%		& $(c,\pi)$ & \\[-7mm]
%		% %& ZK.proof $p$ of $P (h,\  \var{tx}_A) = \true$ & \\
%		(viii) Check proof \\[-7mm]
%		Produce pre signature \\[-7mm]
%		% sign abstract transaction\\[-7mm]
%		& $\var{bsig}$\\[-7mm]
%		& & (ix) Produce signature $\sigma$\\[-7mm]
%		& & from pre signature.\\[-7mm]
%		& & Post $(\var{tx'},\sigma)$ on-chain.\\
%	};
%
%	\draw[shorten <=-1.5cm,shorten >=-1.5cm] (m-1-1.south east)--(m-1-1.south west);
%	\draw[shorten <=-1.5cm,shorten >=-1.5cm] (m-1-3.south east)--(m-1-3.south west);
%	%\draw[shorten <=-1cm,shorten >=-1cm,-latex] (m-6-2.south west)--(m-6-2.south east);
%	\draw[shorten <=-1cm,shorten >=-1cm,-latex] (m-10-2.south east)--(m-10-2.south west);
%	\draw[shorten <=-1cm,shorten >=-1cm,-latex] (m-13-2.south west)--(m-13-2.south east);
%	\draw[shorten <=-1cm,shorten >=-1cm,-latex] (m-16-2.south east)--(m-16-2.south west);
%	\draw[shorten <=-1cm,shorten >=-1cm,-latex] (m-19-2.south west)--(m-19-2.south east);
%	%\draw (m-10-2.north west) rectangle (m-19-2.south east);
%	\node[fit={(m-10-1.north west) (m-18-1.north west) (m-21-3.south east) (m-15-3.south east)},
%      ultra thick, inner sep=0pt, rounded corners=1mm,
%      draw=cyan, label={[cyan,align=center]270:WBPS Execution}]{};
%\end{tikzpicture}



 \section{Blind Signatures for Abstract Transactions}
 \label{sec:blind}
 In order to allow the LC to sign an abstract transaction artifacts we implement blind signatures on (partially) blinded transaction objects.
 At the minimum the SP hides the inputs to the transaction, i.e., the references to UTxO objects which are present in the ledger and required to cover the transaction.

 In Bitcoin, for example, UTxOs are captured in the ledger as follows:
 For Cardano, the structure looks as follows:

 \todobox{ Describe UTxO briefly for all chains we support?  }.


 The specific ledger implementation might require slight adaptations in terms of signature type and used hash functions---in this work, we give a very general construction that can be tailored to specific blockchains.

 Our construction is inspired by the predicate blind signature mechanism in~\cite{blindsigs}.
 It realizes a concurrently secure blind and partially blind signing protocol resulting in  standard Schnorr signatures.
 The signature scheme builds upon the original protocol proposed in~\cite{10.1007/3-540-48071-4_7} with the addition of a commitment phase that binds the signer to her secrets (blinding value and message), preventing a forgery attack (described in~\cite{bibid}) and making the scheme unforgeable under concurrent sessions.

 We describe how this protocol can be applied and simplified for abstract transaction signing.
 The proposed simplification is applicable in our setting due to the fact that our protocol does not rely on concurrent singing sessions.
 Furthermore, unlinkability of signatures and artifacts.
 Once the SP publishes final (unblinded) transactions on the chain, the LC can in theory download recent blocks and observe the transaction that was posted by the SP on behalf of the light client. If downloading is too resource-intensive, the LC might also consult an online block explorer for said purpose. As a consequence, abstract transactions, published transactions and signatures are assumed to be ``linkable'' and unlinkability provided by the signature scheme is redundant in our scenario.



 \todobox{
    \begin{itemize}
 \item Do we consider TLS connection between LC and SP? A network adversary could otherwise link/localize abstract transactions and transactions posted on the ledger.
 \item specify the signing algorithm : $\fun{sign} : (\H, \privkey) \to \H$
 \item specify the function to make real sigs from blind sigs $\fun{mkRealSig} : (\pubkey \times \TxId \times \TxAbs \times \H) \to \H$
\item specify the blind signing algorithm $\fun{blindSign} : \H \times \TxAbs \times \privkey \to \H$
\end{itemize} }.



\subsection{Weakly Blinded Predicate Signatures }

We adapt the definitions of [XXX] to account for the weak variant of blindness.

A WBPS scheme is parameterized by a family of polynomial-time-computable predicates,
which are implemented by a p.t. algorithm $P$, the predicate compiler: on input a predicate
description $prd \in {0, 1}$ and a message $m \in {0, 1}$, $P$ returns 1 or 0 indicating whether $m$
satisfies $prd$.

 A WBPS scheme \WBPSP for predicate $P$ is defined by the following algorithms. We focus on
schemes with 2-round (i.e., 4-message) signing protocols for concreteness.

\begin{itemize}
\item $Setup(1^\lambda) \to par$: the setup algorithm, on input the security parameter, outputs public
parameters $par$, which define a message space $\mathcal{M}_{par}$.
\item $KeyGen(par) \to (sk, vk)$: the key generation algorithm, on input the parameters $par$, outputs
a signing/verification key pair $(sk, vk)$, which implicitly contain $par$, i.e., $vk = (par, vk')$.
–\item $\langle Sign(sk, prd),User(vk, prd, m) \rangle \to (b, \sigma)$: an interactive protocol with shared input $par$
(implicit in $sk$ and $vk$) and a predicate $prd$ is run between the signer and user. The signer
takes a secret key $sk$ as private input, the user’s private input is a verification key $vk$ and
a message $m$. The signer outputs $d = 1$ if the interaction completes successfully and $d = 0$
otherwise, while the user outputs a signature $\sigma$ if it terminates correctly, and $\bot$ otherwise.
\item $Ver(vk, m, \sigma) \to 0/1$: the (deterministic) verification algorithm, on input a verification key
$vk$, a message $m$ and a signature $\sigma$, outputs 1 if $\sigma$ is valid on $m$ under  $vk$ and 0 otherwise.
\end{itemize}

For a 2-round protocol the interaction $\langle Sign(sk, prd),User(vk, prd, m) \rangle \to (d, \sigma)$ can be realized by the following algorithms:
\begin{align*}
(txt_{U,0},st_{U,0}) \gets& User_0(vk, prd, m)\\
(txt_{S,1},st_{S}) \gets& Sign_1(sk, prd, txt_{U,0}) \\
(txt_{U,1},st_{U,1}) \gets& User_1(st_{U,0}, txt_{S,1})\\
(txt_{S,2}, d) \gets& Sign_2(st_S, txt_{U,1}) \\
\sigma \gets& User_2(st_{U,1}, txt_{S,2})
\end{align*}
We write $(d, \sigma) \gets  \langle Sign(sk, prd),User(vk, prd, m) \rangle$  as shorthand for the above sequence.



\begin{definition}. A WBPS scheme $\WBPSP$ satisfies weak blindness if for all
adversaries $\A$:

$$Adv^{\mathsf{BLD}}_{\WBPSP,\mathcal{A}}(\lambda) := \Pr[
\BLD^{\A,1}_{\WBPSP}(\lambda) - \BLD^{\A,0}_{\WBPSP}(\lambda)]
$$

is negligible in $\lambda$.
\end{definition}

\begin{figure}[ht]
    \fbox{

 \begin{minipage}[t]{0.55\linewidth}
$\BLD^{\A,b}_{\WBPSP}(\lambda)$
\hrule

$par \gets Setup(1^\lambda) $\\
$(m_0,m_1,prd,vk',st_\A) \gets \A_1(par)$\\
\textbf{if} $P(prd,m_0) = 0$ \textbf{or} $P(prd,m_1) = 0$
\textbf{then} \quad \textbf{return} 0\\
{\color{red}{$m_r \gets \{0,1\}^\lambda $}}\\
$m \gets m_b{\color{red}||m_r}$\\
$vk \gets (par,vk')$
$sess\gets \mathsf{init}$\\
$b^*\gets\A_2^{\mathsf{ChalUser}}(st_\A)$\\
$ \textbf{return} \quad  b=b^* $
\end{minipage}%
\hspace{0.5cm}%
\begin{minipage}[t]{0.55\linewidth}
$\mathsf{ChalUser}(msg=\emptyset)$
\hrule





\textbf{if} $sess=\mathsf{await}$\\
\null\quad $sess\gets\mathsf{closed}$\\
\null\quad $\sigma \gets User_2(st_u, msg)$



\textbf{if} $sess=\mathsf{closed}$\\
\null\quad $msg' \gets Ver(vk, m, \sigma)$

\textbf{if} $sess=\mathsf{open}$\\
\null\quad $sess\gets\mathsf{await}$\\
\null\quad $(msg',st_u) \gets User_1(st_u, msg)$


\textbf{if} $sess=\mathsf{init}$\\
$\null\quad sess\gets\mathsf{open}$\\
$\null\quad (msg',st_u) \gets User_0(vk, prd, m)$





\textbf{return} $msg'$

\end{minipage}
}
    \caption{Weak Blindness}
\end{figure}

\begin{figure}[ht]
    \fbox{

 \begin{minipage}[t]{0.55\linewidth}
$\textsf{CMA}^{\A,b}_{\WBPSP}(\lambda)$
\hrule

$par \gets Setup(1^\lambda) $\\
$ (sk, vk) \gets KeyGen(par) $\\
$Q \gets 0$\\
$S \gets 0$\\
$P \gets 0$\\
$(\vec{m^*},{\vec\sigma^*}, \vec{prd^*}) \gets \A^{SigInit,SigComplete}(vk)$\\
$n \gets \left |{\vec{m^*}}\right |$\\
$ \textbf{if} \quad \prod_i Ver(vk,m_i^*,\sigma_i^*) \neq 1 \textbf{ return 0} $\\
$ \textbf{if} \quad \prod_i prd_i^*(m_i^*) \neq 1 \textbf{ return 0} $\\
$ \textbf{if} \quad \exists i, j: (i\neq j) \land (m_i^*,\sigma_i^*)=(m_j^*,\sigma_j^*)\\ \quad \textbf{ return 0} $\\
$ \textbf{if} \quad  n > Q \textbf{ return 1}$\\
\textbf{if} $\nexists \rho \in Perm(S):$\\
$\quad \forall i \leq n: (prd_{\rho(i)}(m_i)=1) \land (st_{\rho(i)}=\bot)$\\
\textbf{then} \quad \textbf{return 1} \\
\quad \textbf{return 0} \\

\end{minipage}%
\hspace{0.5cm}%
\begin{minipage}[t]{0.55\linewidth}
$\mathsf{SigInit}(prd, msg_{U,0})$
\hrule

%$(msg_{U,0},st_{U,0}) \gets User_0(vk, prd, m)$\\
$S\gets S+1; P\gets P+1$\\
$(msg,st_{S}) \gets Sign_1(sk, prd, msg_{U,0})$ \\
$prd_S \gets prd$\\
\textbf{return} $msg$\\

$\mathsf{SigComplete}(s,msg_{U,1})$
\hrule


\textbf{If } $s>S$ \textbf{or} $st_s=\bot$ \textbf{ then } \textbf { return } $\bot$\\
%$(msg_{U,1},st_{U,1}) \gets User_1(st_{U,0}, msg_{S,1})$ \\
$(msg, d) \gets Sign_2(st_s, msg_{U,1})$ \\
\textbf{If } $d=1$ \textbf{ then } $Q\gets Q+1$\\
$P \gets P-1$\\
$st_S \gets \bot$\\
\textbf{return} $msg$

%Todo: change OSig1 / OSig2

\end{minipage}
}
    \caption{Chosen Message Unforgeability Experiment}
\end{figure}


\subsection{Our Scheme}
\begin{figure}[ht]
%\centering

\fbox{%
  \begin{minipage}{0.95\textwidth}
    % Top minipage
    
    \begin{minipage}[t]{0.05\textwidth}
    \end{minipage}\hfill
    \begin{minipage}[t]{0.4\textwidth}
    
 \textbf{\uline{\bm{$\WBPS.\Setup$}}}\\[0.5ex]
$(\G,g,q) \gets \GGen(1^\lambda)\\$
$\hash\gets  \HGen(q)\\$
$sp \gets (\G,g,q,\hash)\\$
$(crs,\tau)\gets \NArg.\Setup(sp)\\$
$(ek,dk)\gets \PKE.\Keygen(1^\lambda)\\$
$par \gets (crs,ek)\\$
$\textbf{return } par\\$
    \end{minipage}
    \hfill
    \begin{minipage}[t]{0.30\textwidth}
      \textbf{\uline{\bm{$\WBPS.\Keygen (par)$}}}\\[0.5ex]
$(\G,g,q) :\subseteq par\\$
$x\gets \Z_q; X\gets g^x \\$
$sk\gets(par,x)\\ vk\gets (par,X)\\$
$\textbf{return } (sk,vk)\\$\\


%\vspace{.5cm}
       \textbf{\uline{\bm{$\WBPS.\Ver(vk,m,\sigma)$}}}\\[0.5ex]
$(\G,g,q,\hash,X):\subseteq vk$\\
$(R,s)\gets \sigma$\\
$c\gets \hash(R,X,m)$\\
$\textbf{return } (g^s=R\cdot X^c)$

    \end{minipage}
    \hfill
    \begin{minipage}[t]{0.1\textwidth}
    \end{minipage}

    \vspace{2em}

    % Bottom minipage with invisible table
    \begin{center}
      \begin{tabular}{>{\raggedright\arraybackslash}p{0.3\textwidth} p{0.15\textwidth} >{\raggedright\arraybackslash}p{0.3\textwidth}}
        \textbf{\uline{\bm{$\WBPS.\Sign(sk,prd)$}}} & & \textbf{\uline{\bm{$\WBPS.\Request(vk,prd,m)$}}} \\
        $(\G,g,q,crs,ek,x):\subseteq sk$ &  &  $(\G,g,q,\hash,crs,ek,X):\subseteq vk$ \\
         & $\xleftarrow{\quad C \quad}$  & $C\gets \PKE.\Enc(ek,m;\rho)$ \\
         $r\gets \Z_q; R\gets g^r $& $\xrightarrow{\quad R \quad}$ &$c\gets \hash(R,X,m)$\\
         && $ \theta \gets (X,R,c,C,prd,ek)$\\
         && $w\gets (m,\rho)$\\
         & $\xleftarrow{\quad c,\pi \quad}$ & $\pi \gets \NArg.\Prove(crs,\theta,w)$\\
         \textbf{if } $\NArg.\Ver(crs,\theta,\pi) = 0:$ &&\\
         \textbf{return } $0$ &&\\
         $s\gets (r+c\cdot x) \mod q$ & $\xrightarrow{\quad s \quad}$ & $\textbf{if }g^s \neq R\cdot X^c : \textbf{return } \bot$ \\
          \textbf{return } $1$ && $ \textbf{return } \sigma\gets(R,s)$
      \end{tabular} 
    \end{center}
  \end{minipage}%
}

\caption{The weakly blind predicate Schnorr signature scheme \WBPSP}
\end{figure}



\subsection{Security}

Blindness (initial sketch): If the commitments are perfectly hiding, then the first round does not reveal anything to the adversary. Likewise, the proof  can be simulated with no contradiction (there will always exist a correct witness to proceed with). However, the oracle query cannot be adversarially controlled: $(R,X,m_0)$ and $(R,X,m_1)$ are both known to the adversary. We thus need to replace $m\gets m_b$ with a weaker form.

Blindness (revised sketch): We replace $m\gets m_b$ with the weaker form $m\gets m_b||m_r$. The commitments are hiding, so  the first round does not reveal anything to the adversary. Likewise, the proof  can be simulated with no contradiction (there will always exist a correct witness to proceed with). Finally, the oracle result $c$ cannot be predicted. It is either  $c=(R,X,m_0||m_r)$ or $c=(R,X,m_1||m_r)$, but neither value can be obtained without correctly guessing the value of $m_r$ (Event $H$, which only happens with negligible probability). Conditioning on $\not H$, $c$ is independent of the challenge bit $b$.

Unforgreability (sketch): We follow F\&W: we extract the message from $tx_A$ and use that to preclude any soundness failures. Further, we use the value we learn to ``sign'' ahead of time, so the $R$ value is known to the simulator. We then distinguish between Adversaries who offer forgeries for incomplete sessions or forgeries for ``new" items. In the first case we let the reduction know the signing key and use it to calculate the DLOG of the $R$ value used in the forgery. In the second, we reduce to unforgeability of standard Schnorr: we know the message contained in each session via decrypting, and we can guarantee that at least one message output in the forgery has not been queried to the signing oracle.
.


\section{Analysis}

\textbf{Asset and Non-Asset Compensation.} In our model, we have so far assumed that an SP performs its
services in exchange for a fee. This fee may be either flat, or calculated on the basis of transaction size,
or complexity of transaction specification, etc. - the details of this fee calculation are dependent on the
specifics of the implementation of our protocol, the marketplace, and SPs preferences.
The fee may also be specified in either an amount of primary asset tokens,
or some other user-defined tokens. A fee requirement in user-defined tokens could be useful in the case
of the SP service for LCs being associated with another type of service trading in such tokens, such as a
videogame token marketplace. Yet another option for SPs is to request non-monetary compensation
for their services, such as requiring engagement with a specific smart contract (i.e. SP does DApp fee
sponsorship), voting for a specific update,
delegating stake, etc. All these actions can be performed by the very transaction that SP constructed
according to LC's specification.


\section{Implementation}
To assess the efficiency of our construction we implement and benchmark the NArg component. Given today's implementations of zk-SNARKs, running the NArg is expected to be the most time and resource intensive part of our assisted TX creation protocol.

We base our tests on the trusted-setup zk-SNARK system \emph{Groth16}~\cite{cryptoeprint:2016/260} implemented by \emph{Iden3}~\cite{circom}.
The circuits we construct and benchmark are inspired by the implementation of~\cite{blindsigs} and written in the domain-specific language of Circom~2.1.

The NArg we instanciate captures proof $\pi$ : $\fun{chkSpec}(\var{int_{post}},\var{tx}) = 1$ $\land \var{Com}(\var{tx};\rho)$ from the protocol in Section~\ref{}.
\begin{equation}
\label{narg}
\mathrm{R}_{\mathrm{WBPS}}
	(\underbrace{(q, \mathbb{G}, G, \mathsf{H})}_{\text{parameters}\ par},\
	\overbrace{(X_{\mathrm{client}}, R, com_\mathrm{tx}, \var{tx}_A, c, \var{int_{post}} )}^{\text{known statement}\ \theta}\ ,\
	 \underbrace{(\var{tx}||\var{nt}, \rho)}_{\text{witness}\ \omega}\ )
\end{equation}

We measure the arithmetic complexity of the relation in~\ref{narg} in terms of number of contraints, proving key size and proof size.
 We also measure the time it takes to creat the resulting circuit, the proving time and the proof verification time.
 The results are summarized in Table~\ref{table_results} for different scenarios we describe in the following. The experiments were executed on commodity hardware based on an Intel(R) Core(TM) i7-8750H CPU operating at 2.20GHz with 12 cores and 16GB of RAM.


\subsection{Circuit construction framework}

\subsection{Benchmarks for different scenarios}

\begin{table}[h!]
\centering
\caption{Different scenarios for Bitcoin and Cardano. Predicate blindness and partial blindness.}
\label{table_results}
\begin{tabular}{@{} lcccccc @{}}
\toprule
  & \multicolumn{2}{c}{\textbf{Bitcoin}} & & & \multicolumn{2}{c}{\textbf{Cardano}} \\
\midrule
Signature scheme & \multicolumn{2}{c}{Schnorr} & & & \multicolumn{2}{c}{EdDSA} \\
Curve        & \multicolumn{2}{c}{Secp256k1} & & & \multicolumn{2}{c}{Ed25519}           \\
Hash          & \multicolumn{2}{c}{SHA-256} & & & \multicolumn{2}{c}{SHA-512}          \\
\midrule
\multirow{2}{*}{Blindness type}     & predicate & partial & & & predicate & partial \\
 & $TX.out \le t$ & xzy blinded & & & $TX.out \le t$  & 333b blinded\\
Transaction size & 256B & xzy B & & & 285B & xzy B\\
\midrule
Proving key size & 115 MB & & & & & 116 MB\\
Proving key verification time & 19.3 s & & & & & 20.5 s \\
Verification key size & 3.3 kB & & & & & 93 kB\\
Proving time & 5.2 s & & & & & 5.8 s\\
Proof size & 804 B & & & & & 805 B\\
Proof verification time & 0.6s & & & & & 0.6s\\
Number of constraints & 228k & & & & & 245k\\
\bottomrule
\end{tabular}
\end{table}

We note that message being signed in our relation is the hash of the transaction $\mathsf{H}(tx)$.
Thus the witness--strictly speaking--only contains the transaction $tx$, but we list $\mathsf{H}(tx)$, for completeness as the message being signed is $m \equiv \mathsf{H}(tx)$.

%\todobox{Incorporate $\rho$. If we add enough entropy using the meta/text field then we don't necessarily need $\rho$? Looks like we do need $\rho$ to be consistent with proof. Overhead of encryption is minimal too.}

The parameters $(q, \mathbb{G}, G, \mathsf{H})$ can be interpreted as the global parameters for the Schnorr signature scheme used.
We assume that our WBPS scheme is meant to extend an existing Schnorr signature scheme, in which case $(q, \mathbb{G}, G, \mathsf{H})$ are predefined, or, the parameters can be chosen to fit the underlying NIZK/SNARK system.

We use BN254 as the curve for \emph{Groth16} to rely on. The relation is instantiated over an arithmetic circuit with modulus of 254 bits, i.e., the order of the group given by  BN254 has 254 bits.

We encode the message that is to be signed in the Baby JubJub field spanned by


We test and benchmark (1) partially blind and (2) predicate blind signatures for both, Bitcoin and Cardano.

Partial blindness can be interpreted as a predicate and hence our scheme allows to combine partial blindness and other predicates.

As the Narg can prove a combination of preciates our scheme is expressive in the sense that complex templates and abstract transaction can be defined.

The most obvious choice is to blind/redact the UTxO references serving as the inputs to the transaction preventing a light client from constructing the transaction on its own, without the inclusion of the output that represents the tip/reimbursement to the service provider. If the light client posts the modified transaction on chain and thereby circumvents the service provider, no tip is due and the light client is able to obtain information that constitutes a valid transaction for free.

\todobox{add results and benchmarks}

\section{Conclusion}



%%
%% Bibliography
%%

%% Please use bibtex, 

\bibliography{lcbib}


\appendix
\section{Expanded Technical Backgroud}
\label{app:techbg}

\subsection{Discrete Log Groups}
\begin{definition} A group generator \GGen{} is a probabilistic polynomial time (p.p.t.) algorithm with input a security parameter $\lambda$ and outputs a group description $\G,g,q$ such that $\G$ is a group of prime order $q\approx 2^\lambda$ with generator $g$. We say that the discrete logarithm problem is hard w.r.t. \GGen{} if for all p.p.t $\A$ we have that

$$\Pr[(\G,g,q) \gets \GGen(1^\lambda); t\gets \Z_q; h\gets g^t: t=A(\G,g,q,h) ] \mbox{ is negligible in }\lambda.$$
\end{definition}

\subsection{ Public Key encryption and Schnorr Signatures}

A public key encryption scheme $\PKE$ comprises a set of polynomial time algorithms $\Keygen,\Enc,\Dec$ with the following syntax:
\begin{itemize}
\item $\Keygen(1^\lambda) \to (ek,dk)$. Creates a public/private keypair $ek,dk$. The encryption key $ek$ also defines the message space $\mathcal{M}$
\item $\Enc(ek,m;\rho) \to C$. Encrypts a message $m\in \mathcal{M}$ under the public encryption key $ek$. \item $\Dec(dk,C) \to m$. Decrypts a ciphertext $C$ using the private decryption key $dk$.
\end{itemize}

\begin{definition} A public key encryption scheme $\PKE$ is \emph{correct} if 
$$\Pr[ (ek,dk) \gets \PKE.\Keygen(1^\lambda); m\gets \mathcal{M}:\PKE.\Dec(sk,\PKE.\Enc(pk,m))=m ] =1.$$
\end{definition}

\begin{definition} A public key encryption scheme $\PKE$ is \emph{IND-CPA secure} if for all stateful p.p.t. adversaries \A, the difference
$$\left| \Pr[ (ek,dk) \gets \PKE.\Keygen(1^\lambda); (m_0,m_1)\gets A(ek);d\gets\{0,1\}; c^*\gets \PKE.\Enc(ek,m_d):A(c^*)=d ] - {1\over2} \right|$$
is negligible in $\lambda$.
\end{definition}

\begin{definition} The Schnorr digital signature scheme $\SDS$ is defined for a group generator $\GGen$ and a hash function generator $\HGen$ and operates as shown on Figure \ref{fig:schnorr}.\end{definition}


\begin{figure}[ht]
\fbox{%
  \begin{minipage}{0.95\textwidth}
    % Top minipage
    
    \begin{minipage}[t]{0.05\textwidth}
    \end{minipage}\hfill
    \begin{minipage}[t]{0.4\textwidth}
    
 \textbf{\uline{\bm{$\SDS.\Setup(1^\lambda)$}}}\\[0.5ex]
\mbox{$(\G,g,q) \gets \GGen(1^\lambda)$}\\
\mbox{$\hash\gets  \HGen(q)$}\\
\mbox{$sp \gets (\G,g,q,\hash)$}\\
$\textbf{return } sp\\$


      \textbf{\uline{\bm{$\SDS.\Sign (sk,m)$}}}\\[0.5ex]
\mbox{$(\G,g,q,\hash,x):\subseteq sk$}\\
\mbox{$r\gets \Z_q;\ R\gets g^x ;\ c\gets \hash(R,X,m)$}\\
\mbox{$s\gets (r+c\cdot x) \mod q$}\\
\mbox{$\textbf{return } \sigma \gets(R,s)$}\\
    \end{minipage}
    \hfill
    \begin{minipage}[t]{0.40\textwidth}
      \textbf{\uline{\bm{$\SDS.\Keygen (sp)$}}}\\[0.5ex]
\mbox{$(\G,g,q) :\subseteq sp$}\\
\mbox{$x\gets \Z_q;\ X\gets g^x$}\\
\mbox{$sk,vk \gets (par,x),(par,X)$}\\
$\textbf{return } (sk,vk)\\$\\


%\vspace{.5cm}
       \textbf{\uline{\bm{$\SDS.\Ver(vk,m,\sigma)$}}}\\[0.5ex]
$(\G,g,q,\hash,X):\subseteq vk$\\
$(R,s)\gets \sigma; c\gets \hash(R,X,m)$\\
$\textbf{return } (g^s=R\cdot X^c)$

    \end{minipage}
    \hfill
    \begin{minipage}[t]{0.1\textwidth}
    \end{minipage}

  \end{minipage}%
}

\caption{The  Schnorr signature scheme \SDS, with group and hash generators \GGen,   \HGen.}\label{fig:schnorr}

\end{figure}


\begin{definition} A digital signature scheme $\DS$ is \emph{correct} if 
$$\Pr[sp\gets \DS.\Setup(1^\lambda); (sk,vk) \gets \DS.\Keygen(sp); m\gets \mathcal{M_{S}}:\DS.\Ver(vk,m,Sign(sk,m))=1 ] =1.$$
\end{definition}

\begin{definition} A digital signature scheme $\DS$ is \emph{strongly existentially unforgeable against chosen message attacks}  (sEUF-CMA)  if for all p.p.t.  adversaries \A, we have  
$$\Pr[\textsf{sEUF-CMA}^{\A}_{\DS}(1^\lambda)] $$
is negligible in $\lambda$ where the game $\textsf{GsEUF-CMA}^{\A}_{\DS}$ is defined in Figure \ref{fig:seufcma}. \label{def:seufcma}
\end{definition}


\begin{figure}[ht]
\fbox{%
  \begin{minipage}{0.95\textwidth}
    % Top minipage
    
    \begin{minipage}[t]{0.05\textwidth}
    \end{minipage}\hfill
    \begin{minipage}[t]{0.5\textwidth}

      \textbf{\uline{\bm{$\textsf{GsEUF-CMA}^{\A}_{\DS}(1^\lambda)$}}}\\[0.5ex]
$sp \gets \DS.\Setup(1^\lambda); Q\gets \emptyset$\\
$(sk,vk)\gets \DS.\Keygen(sp)\\ (m^*,\sigma^*)\gets \A^{\textsf{OSig}}(vk)$\\
$\textbf{return } (m^*,\sigma^*)\neq Q \land \DS.\Ver(vk,m^*,\sigma^*)$
    \end{minipage}
    \hfill
    \begin{minipage}[t]{0.35\textwidth}
      \textbf{\uline{\bm{$\textsf{OSig}(m)$}}}\\[0.5ex]
$\sigma \gets \DS.\Sign(skk,m)$\\
$Q \gets Q\cup \{(m,s)\}$\\
$\textbf{return } \sigma$
    \end{minipage}
    \hfill
    \begin{minipage}[t]{0.1\textwidth}
    \end{minipage}

  \end{minipage}%
}

\caption{The security experiment $\textsf{sEUF-CMA}^{\A}_{\DS}$ and supporting oracle $\textsf{OSig}$.}\label{fig:seufcma}

\end{figure}

\subsection{Parametrized Non-Interactive Zero Knowledge Arguments}

Following \cite{blindsigs}, we define non-interactive zero knowledge arguments with regards to \emph parametrized polynomial relations $\mathcal{P}: \{0,1\}^* \times \{0,1\}^* \times \{0,1\}^*  \to \{0,1\}$, where the first argument represents a parameter set $par$ (e.g. a group description). Given a value of $par$, we say that $w$ is a witness for statement $\theta$ if $\mathcal{P}(par,\theta,w)=1$, i.e. $R=R_{par}(\theta,w):=\mathcal{P}(par,\theta,w)$ is an NP-relation, and $\Lang=\Lang_{par}$ is an NP-language. A NIZK for a relation  $\mathcal{P}$ operates as follows:
\begin{itemize}

\item $\mathsf{Rel}(1^\lambda) \to par$. Generates a parameter set $par$ that defines $R$ and $\Lang$.
\item $\Setup(par)\to (\crs,\tau)$. Generates a common reference string (CRS) and trapdoor $\tau$ used by the simulator. We assume the CRS contains a description of $R$.
\item $\Prove(\crs,\theta,w)\to \pi$. Given a CRS $\crs$, statement $\theta$ and witness $w$ for $\theta$, produces a proof $\pi$.  
\item $\Ver(\crs,\theta,\pi)\to \{0,1\}$. Given a CRS $\crs$, a statement $\theta$ and a  proof $\pi$ outputs 1 or 0, accepting or rejecting the proof.
\item $\SimProve(\crs,\theta,\tau)$. Given a CRS $\crs$, statement $\theta$ and traproor $\tau$ for $\crs$, produces a simulated proof $\pi$.
\end{itemize}


\begin{definition} A system $\NArgr$ is perfectly correct if for all unbounded adversaries $\A$ 
\begin{align*}
\Pr [par \gets \NArg.\mathsf{Rel}(1^\lambda);  (\crs,\tau)\gets \NArg.\Setup(par); (\theta,w) \gets A(crs):  \\
\neg{}R(\theta,w) \lor \NArg.\Ver(\crs,\theta,\NArg.\Prove(\crs,\theta,w))]=1 \\
\end{align*}
\end{definition}

\begin{definition} A system $\NArgr$ is adaptably computationally sound if for all p.p.t. adversaries $\A$ 
\begin{align*}
\Pr [par \gets \NArg.\mathsf{Rel}(1^\lambda);  (\crs,\tau)\gets \NArg.\Setup(par); (\theta,\pi) \gets A(crs):  \\
\neg{}L(\theta) \land {}\NArg.\Ver(\crs,\theta,\pi)] \mbox{ is negligible in $\lambda$.}
\end{align*}
\end{definition}

\begin{definition} A system $\NArgr$ is  computationally zero-knowledge if for all p.p.t. adversaries $\A$ 
\begin{align*}
|\Pr [par \gets \NArg.\mathsf{Rel}(1^\lambda);  (\crs,\tau)\gets \NArg.\Setup(par); d\gets\{0,1\}:  \\
d = A^{\mathsf{OProve}_d}(crs)] - {1\over2}|\mbox{ is negligible in $\lambda$, where}\\
\mathsf{OProve}_0(\theta,w):= \textbf{if } \neg R(\theta,w) \textbf{ return }\bot; \textbf{ return } \NArg.\Prove(\theta,w), \mbox{and} \\
\mathsf{OProve}_1(\theta,w):= \textbf{if } \neg R(\theta,w) \textbf{ return }\bot; \textbf{ return } \NArg.\SimProve(\theta,\tau).
  \end{align*}
\end{definition}

\newpage

\section{Additional Ledger Syntax}
\label{sec:appendix}

In Figure~\ref{fig:notation:nonstandard} we introduce the standard ledger syntax that we use throughout.

\begin{figure}[h!tb]
  \begin{align*}
    \H{}
    & =~\bigcup_{n=0}^{\infty}\{0,1\}^{8n}
    & \mbox{the type of bytestrings }
    \\
    (a, b)
    & :~\Interval{A}
    & \text{intervals over a totally-ordered set $A$}
    \\
    \var{Key} \mapsto \var{Value}
    & \subseteq \{~ k \mapsto v ~\mid~ k \in \var{Key},~v \in \var{Value}~ \}
    & \text{finite map with unique keys}
    \\
    [a1 ; ...; ak]
    & :~[C]
    & \text{finite list with terms of type $C$}
    \\
    h :: t
    & :~[C]
    & \text{list with head $h$ and tail $t$}
    \\
    x \cup \fun{nothing}
    & :~A^?
    & \text{maybe type over $A$}
    \\
    a ~\{ ~\fun{field} = ~x~\}
    & :~A
    & \text{record of type A with $\fun{field}$ changed to $x$}
  \end{align*}
  \caption{Notation}
  \label{fig:notation:nonstandard}
\end{figure}

Figure~\ref{fig:eutxo-types} lists the primitives and derived types
that comprise the foundations of the EUTxO model,
along with some ancillary definitions.
(Outputs normally refer to transaction IDs by hash,
but we simplify here for clarity.)

\begin{figure}[h!tb]

\textsc{Ledger primitives}
\begin{displaymath}
\begin{array}{rlll}
  \checkSig &:&
  \Tx \to \pubkey \to \H \to \B \\
  & &\mbox{\emph{checks that a given key signed a transaction}}
\end{array}
\end{displaymath}

\textsc{Helper functions}
    \begin{align*}
        &\fun{txid} : \Tx \to \TxId \\
        &\fun{txid}~\var{tx} = \fun{hash}~(\var{tx} ~\{ ~\fun{sigKeys} = ~\fun{dom}~(\var{tx} . \sigs) ,~\sigs = ~\emptyset~\})
        \nextdef
        &\fun{toMap} : \N \to [\TxOut] \to (\N \mapsto \TxOut) \\
        &\fun{toMap}(\_,~[~]) ~~~~~~~~~~~~~= [~] \\
        &\fun{toMap}(\var{ix},~u~::~\var{outs}) = \{~\var{ix}\mapsto u~\} \cup \fun{toMap}(\var{ix}+1,~\var{outs}) \\
        & \mbox{\emph{constructs a map from a list of outputs}}
        \nextdef
        &\fun{mkOuts} : \Tx \to \UTxO \\
        &\fun{mkOuts}(tx) = \{~(\var{tx},~\var{ix}) \mapsto o~ \mid~(\var{ix} \mapsto o)\in~\fun{toMap}(0,~\var{tx}.\outputs)~\} \\
        &\mbox{\emph{constructs a UTxO set from a list of outputs of a given transaction}}
        \nextdef
        &\fun{MOf} : \N \to \N \to (A \to \B) \to [A] \to \B \\
        &\fun{MOf}~k~m~f~[~] = m~\leq~k \\
        &\fun{MOf}~k~m~f~( h~ ::~ t) = \fun{if}~ (m~\leq~k)~\fun{then}~\true~\fun{else}~(\fun{MOf}~(k~+~a)~m~f~t) \\
        &~~\where~~a~=~\fun{if}~ (f~(h))~\fun{then}~1~\fun{else}~0 \\
        &\mbox{\emph{returns $\true$ if enough elements of a list satisfy given function}}
    \end{align*}
\caption{Primitives and basic types for the $\UTxO_{ma}$ model}
\label{fig:eutxo-types}
\end{figure}
\newpage 
\subsection{Transaction Validation Rules}

A transaction $\var{tx}$ is \emph{valid} if it follows the following rules. 

\label{sec:check-tx}


\begin{itemize}
    \item[(i)] \textbf{The transaction has at least one input:}
  \[
  \var{tx}.\inputs~\neq~\{\}
  \]
    \item[(ii)] \textbf{The current slot is within transaction validity interval:}
  \[
  \var{slot} \in \var{tx}.\fun{validityInterval}
  \]
    \item[(iii)] \textbf{All outputs have positive values:}
  \[
  \forall o \in \var{tx}.\outputs,~\var{o}.\val > \emptyset
  \]
    \item[(iv)] \textbf{All output references of transaction inputs exist in the UTxO:}
  \[
  \var{tx}.\inputs~ \subseteq~ \fun{dom}~\var{utxo}
  \]
    \item[(v)] \textbf{Value is preserved:}
  \[
  \var{tx}.\mint +
  \sum_{i \in~\var{tx}.\inputs,~(i~\mapsto~o) \in~\var{utxo}} \var{o}.\val =
  \sum_{o \in~\var{tx}.\outputs} \var{o}.\val ~+~\fun{toValue}~(\var{tx}.\fun{fee})
  \]
    \item[(vii)] \textbf{All inputs validate:}
  \[
  \forall~ i \in \var{tx}.\inputs,~i \mapsto (s, v) \in \var{utxo},~
     \applyScript{s}(\fun{dom}~(\var{tx}.\sigs),~\var{tx}.\validityInterval) = \true
  \]
    \item[(ix)] \textbf{All minting scripts validate:}
  \[
  \forall~ p\mapsto \_ \in \var{tx}.\mint,~ \applyScript{p} (\fun{dom}~(\var{tx}.\sigs),~\var{tx}.\validityInterval) = \true
  \]
    \item[(x)] \textbf{All signatures are correct:}
  \[
  \forall~ (pk \mapsto s) \in \var{tx}.\sigs,~ \checkSig (\var{tx}, pk, s) = \true
  \]
  \item[(i)] \textbf{The fee is sufficient:}
  \[
    \var{s}.\fun{minfee} \leq \var{tx}.\fun{fee} 
  \]
  \end{itemize}

\subsection{Script construction and Evaluation}
\label{sec:check-script}

In Figure \ref{fig:script} we present the constructors and evaluation rules for scripts, and in Figure \ref{fig:mktospec} we explain the transaction building function \fun{mkToSpec}.

\begin{figure}
    \textsc{Constructors of $\Script$} 
\begin{displaymath}
\begin{array}{rlll}
    \fun{RequireMOf}         &: \N \to [\Script] &\to~ \Script \\
    \fun{RequireSig}         &: \pubkey      &\to~ \Script \\
    \fun{RequireTimeStart}   &: \Slot        &\to~ \Script \\
    \fun{RequireTimeExpire}  &: \Slot        &\to~ \Script \\
\end{array}
\end{displaymath}
\nextdef
\textsc{Evaluation of $\Script$} 
\begin{displaymath}
\begin{array}{rlll}
  \applyScript{\_} &:& \Script \to ((\type{Set} \pubkey) \times (Slot \times Slot)) \to \B \\
  \applyScript{\fun{RequireMOf}~n~ls} (\var{khs}, (t1, t2)) &=&  \fun{MOf}~0~ n ~(\applyScript{\_}~(\var{khs}, (t1, t2))) ~ls \\
  \applyScript{\fun{RequireSig}~k} (\var{khs}, (t1, t2)) &=& k~\in~\var{khs} \\
  \applyScript{\fun{RequireTimeStart}~t1'} (\var{khs}, (t1, t2)) &=& t1'~\leq~t1 \\
  \applyScript{\fun{RequireTimeExpire}~t2'} (\var{khs}, (t1, t2)) &=& t2~\leq~t2' 
\end{array}
\end{displaymath}
\caption{$\Script$ constructors and evaluation}
\label{fig:script}
\end{figure}

\begin{figure}
      \textsc{Trivial transaction $\fun{initTx}_{a,mf}$} 
\begin{displaymath}
\begin{array}{rlll}
  \fun{initTx}_{a,mf}  &=& \{ ~\inputs = \emptyset,\\
  & &\ \outputs = [ (a , \fun{tip})],\\
  & &\ \fun{validityInterval} = [\fun{nothing} , \fun{nothing}],\\
  & &\ \mint = 0,\\
  & &\ \fun{fee} = mf \\
  & &\ \fun{aux} = []~\\
  & &\ \sigs = \emptyset~ \}
\end{array}
\end{displaymath}
\nextdef
\textsc{Input selection function} 
\begin{displaymath}
\begin{array}{rlll}
  \fun{mkIns}  &:& (\LState \times (\type{Set} \TxIn) \times \Value \times \Script) \to (\type{Set} \TxIn) \\
  \fun{mkIns} ~(l,~i,~v,~s) &=& \fun{if}~\neg~(v \leq 0) ~ \fun{then} \\
  & & ~~~~ \fun{if}~(v > 0)~\fun{then}~\\
  & & ~~~~ ~~~~\fun{mkIns}~(l~\setminus~(\{j \mapsto \_\}),~i \cup j,~v~-~(u(j) . \val) ,~s) \\
  & & ~~~~ \fun{else}~i\\
  & & \fun{else } \\
  & & ~~~~\fun{nothing} \\
  & & \fun{where}~\\
  & & ~~~j = \fun{pickInput}~l~s \\
  & & ~~~u = \fun{utxo}~l
\end{array}
\end{displaymath}
\nextdef
\textsc{Auxiliary $\fun{mkToSpec'}$ definition} 
\begin{displaymath}
\begin{array}{rlll}
\fun{mkToSpec'} &:& \LState \to \mathcal{I}_\mathsf{post} \to \Tx \to \Tx^? \\
\fun{mkToSpec'}~l~(\fun{MustMint}~v) (\var{tx}) &=&  \var{tx}~\{ \mint~=~v ~+~ \var{tx} . \mint , \fun{sigs} = tx.\fun{sigs} \cup \fun{getSigsVal}~v ,\\
& & ~~~~\fun{validityInterval} = \fun{restrictIntervalVal}~tx.\fun{validityInterval}~v\} \\
\fun{mkToSpec'}~l~(\fun{SpendFrom}~s) (\var{tx}) &=& \var{tx}~\{ \outputs~=~ \var{tx} . \inputs~\\
& & ~~~~\cup~\fun{newIns} , \fun{sigs} = tx.\fun{sigs} \cup \fun{getSigsUTxO}~\fun{newIns}~l ,\\ 
& & ~~~~ \fun{validityInterval} = \\ 
& &~~~~~~~~\fun{restrictIntervalUTxO}~tx.\fun{validityInterval}~\fun{newIns}~l\} \\
& & ~~~~\where \\
& & ~~~~\fun{newIns} = \fun{mkIns}~l~(\var{tx} . \inputs)~(\fun{produced} - \fun{consumed})~ s \\
\fun{mkToSpec'}~l~ (\fun{MaxInterval}~i) (\var{tx}) &=& \var{tx}~\{ \validityInterval~=~(l . \fun{slot}, \\
& & ~~~~\fun{min}~\{l . \fun{slot}~+~i , \var{tx} . \fun{validityInterval}_2)\} \}  \\
\fun{mkToSpec'}~l~ (\fun{PayTo}~(s, v)) (\var{tx}) &=& \var{tx}~\{ \outputs~=~ \var{tx} . \outputs~\cup~ (s, v) \}   \\
\fun{mkToSpec'}~l~ (\fun{ChangeTo}~s) (\var{tx}) &=& \fun{if}~ \fun{consumed} - \fun{produced} ~>~0 ~\\ 
& & ~~~~\fun{then}~ \var{tx}~\{ \outputs~=~ \var{tx} . \outputs~\\
& & ~~~~\cup~ \{(s,~ \fun{consumed} - \fun{produced})\} \} ~\fun{else}~\fun{nothing}   \\
\fun{mkToSpec'}~l~ (\fun{MaxFee}~f)~ (\var{tx}) &=&  \fun{if}~ \var{tx} . \fun{fee} \leq f~\\ 
& & ~~~~\fun{then}~\var{tx}~\fun{else}~\fun{nothing} \\
\fun{mkToSpec'}~l~ (\fun{AndExps}~[a1 ; a2 ; ... ; ak]) (\var{tx}) &=& \fun{mkToSpec'}~l~ak~(... ~(\fun{mkToSpec'}~l~a2~(\fun{mkToSpec'}~l~a1~\var{tx})))
\end{array}
\end{displaymath}
\nextdef 
\textsc{$\fun{mkToSpec}$ definition} 
\begin{displaymath}
\begin{array}{rlll}
\fun{mkToSpec} &:& (\LState \times \mathcal{I}_\mathsf{post}) \to \Tx^? \\
\fun{mkToSpec}~(l , i) &=& \fun{mkToSpec'}~l~i~\fun{initTx}_{a,mf}
\end{array}
\end{displaymath}
\caption{Building transactions according to the specific intent}
\label{fig:mktospec}
\end{figure}




\begin{figure}
\begin{displaymath}
\begin{array}{rlll}
  \fun{txi}_1 &=&(\outputs = \{~(s, t)~\},\\
  & &\ \fun{validityInterval} = [ \fun{slot}~l , (\fun{slot}~l) + j ],\\
  & &\ \mint = t,\\
  & &\ \fun{fee} = \fun{minfee}~l \\
  & &\ \fun{sigKeys} = \fun{getSignersVal}~t) 
  \nextdef 
  \fun{txi}_2 &=& (\outputs = \{ (\fun{RequireSig}~k2,~x)~,~(\fun{RequireSig}~k1~,\\& & ~(\fun{balance}~(\fun{mkIns}~l~\{\}~x~ (\fun{RequireSig}~k1))) - x) \} ,\\
  & &\ \fun{validityInterval}  = [~ \fun{nothing}~ , \fun{nothing}~],\\
  & &\ \mint  = \{~\} ,\\
  & &\ \fun{fee}  = \fun{minfee}~l \\
  & &\ \fun{sigKeys}  = \{~k1~\} )
\end{array}
\end{displaymath}
\caption{Abstract transaction examples}
\label{fig:txs}
\end{figure}



\end{document}
