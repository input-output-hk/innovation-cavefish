% !TEX root = ../lc_main.tex
\section{Background and Related Work}

\textbf{Light clients and off-chain payments.}
Our approach facilitates transaction submission for a light client that is not aware of the history of a blockchain.
Therefore, we compare our work with existing solutions for light clients as well as mechanisms that allow the interaction with a blockchain when in resource constrained operation.
Our approach is different from a more ``traditional'' (i.e., what commonly is referred to) light client that has the primary goal of syncing to the blockchain in order to acquire the information necessary to interact with a smart contract or to submit a transaction.
We summarize the most important concepts and works in the area of light clients below.
We first describe the common idea of a light client and then outline concepts related to our work that complement light clients.

There is existing work analyzing light client functionality \cite{committee} \cite{soklc}. The majority of light client designs include the following main functionalities it is expected to perform: (1) issue queries, such as for the balance of an account, or the state of a transaction, and (2) safeguard secret information and submit transactions to the blockchain. 
In order to implement these functionalities, light clients use several generic techniques, most notably: header verification and consensus evolution verification. Unlike a \emph{full node}, a light client \emph{only verifies the headers of blocks}, and skips the verification of transactions and account balances \cite{committee}. This technique was made popular with SPV in Bitcoin~\cite{bitcoin} and nearly universally adopted by most practical approaches for light clients, e.g.,~\cite{ethereum}.
Because the validator set can change, consensus evolution verification is needed for blockchains running on proof-of-stake.
Another common technique is to compress the blockchain and/or ledger state in order to reduce the information a light client need for functioning reliably (see, e.g.,~\cite{flyclient}).

Some light clients \cite{lazy} use game-theoretic assumptions, e.g. to implement the slashing of previously deposited collateral in case of misbehavior, for example~\cite{superlight}.
The main cryptographic building blocks that are used to realize those techniques are succinct representation and proofs, such as data accumulators (often Merkle trees) and commitments and SNARKs.
Suitable signatures and hash functions are needed for certain light client designs. Examples of these include aggregate signatures and threshold signatures \cite{mithril} \cite{committee}. Yet other designs propose building networks for identity-based light payments that are not fully untrusted \cite{celo}.

\textbf{Intents}.
We also consider \emph{intents} and \emph{solver networks} to be related work. These designs attempt to establish a relationship between solvers and users via their (light) clients~\cite{ethresearch-solvers}. 
A light client issues an intent via as an abstracted transaction object. Then, solvers process the intent, incentivized by transaction fee or intent execution reward. The users are then free to accept or reject a solver’s proposal. In case the solvers are required to provide a deposit, slashing incentivizes honest behavior of a rational actor.
Since concepts around solver networks are relatively new, there is currently no universal standard governing the specification for intents and abstract transaction objects. To the best of our knowledge, this is the first work describing an intent-based light client protocol taking advantage of specific features of the UTxO ledger model that both minimizes communication and offers built-in incentive structure. 

\textbf{Cross-chain intents.}
A lot of work on intents applies is focused on building cross-chain intent resolution, which 
requires more sophisticated protocols that are able to communicate with 
multiple chains at once \cite{celer} .
Some protocol designs choose to resolve intents on-chain directly \cite{khalani}, while 
other offer both layer-1 and layer-2 options \cite{anoma}. Yet another designs is a consensus protocol
relying on on minimal trust assumptions while conforming to the interledger standard \cite{ripple}. 
Focusing on a single chain allows 
us to construct a simple low-communication protocol which makes use of the unique features of the 
underlying blockchain, and requires reduced little communication (only 
between two parties). In its simplicity, our protocol is more similar to payment channels. 

\textbf{Payment channels}.
The concept of \emph{payment channels}, and the related notion of \emph{state channels}, also bear similarities
to our approach.
Payment channels are a type of off-chain mechanism for blockchains. Payment channels allow users to establish a private payment
channel between two parties, e.g. \cite{lightning} \cite{raiden} \cite{ethereum}. 
A payment channel can be used to conduct a series of transactions without interacting
with the main blockchain. The creation of a payment channel requires first locking funds in an on-chain smart contract ~\cite{payment-channel}. 
Some channels, such as Hydra \cite{hydra}, not only allow
simple payments between users, but can simulate the majority of the on-chain transaction processing mechanism internally 
that are within the user base and liquidity limits of the channel.
This type of channel is usually called a \emph{state channel}. 

Generally, interactions in state and payment channels are limited to a pre-determined set of participants, liquidity, and 
types of actions users can perform. \footnote{Payment networks attempt to address this issue
by using multi-path payments, where the original payment is split into smaller parts,
each routed through different channels, allowing for greater liquidity, adding complexity and risk}
The limits of our proposed design to support 
different actions is dictated by the language used for request specification, which may be augmented in future work. 
At the same time, our design provides the same level of atomicity expected from payment channels, 
i.e., payments are either fully successful or completely fail.

After the initial on-chain setup has been completed, payment and state channels
are capable of greatly reducing both transaction processing times and fees.
The target user base for our light client design, however, are users
that either do not wish to (or are not capable of) engaging in establishing relationships with other chain users, participating
in an on-chain setup process, or intend to make repeat or scheduled interactions with the same
users. For this reason, the on-chain space- and cost-saving benefits of payment channels,
are unlikely to be accessible to a user base which
we target with our low-commitment design.


\textbf{API and Explorer Services}.
Many realistic UTxO ledger implementations (e.g. Ergo \footnote{\url{https://ergoplatform.org/}},
Cardano \footnote{\url{https://cardano.org/}}, and BitCoin \footnote{\url{https://bitcoin.org/}}) are set up
in a way that an invalid transaction will, in most cases, not result in any update to the ledger
state, getting rejected instead. This makes
services such as blockchain explorers \footnote{\url{https://beta.explorer.cardano.org/}} or Blockfrost (API
as a service for accessing the Cardano blockchain) \footnote{\url{https://blockfrost.dev/}} some of the
strongest competitors with our proposal, as they provide the data needed
for the LC to construct their transaction, often with a relatively high degree of reliability.
Such services currently do not appear to have the infrastructure required to perform the construction of transactions
for the light client. Along with providing access to such a service, 
the design we propose also includes a compensation structure for the service provider. 
Revenue from explorer services services such as those above is
either ad-based, requires users to create an account paid for with fiat currency, or is free in order to
promote use of the specific service/blockchain. We, on the other hand, propose a protocol 
in which service providers are compensated
in assets on the same blockchain as the one to which their transaction gets applied, and the payment structure
consists of a one-time, no-setup atomic exchange of assets for services. 

\subsection{Hierarchical Deterministic wallets and address discovery}
UTXO based blockchains, including Bitcoin and Cardano use the concept of Hierarchical Deterministic (HD) wallets \cite{bip32} where child addresses can be created in a deterministic way using a public key of a keypair $sk,pk$, and some short auxiliary data (the chain code). Given a public key $pk$, chain code $c$ and requested index $i$ for a child key, a hash function is used to produce a scalar $s_i$ enablind the derivation of the child key as $sk_i=sk+s_i$. It is also possible to use $s_i$ without knowing $sk$ to directly derive the corresponding public key $pk_i=pk\cdot g^{s_i}$. The same method is used to produce a child chain code $c_i$, so that multiple generations of keys are possible.

In our application, we can assume that a wallet can give the SPO a single public key and chain code, letting the SPO do address derivation and lookup on their side, using some agreed-upon bounds, or heuristics on the index depth. This keeps the communication cost from the light client to the SPO constant.


\subsection{Schnorr Blind Signatures}
As soon as their patent expired in 2008, Schnorr signatures~\cite{10.1007/0-387-34805-0_22} have been gaining significant adoption,
replacing RSA in many scenarios due to their smaller size and faster verification.
Compared to (EC)DSA, Schnorr offers similar efficiency and guarantees,
but (EC)DSA security proofs are most likely not possible without 
strong idealization~\cite{cryptoeprint:2023/914}.
As a consequence, EdDSA~\cite{cryptoeprint:2011/368}, a popular variant of the
 Schnorr scheme, is currently under consideration by NIST for standardization.
The security of Schnorr signaatures is based on the discrete 
logarithm assumption~\cite{jofc-2000-14276},
with proofs in the random oracle model (ROM) as well as in 
the algebraic group model (AGM) and the ROM~\cite{10.1007/978-3-030-45724-2_3}.

Standard Schnorr signatures are now widely used in blockchains such as Bitcoin, Bitcoin Cash, Litecoin, and Polkadot.
Monero, Zcash, and Cardano use the EdDSA variant.
The adption of Schnorr and EdDSA signatures is driven by privacy~\cite{cryptoeprint:2022/1636} and
 scalability gains~\cite{10.1007/s10623-019-00608-x}
but also the straightforward extension to \emph{blind Schnorr signatures}~\cite{10.1007/3-540-48071-4_7}.
Most Schnorr-based blind signature schemes
require multiple rounds~\cite{10.1007/3-540-48071-4_7}, which can make the protocol suceptible to denial-of-service attacks.
Research has thus been focusing on making blind signatures \emph{concurrently} secure where more than one session can be intertwined~\cite{}. % as well as round-optimal schemes~\cite{}.

Our construction hinges on the the concurrently secure blind and partially blind signing protocol
for standard Schnorr signatures found in~\cite{blindsigs} by Fuchsbauer, et al.---the first work stating rigorous security guarantees for a practical blind signature scheme based on Schnorr signatures.
Unlike schemes that had been presented before,~\cite{blindsigs} is not vulnerable to the attacks described in~\cite{10.1007/3-540-45708-9_19,cryptoeprint:2020/945} which showed that the hardness-assumption\footnote{The ROS (Random inhomogeneities in a Overdetermined
Solvable system of linear equations) problem was initially studied by Schnorr in~\cite{10.1007/3-540-45600-7_1}} existing schemes relied on for concurrent security can be solved in polynomial-time.
As our light client protocol is based on~\cite{blindsigs}, it also provides partial and predicate blindness, two properties that allow us to describe a transaction in an abstract way featuring ``redacted'' parts.
We briefly outline the construction of Fuchsbauer et al.~\cite{blindsigs} in the following.
The scheme is equivalent to the blind signature scheme~\cite{10.1007/3-540-48071-4_7} by Chaum and Pedersen but, in order to make the protocol concurrently secure, adds a commitment phase at the beginning where the user sends an encrypted version of the message $m$ and blinding values $(\alpha,\beta)$ to the signer.
In addition to that, the second message by the user includes a zero-knowledge proof alongside the Schnorr challenge $c$.
The zero-knowledge proof asserts to the signer that the initial (encrypted) commitment and $c$ have been derived from the same $m, \alpha, \beta$.

In addition to obtining concurrent security, one can leverage the zero-knowledge proof to assert additional facts, most notably, the user can prove to the signer that certain predicate(s) over message $m$ holds. This effectively turns the fully blind scheme into a predicate blind scheme.
Furthermore, support for predicate blindness implies partial blindness~\cite{blindsigs} since constructing a predicate that checks equality for parts of $m$ can be used to assert to the user that parts of the message correspond to the expected value(s).

\todobox{
Compare our approach  with :
\begin{itemize}
    \item "Free" websites monitoring the chain -- mention they lack long-term sustainability. mention big APIs. blockfrost, etc.
    \item Bridges (trustless and trusted). only overview. maybe SOK paper. and describe how LC and bridge are related.
    \item Payment channels (require upfront capital) + Payment networks~\cite{cryptoeprint:2017/823}
    \item LCs that operate on single-prover model (eg. with an established relationship via deposit)
    \item LCs that operate on multi-prover model
	%\item Time lock micro payment channel~\cite{{10.1007/978-3-319-21741-3_1}
\end{itemize}}


